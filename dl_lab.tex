% *********************************************************
% Digital Logic
% *********************************************************

\RequirePackage{fix-cm} % fix some latex issues see: http://texdoc.net/texmf-dist/doc/latex/base/fixltx2e.pdf
\documentclass[ twoside,openright,titlepage,numbers=noenddot,headinclude,
                footinclude=true,cleardoublepage=empty,abstractoff, 
                BCOR=5mm,fontsize=11pt,american]{scrreprt}

% *********************************************************
% Note: Make all your adjustments in here
% *********************************************************
% *********************************************************
% dl_lab-config.tex 
% *********************************************************

% *********************************************************
% Set the encoding of your files.
% *********************************************************
\PassOptionsToPackage{utf8}{inputenc}
	\usepackage{inputenc}

% *********************************************************
% Remove "drafting" below to deactivate the time-stamp on the pages
% *********************************************************
\PassOptionsToPackage{eulerchapternumbers,listings,%drafting,
  pdfspacing,floatperchapter,%linedheaders,%
  subfig,beramono,eulermath,parts}{classicthesis}
% *********************************************************
% Available options 
% (see ClassicThesis.pdf for more information):
% drafting
% parts nochapters linedheaders
% eulerchapternumbers beramono eulermath pdfspacing minionprospacing
% tocaligned dottedtoc manychapters
% listings floatperchapter subfig
% *********************************************************

% *********************************************************
% 2. Personal data and user ad-hoc commands
% *********************************************************
\newcommand{\myTitle}{Logisim-Evoluation Lab Manual\xspace}
\newcommand{\mySubtitle}{}
\newcommand{\myDegree}{}
\newcommand{\myName}{George Self\xspace}
\newcommand{\myProf}{}
\newcommand{\myOtherProf}{}
\newcommand{\mySupervisor}{}
\newcommand{\myFaculty}{}
\newcommand{\myDepartment}{Computer Information Systems\xspace}
\newcommand{\myUni}{Cochise College\xspace}
\newcommand{\myLocation}{Arizona\xspace}
\newcommand{\myTime}{July 2019\xspace}
\newcommand{\myVersion}{Edition 4.0\xspace}

% *********************************************************
% Setup, finetuning, and useful commands
% *********************************************************
\newcounter{dummy} % necessary for correct links to index, bib, etc.
\newlength{\abcd} % for ab..z string length calculation
\providecommand{\mLyX}{L\kern-.1667em\lower.25em\hbox{Y}\kern-.125emX\@}
\newcommand{\ie}{i.\,e.}
\newcommand{\Ie}{I.\,e.}
\newcommand{\eg}{e.\,g.}
\newcommand{\Eg}{E.\,g.}
\newcommand{\LE}{\textit{Logisim-Evolution} }
\newcommand{\blankpage}{ % Create a blank page at the end of the document
  \newpage
  \thispagestyle{empty}
  \mbox{}
  \newpage
  }
% The following creates a function named maxwidth that permits
% me to set a maximum width for images. If the natural width of
% the image is less than maxwidth then the image is rendered at
% its natural size, else scaled to maxwidth.
\makeatletter
\def\maxwidth#1{\ifdim\Gin@nat@width>#1 #1\else\Gin@nat@width\fi}
\makeatother

% *********************************************************
% 3. Loading some handy packages
% *********************************************************

% *********************************************************
% Packages with options that might require adjustments
% *********************************************************
\PassOptionsToPackage{american}{babel}   % change this to your language(s)
  \usepackage{babel}                  

\usepackage{csquotes}
%\PassOptionsToPackage{%
%    %backend=biber, %instead of bibtex
%  backend=bibtex8,bibencoding=ascii,%
%  language=auto,%
%  style=numeric-comp,%
%  %style=authoryear-comp, % Author 1999, 2010
%  %bibstyle=authoryear,dashed=false, % dashed: substitute rep. author with ---
%  sorting=nyt, % name, year, title
%  maxbibnames=10, % default: 3, et al.
%  %backref=true,%
%  natbib=true % natbib compatibility mode (\citep and \citet still work)
%}{biblatex}
%  \usepackage{biblatex}

\PassOptionsToPackage{fleqn}{amsmath}       % math environments and more by the AMS 
  \usepackage{amsmath}

% *********************************************************
% General useful packages
% *********************************************************
\PassOptionsToPackage{T1}{fontenc} % T2A for cyrillics
  \usepackage{fontenc}     
\usepackage{textcomp} % fix warning with missing font shapes
\renewcommand{\textuparrow}{$\uparrow$}
\renewcommand{\textdownarrow}{$\downarrow$}

\usepackage{scrhack} % fix warnings when using KOMA with listings package
\usepackage{xspace} % to get the spacing after macros right  
\usepackage{mparhack} % get marginpar right
\usepackage{fixltx2e} % fixes some LaTeX stuff --> since 2015 in the LaTeX kernel (see below)
%\usepackage[latest]{latexrelease} % will be used once available in more distributions (ISSUE #107)
\PassOptionsToPackage{printonlyused,smaller}{acronym} 
  \usepackage{acronym} % nice macros for handling all acronyms in the thesis
%\renewcommand{\bflabel}[1]{{#1}\hfill} % fix the list of acronyms --> no longer working
%\renewcommand*{\acsfont}[1]{\textsc{#1}} 
\renewcommand*{\aclabelfont}[1]{\acsfont{#1}}
\usepackage[paperheight=11in,paperwidth=8.5in]{geometry}
\usepackage{shorttoc} % generate brief version of the table of contents

% *********************************************************
% 4. Setup floats: tables, (sub)figures, and captions
% *********************************************************
\usepackage{tabularx} % better tables
\setlength{\extrarowheight}{3pt} % increase table row height
\newcommand{\tableheadline}[1]{\multicolumn{1}{c}{\spacedlowsmallcaps{#1}}}
\newcommand{\myfloatalign}{\centering} % to be used with each float for alignment
\usepackage{caption}
\captionsetup{font=small}
\usepackage{subfig}  

% *********************************************************
% 5. Setup code listings - format Verilog listings
% *********************************************************
\usepackage{listings} 
%\lstset{emph={trueIndex,root},emphstyle=\color{BlueViolet}}%\underbar} % for special keywords
\lstset{language=Verilog,%[LaTeX]Tex,
  morekeywords={PassOptionsToPackage,selectlanguage},
  keywordstyle=\color{RoyalBlue},%\bfseries,
  %basicstyle=\small\ttfamily,
  basicstyle=\ttfamily,
  identifierstyle=\color{DarkRed},
  commentstyle=\color{Green}\ttfamily,
  stringstyle=\rmfamily,
  numbers=left,
  numberstyle=\scriptsize,%\tiny
  stepnumber=2,
  numbersep=8pt,
  showstringspaces=false,
  breaklines=true,
  %frameround=ftff,
  frame=lines,
  captionpos=b,  % put captions at the bottom of the listing
  aboveskip=.75\baselineskip,
  belowskip=.75\baselineskip
  %abovecaptionskip=.75\baselineskip
  %belowcaptionskip=.75\baselineskip
  %frame=L
} 

% *********************************************************
% 6. PDFLaTeX, hyperreferences and citation backreferences
% *********************************************************

% *********************************************************
% Using PDFLaTeX
% *********************************************************
\PassOptionsToPackage{pdftex,hyperfootnotes=false,pdfpagelabels}{hyperref}
  \usepackage{hyperref}  % backref linktocpage pagebackref
\pdfcompresslevel=9
\pdfadjustspacing=1 
\PassOptionsToPackage{pdftex}{graphicx}
  \usepackage{graphicx} 
 
% *********************************************************
% Hyperreferences in the PDF output
% *********************************************************
\hypersetup{%
  %draft, % = no hyperlinking at all (useful in b/w printouts)
  colorlinks=true, linktocpage=true, pdfstartpage=3, pdfstartview=FitV,%
  % uncomment the following line if you want to have black links (e.g., for printing)
  %colorlinks=false, linktocpage=false, pdfstartpage=3, pdfstartview=FitV, pdfborder={0 0 0},%
  breaklinks=true, pdfpagemode=UseNone, pageanchor=true, pdfpagemode=UseOutlines,%
  plainpages=false, bookmarksnumbered, bookmarksopen=true, bookmarksopenlevel=1,%
  hypertexnames=true, pdfhighlight=/O,%nesting=true,%frenchlinks,%
  urlcolor=webbrown, linkcolor=RoyalBlue, citecolor=webgreen, %pagecolor=RoyalBlue,%
  %urlcolor=Black, linkcolor=Black, citecolor=Black, %pagecolor=Black,%
  pdftitle={\myTitle},%
  pdfauthor={\textcopyright\ \myName, \myUni, \myFaculty},%
  pdfsubject={},%
  pdfkeywords={},%
  pdfcreator={pdfLaTeX},%
  pdfproducer={LaTeX with hyperref and classicthesis}%
}   

% *********************************************************
% Setup autoreferences
% *********************************************************
\makeatletter
\@ifpackageloaded{babel}%
  {
    \addto\extrasamerican{%
    \renewcommand*{\figureautorefname}{Figure}%
    \renewcommand*{\tableautorefname}{Table}%
    \renewcommand*{\partautorefname}{Part}%
    \renewcommand*{\chapterautorefname}{Chapter}%
    \renewcommand*{\sectionautorefname}{Section}%
    \renewcommand*{\subsectionautorefname}{Section}%
    \renewcommand*{\subsubsectionautorefname}{Section}%
  }%
  \addto\extrasngerman{% 
    \renewcommand*{\paragraphautorefname}{Absatz}%
    \renewcommand*{\subparagraphautorefname}{Unterabsatz}%
    \renewcommand*{\footnoteautorefname}{Fu\"snote}%
    \renewcommand*{\FancyVerbLineautorefname}{Zeile}%
    \renewcommand*{\theoremautorefname}{Theorem}%
    \renewcommand*{\appendixautorefname}{Anhang}%
    \renewcommand*{\equationautorefname}{Gleichung}%        
    \renewcommand*{\itemautorefname}{Punkt}%
  }%  
  \providecommand{\subfigureautorefname}{\figureautorefname}%
}{\relax}
\makeatother

% *********************************************************
% Setup drawing environment
% *********************************************************
\usepackage[svgnames,table]{xcolor}
\usepackage{tikz}
\usetikzlibrary{circuits.logic.US,circuits.logic.IEC,circuits.ee.IEC,shapes.geometric}
\usepackage{circuitikz}
\usepackage{tikz-timing} % Timing Diagrams
\usetikztiminglibrary[new={char=Q,reset char=R}]{counters}
\usepackage{rotating} % Rotate an image
\usetikzlibrary{calc} % Do some math in tikz
\usepackage{smartdiagram} % draw ``smart'' diagrams the easy way


% *********************************************************
% This enables enumerated lists using first, second, etc.
% *********************************************************
\usepackage{moreenum}

% *********************************************************
% Used to create framed paragraphs, like "interest boxes"
% *********************************************************
\usepackage{tcolorbox}

% *********************************************************
% Used to enable strike-through text
% *********************************************************
\usepackage[normalem]{ulem}

% *********************************************************
% The Creative Commons License Package
% *********************************************************
\usepackage[type={CC},modifier={zero},version={1.0}]{doclicense}

% *********************************************************
% This package permits me to align a table column on a decimal point
% *********************************************************
\usepackage{siunitx}

% *********************************************************
% Enable certain special chars in verbatim, like underlines
% *********************************************************
\usepackage{fancyvrb}

% *********************************************************
% Used for wrapping figures
% *********************************************************
\usepackage{float}
\usepackage{wrapfig}
\restylefloat{figure}

% *********************************************************
% Used for merging cells in tables, creating a slash in a cell, 
% adjusting the width of a table to fit the text column
% *********************************************************
\usepackage{multicol}
\usepackage{multirow}
\usepackage{slashbox}
\usepackage{adjustbox}

% *********************************************************
% Permit URLs to line-break at any letter or a slash
% *********************************************************
\usepackage{url}
\renewcommand{\UrlBreaks}{\do\/\do\a\do\b\do\c\do\d\do\e\do\f\do\g\do\h\do\i\do\j\do\k\do\l\do\m\do\n\do\o\do\p\do\q\do\r\do\s\do\t\do\u\do\v\do\w\do\x\do\y\do\z\do\A\do\B\do\C\do\D\do\E\do\F\do\G\do\H\do\I\do\J\do\K\do\L\do\M\do\N\do\O\do\P\do\Q\do\R\do\S\do\T\do\U\do\V\do\W\do\X\do\Y\do\Z}

% *********************************************************
% Generate  lipsum
% *********************************************************
\usepackage{lipsum}

% *********************************************************
% Give myself some enumerate options
% *********************************************************
\usepackage{enumitem}

% *********************************************************
% I used this to just find the width of a line in cm (so I can create
% appropriately sized graphics). Load the package here and then copy the
% other line in a separate paragraph where you need the size displayed.
% *********************************************************
\usepackage{layouts}
%textwidth in cm: \printinunitsof{cm}\prntlen{\textwidth}

% *********************************************************
% 7. Last calls before the bar closes
% *********************************************************

% *********************************************************
% Development Stuff
% *********************************************************
\listfiles
%\PassOptionsToPackage{l2tabu,orthodox,abort}{nag}
%   \usepackage{nag}
%\PassOptionsToPackage{warning, all}{onlyamsmath}
%   \usepackage{onlyamsmath}

% *********************************************************
% Last, but not least...
% *********************************************************
\usepackage{classicthesis} 



% *********************************************************
% Hyphenation
% *********************************************************
%\hyphenation{put special hyphenation here}

% *********************************************************
% Begin Document
% *********************************************************
\begin{document}
\frenchspacing
\raggedbottom
\selectlanguage{american} 
\pagenumbering{roman}
\pagestyle{plain}
% *********************************************************
% Frontmatter
% *********************************************************
\include{FrontBackmatter/Titlepage}  % >>>>> Include <<<<<
\thispagestyle{empty}

\hfill

\vfill

\noindent\myName: \emph{\myTitle,} \mySubtitle %\myDegree, 
%\textcopyright\ \myTime

\doclicenseThis

%\bigskip
%
%\noindent\spacedlowsmallcaps{Supervisors}: \\
%\myProf \\
%\myOtherProf \\ 
%\mySupervisor
%
%\medskip
%
%\noindent\spacedlowsmallcaps{Location}: \\
%\myLocation
%
%\medskip
%
%\noindent\spacedlowsmallcaps{Time Frame}: \\
%\myTime
  % >>>>> Include <<<<<
%*****************************************
\chapter*{Preface}\label{preface}
%*****************************************

I have taught CIS 221, \textit{Digital Logic}, for Cochise College since about 2003 and enjoy working with students on this topic. From the start, I wanted students to work with labs as part of our studies and actually design circuits to complement our theoretical instruction. As I evaluated circuit design software I had three criteria:

\begin{itemize}
  \item \textbf{\ac{OER}}. It is important to me that students use software that is available free of charge and is supported by the entire web community. 
  \item \textbf{Platform}. While most of my students use a Windows-based system, some use Macintosh and it was important to me to use software that is available for both of those platforms. As a bonus, most OER software is also available for the Linux system, though I'm not aware of any of my students who are using Linux.
  \item \textbf{Simplicity}. I wanted to use software that was easy to master so students could spend their time understanding digital logic rather than learning the arcane structures of a simulation language.
\end{itemize}

I originally wrote a number of lab exercises using \textit{Logisim}, but the creator of that software, Carl Burch, announced that he would quit developing it in 2014. Because it was published as an open source project, a group of Swiss institutes started with the \textit{Logisim} software and developed a new version that integrated several new tools, like a chronogram, and released it under the name \LE.

It is my hope that students will find these labs instructive and the labs enhance their learning of digital logic. This lab manual is written with \LaTeX\ and published under a \href{https://creativecommons.org/publicdomain/zero/1.0/}{Creative Commons Zero} license with a goal that other instructors can modify it to meet their own needs. The source code can be found at \href{https://github.com/grself/CIS221_Lab_Manual}{my personal GITHUB page} and I always welcome comments that will help me improve this manual.

\bigskip
\begin{flushright}
  \textemdash  George Self
\end{flushright}


   % >>>>> Include <<<<<
\pagestyle{scrheadings}              % >>>>> Include <<<<<
\cleardoublepage%*******************************************************
% Brief Table of Contents
%*******************************************************
%\phantomsection
\refstepcounter{dummy}
%\pdfbookmark[1]{\briefcontentsname}{brieftableofcontents}
%\setcounter{tocdepth}{0} % <-- 2 includes up to subsections in the ToC
%\setcounter{secnumdepth}{0} % <-- 3 numbers up to subsubsections
%\manualmark
%\markboth{\spacedlowsmallcaps{\contentsname}}{\spacedlowsmallcaps{\contentsname}}
\shorttoc{Brief Contents}{0} % Only Chapter Headings 

%*******************************************************
% Table of Contents
%*******************************************************
%\phantomsection
\refstepcounter{dummy}
\pdfbookmark[1]{\contentsname}{tableofcontents}
\setcounter{tocdepth}{2} % <-- 2 includes up to subsections in the ToC
\setcounter{secnumdepth}{3} % <-- 3 numbers up to subsubsections
\manualmark
\markboth{\spacedlowsmallcaps{\contentsname}}{\spacedlowsmallcaps{\contentsname}}
\tableofcontents 
\automark[section]{chapter}
\renewcommand{\chaptermark}[1]{\markboth{\spacedlowsmallcaps{#1}}{\spacedlowsmallcaps{#1}}}
\renewcommand{\sectionmark}[1]{\markright{\thesection\enspace\spacedlowsmallcaps{#1}}}
%*******************************************************
% List of Figures and of the Tables
%*******************************************************
\clearpage

\begingroup 
    \let\clearpage\relax
    \let\cleardoublepage\relax
    \let\cleardoublepage\relax
    %*******************************************************
    % List of Figures
    %*******************************************************    
    %\phantomsection 
    \refstepcounter{dummy}
    \addcontentsline{toc}{chapter}{\listfigurename}
    \pdfbookmark[1]{\listfigurename}{lof}
    \listoffigures

    \vspace{8ex}

    %*******************************************************
    % List of Tables
    %*******************************************************
    %\phantomsection 
    \refstepcounter{dummy}
    \addcontentsline{toc}{chapter}{\listtablename}
    \pdfbookmark[1]{\listtablename}{lot}
    \listoftables
        
    \vspace{8ex}
%   \newpage
    
    %*******************************************************
    % List of Listings
    %*******************************************************      
      %\phantomsection 
    \refstepcounter{dummy}
    \addcontentsline{toc}{chapter}{\lstlistlistingname}
    \pdfbookmark[1]{\lstlistlistingname}{lol}
    \lstlistoflistings 

    \vspace{8ex}
       
    %*******************************************************
    % Acronyms
    %*******************************************************
    %\phantomsection 
    \refstepcounter{dummy}
    \pdfbookmark[1]{Acronyms}{acronyms}
    \markboth{\spacedlowsmallcaps{Acronyms}}{\spacedlowsmallcaps{Acronyms}}
    \chapter*{Acronyms}
    % Notes: alphabitize this list as you create it.
    % Accronyms in this list are not printed unless they show up in the text somewhere
    \begin{acronym}[UMLX]
        \acro{ALU}{Arithmetic Logic Unit}
        \acro{ASCII}{American Standard Code for Information Interchange}
        \acro{ASIC}{Application-Specific Integrated Circuit}
        \acro{BCD}{Binary Coded Decimal}
        \acro{BDD}{Binary Decision Diagram}
        \acro{BIOS}{Basic Input/Output System}
        \acro{BLIF}{Berkeley Logic Interchange Format}
        \acro{CAT}{Computer-Aided Tools}
        \acro{CISC}{Complex Instruction Set Computer}
        \acro{CPU}{Central Processing Unit}
        \acro{DRAM}{Dynamic Random Access Memory}
        \acro{DUT}{Device Under Test}
        \acro{EBCDIC}{Extended Binary Coded Decimal Interchange Code}
        \acro{EDA}{Electronic Design Automation}
        \acro{FSM}{Finite State Machine}
        \acro{HDL}{Hardware Description Language}
        \acro{IC}{Integrated Circuit}
        \acro{IEEE}{Institute of Electrical and Electronics Engineers}
        \acro{KARMA}{KARnaugh MAp simplifier}
        \acro{LED}{Light Emitting Diode}
        \acro{LSB}{Least Significant Bit}
        \acro{LSN}{Least Significant Nibble}
        \acro{MSB}{Most Significant Bit}
        \acro{MSN}{Most Significant Nibble}
        \acro{NaN}{Not a Number}
        \acro{OER}{Open Educational Resource}
        \acro{PCB}{Printed Circuit Board}
        \acro{POS}{Product of Sums}
        \acro{PROM}{Programmable Read-Only Memory}
        \acro{RAM}{Random Access Memory}
        \acro{RISC}{Reduced Instruction Set Computer}
        \acro{ROM}{Read Only Memory}
        \acro{RPM}{Rotations Per Minute}
        \acro{RTL}{Register Transfer Language}
        \acro{SECDED}{Single Error Correction, Double Error Detection}
        \acro{SOP}{Sum of Products}
        \acro{SDRAM}{Synchronized Dynamic Random Access Memory}
        \acro{SRAM}{Static Random Access Memory}
        \acro{TTL}{Transistor-Transistor Logic}
        \acro{USB}{Universal Synchronous Bus}
        \acro{VHDL}{VHSIC Hardware Descriptive Language}
        \acro{VHSIC}{Very High Speed Integrated Circuit}
        \acro{VLIW}{Very Long Instruction Word}
    \end{acronym}                     
\endgroup
  % >>>>> Include <<<<<

% *********************************************************
% Part 1: Introduction
% *********************************************************
\cleardoublepage\pagenumbering{arabic}
%\setcounter{page}{90}
% use \cleardoublepage here to avoid problems with pdfbookmark
\cleardoublepage  %  >>>>> Include <<<<<
\ctparttext{\textsc{Logisim-evolution} is used to create and test simulations of digital circuits. This part of the lab manual includes only one lab designed to introduce \textit{Logisim-evolution} and teach the fundamentals of using this application.}
\part{Introduction To Logisim-Evolution}
%\printinunitsof{cm}\prntlen{\textwidth} % Print the width of the text line
%%********************************************
% Lab 01: Introduction to Logisim-evolution
%********************************************

\chapter{Introduction To Logisim-evolution}

%********************************************

% Section: Introduction
%********************************************

\section{Purpose}

This lab introduces the \LE logic simulator, which is used for all lab exercises in this manual. 

%********************************************
% Section: Procedure
%********************************************

\section{Procedure}

\subsection{Installation}

\LE is a Java application, so a Java runtime environment will need to be installed before using the application. Many students who are taking a digital logic class already have a Java runtime on their computer and can skip this step, but those who do not will need to install the Java runtime. That process is not covered in this manual but information about installing the Java runtime environment is available at \url{http://www.oracle.com/technetwork/java/javase/downloads/index.html}. It can be confusing to know which version of Java to download but students working on the labs in this manual only need the runtime, called \textit{JRE} on the website. Students who are also in programming classes will likely already have the runtime as part of the Java Developer's Kit (JDK). It can be tricky testing the Java installation since the Chrome, Firefox, and Edge browsers will not run Java apps, but students can open a command prompt and enter \lstinline|java -version| to see what version of Java their computers are running, if any.

\LE (\url{https://github.com/reds-heig/logisim-evolution}) is available as a free download. Visit the website and about halfway down the page find a section named ``Running logisim-evolution.'' Click the ``here'' link at the end of the first sentence in that section. 

Since the \LE file is a Java application, it does not need to be installed like most software. To start \LE, double-click the \LE shortcut. That will start Java and then run the \LE application. Also, \LE will not need to be uninstalled when it is no longer needed since it is not actually installed, the \LE file can simply be deleted.

\subsection{Beginner's Tutorial}

\LE comes with a beginner's tutorial available in \textsc{Help -> Tutorial}. That tutorial only takes a few minutes and introduces students to the major components of the application. Students should complete that tutorial before starting this lab.

\subsection{Logisim-evolution Workspace}

Start \LE by double-clicking its icon. The initial \LE window will be similar to Figure \ref{fig:01-01}.

\begin{figure}[H]
	\centering
	\includegraphics[width=\maxwidth{.95\linewidth}]{gfx/01-01}
	\caption{Logisim-evolution Initial Screen}
	\label{fig:01-01}
\end{figure}

The \LE space is divided into several areas. Along the top is a text menu that includes the types of selections found in most programs. For example, the ``File'' menu includes items like ``Save'' and ``Exit.'' The ``Edit'' menu includes an ``Undo'' option that is useful. In later labs, the various options under ``Project'' and ``Simulate'' will be described and used. Items in the ``FPGAMenu'' are beyond the scope of this class and will not be used. Of particular importance at this point is ``Library Reference'' in the ``Help'' menu. It contains information about every logical device available in \LE and is very useful while using those components in new circuits. 

Under the menu bar is the Toolbar, which is a row of eight buttons that are the most commonly used tools in \LE: 

\begin{itemize}
	\item \textbf{Pointing Finger}: Used to ``poke'' and change input values while the simulator is running. 
	\item \textbf{Arrow}: Used to select components or wires in order to modify, move, or delete them. 
	\item \textbf{A}: Activates the Text tool so text information can be added to the circuit. 
	\item \textbf{Green Input Port}: Creates an input port for a circuit. 
	\item \textbf{White Output Port}: Creates an output port for a circuit. 
	\item \textbf{NOT Gate}: Creates a NOT gate. 
	\item \textbf{AND Gate}: Creates an AND gate. 
	\item \textbf{OR Gate}: Creates an OR gate. 
\end{itemize}

The Explorer Pane is on the left side of the workspace and contains a folder list. The folders contain ``libraries'' of components organized in a logical manner. For example, the ``Gates'' folder contains various gates (AND, OR, XOR, etc.) that can be used in a circuit. The four icons across the top of the Explorer Pane are used for advanced operations and will be covered as they are needed. 

The Properties panel on the lower left side of the screen is where the properties for any selected component can be read and set. For example, the number of inputs for an AND gate can be set to a specific number.

The drawing canvas is the largest part of the screen. It is where circuits are constructed and simulated. 

\subsection{Simple Multiplexer}

\marginpar{Do not be concerned with the exact placement of components on the drawing canvas. They can be rearranged as the build progresses.}A multiplexer is used to select which of two or more inputs will be connected to a single output. For this lab, a simple two-input, one-bit multiplexer will be built. It is understood that students will not know the significance of a multiplexer at this point in the class, but the purpose of this lab is to use \LE to build a simple circuit and a multiplexer serves that purpose well. 

Start by clicking the \textit{And} button on the toolbar and placing two \texttt{AND} gates on the canvas. The canvas should resemble Figure \ref{fig:01-02}

\begin{figure}[H]
	\centering
	\includegraphics[width=\maxwidth{.95\linewidth}]{gfx/01-02}
	\caption{Two AND Gates}
	\label{fig:01-02}
\end{figure}

Click one of the AND gates to select it and observe the various properties available for that gate, as seen in Figure \ref{fig:01-03}. The default values do not need to be changed for this circuit; however, all circuits in this manual use the ``Narrow'' gate size in order to make the circuit fit the screen better. The other properties will be explained as they are needed.

\begin{figure}[H]
	\centering
	\includegraphics[width=\maxwidth{.95\linewidth}]{gfx/01-03}
	\caption{AND Gate Properties}
	\label{fig:01-03}
\end{figure}

The outputs of the two \texttt{AND} gates need to be combined with an \texttt{OR} gate. Add an \texttt{OR} gate as illustrated in Figure \ref{fig:01-04}.

\begin{figure}[H]
	\centering
	\includegraphics[width=\maxwidth{.95\linewidth}]{gfx/01-04}
	\caption{OR Gate Added to Circuit}
	\label{fig:01-04}
\end{figure}

The top input for the first \texttt{AND} gate needs two \texttt{NOT} gates (inverters) so the two \texttt{AND} gates can function as on/off switches. This is a rather common digital logic construct and when the circuit is complete it will become clear how the switching function works.

\begin{figure}[H]
	\centering
	\includegraphics[width=\maxwidth{.95\linewidth}]{gfx/01-05}
	\caption{Two NOT Gates Added to Circuit}
	\label{fig:01-05}
\end{figure}

All inputs and outputs need to be added as in Figure \ref{fig:01-06}. Note: inputs are square and outputs are round. The \textit{Label} property for each input and output should be specified as in the figure. The pins are labeled according to their function in the circuit. Pin \textit{Sel} carries a signal that selects which input to connect to the output, pins \textit{In1} and \textit{In2} are the two inputs, and pin \textit{Out1} is the output. Note: output pins display a blue-colored \textsf{X} until they are actually wired to some device like the \texttt{OR} gate in the illustration.
 
\begin{figure}[H]
	\centering
	\includegraphics[width=\maxwidth{.95\linewidth}]{gfx/01-06}
	\caption{Inputs and Output Added}
	\label{fig:01-06}
\end{figure}

Finally, connect each device with a wire by clicking on the various ports and dragging a wire to the next port. To start the wire in the middle of the two NOT gates click the wire connecting those gates and drag downward. Wires will automatically ``bend'' one time but to get two bends, like between the output of an \texttt{AND} gate and the input of the \texttt{OR} gate, click-and-drag the wire from the output of the \texttt{AND} gate to a spot a short distance in front of that same gate, then release the mouse button and then immediately click again to start a new wire that will ``bend'' to the input of the \texttt{OR} gate. Only a little practice is needed to master this wiring technique.

\begin{figure}[H]
	\centering
	\includegraphics[width=\maxwidth{.95\linewidth}]{gfx/01-07}
	\caption{Circuit Wiring Added}
	\label{fig:01-07}
\end{figure}

To operate the circuit in a simulator, click the \textit{Pointing Finger} and ``poke'' the various inputs. If it is working properly, when the \textit{Sel} input is high then the value of \textit{In2} should be transmitted to the output, but when \textit{Sel} is low then the value of \textit{In1} should be transmitted to the output. This circuit is used to select one of two inputs to be transmitted to the output.

\subsection{Identifying Information}

Before finishing, add standard identification information near the top left corner of the circuit using the text tool (the \textit{A} button on the toolbar). That information should include the designer's name, the lab number and circuit name, and the date. Standard identification information for this lab would look like this:

\bigskip
% The minipage environment keeps the three lines together - no page break.
\begin{minipage}{\linewidth}
\begin{verbatim}
George Self
Lab 01: 2-Way, 1-Bit multiplexer
February 13, 2018
\end{verbatim}
\end{minipage}
\bigskip

\marginpar{The font properties in Figure \ref{fig:01-08} have been set to bold and a large size to make the text easier to read.}Note that \textit{Logisim-evolution} will automatically center text in a new box, so text boxes will need to be aligned after they have been created. To align the text boxes, click the \textit{Arrow} tool and use it to drag the boxes to their desired location. The completed circuit should look like Figure \ref{fig:01-08}.

\begin{figure}[H]
	\centering
	\includegraphics[width=\maxwidth{.95\linewidth}]{gfx/01-08}
	\caption{Simple multiplexer}
	\label{fig:01-08}
\end{figure}

\section{Deliverable}

The purpose of this lab is to install and test the \textit{Logisim-evolution} system and become comfortable creating a digital logic circuit. 

To receive a grade for this lab, create the Simple Multiplexer as defined in this lab, be sure the standard identifying information is at the top left of the circuit, and then save the file with this name: \textit{\texttt{Lab01\_Mux21}} (that stands for multiplexer, 2-way, 1-bit). Submit that circuit file for grading.

 

% *********************************************************
% Part 2: Foundations
% *********************************************************
% use \cleardoublepage here to avoid problems with pdfbookmark
\cleardoublepage
\ctparttext{\textsc{Foundational Exercises} are designed to provide practice with simple logic circuits in order to both develop skill with \textit{Logisim-evolution} and illustrate the foundations of digital logic.}
\part{Foundations}
%%********************************************
% Lab 02: Converting Boolean Logic to a Circuit
%********************************************
\chapter{Boolean Logic}

\section{Purpose}

This lab has three goals: 

\begin{itemize}
	\item Design circuits when given a Boolean expression.
	\item Create subcircuits.
	\item Create and exercise a test of the subcircuits.
\end{itemize}

\textit{Logisim-evolution} permits designers to work with a main circuit and any number of subcircuits. Students who have studied programming languages are familiar with ``functions'' or ``classes'' that can be designed and built one time and then reused many times whenever they are needed. \textit{Logisim-evolution} permits that same type of modular design by using subcircuits. 

The \textit{Logisim-evolution} starter for this lab includes a \lstinline[columns=fixed]|main| circuit and one subcircuit, named \lstinline[columns=fixed]|Equation_1|. The starter subcircuit is used to practice creating a circuit from a Boolean expression and then a new subcircuit is added and a second Boolean expression is used to build that circuit.

\section{Procedure}

\subsection{Subcircuit: Equation 1}

\marginpar{A magnifying glass icon is used to indicate which circuit is active on the drawing canvas.}The starter circuit includes a subcircuit named \lstinline[columns=fixed]|Equation_1|. Double-click that circuit in the Explorer Pane to activate it. The drawing canvas for this subcircuit is mostly blank except for a Boolean expression: $ (A'BC')+(AB'C')+(ABC) $. Before starting to design a circuit, it is helpful to take a minute to analyze the expression. 

\begin{itemize}
	\item There are only three variables used in the entire expression: \textit{A}, \textit{B}, and \textit{C}. Therefore, there would be three inputs into the circuit.
	\item There are three groups of variables and within each group the variables are joined with an \texttt{AND}. Therefore, the circuit must include three \texttt{AND} gates with three inputs for each gate.
	\item The three groups of variables are joined with an \texttt{OR}. Therefore, the circuit must include an \texttt{OR} gate with three inputs.
	\item While the expression does not name an output variable, it is reasonable to assume that the circuit would output a logic 1 or 0. Therefore, a one-bit output variable must be specified.
\end{itemize}

\marginpar{Do not be concerned with the exact placement of components on the drawing canvas. They can be rearranged as the build progresses.}Start by placing three inputs and an output on the drawing canvas. Inputs are indicated by a square pin found on the tool bar above the drawing canvas. Click that tool and place three input pins named \textit{In1A}, \textit{In1B}, and \textit{In1C} \textemdash that means ``Input for Equation One, variable A'' and so forth. 

Outputs are indicated by a round pin found on the tool bar above the drawing canvas. Click that tool and place an output named \textit{Out1}. The circuit should look like Figure \ref{fig:02-01}.

\begin{figure}[H]
	\centering
	\includegraphics[width=\maxwidth{.95\linewidth}]{gfx/02-01}
	\caption{Equation 1 Inputs-Outputs}
	\label{fig:02-01}
\end{figure}

Next, the gates should be added. The \texttt{AND} gate tool can be found on the tool bar. Click that tool and place three \texttt{AND} gates on the circuit. Click each gate and in its properties panel set the \textit{Number of Inputs} to 3. 

The \texttt{OR} gate tool can be found on the tool bar. Click that tool and place one \texttt{OR} gate on the circuit. Click that gate and in its properties panel set the \textit{Number of Inputs} to 3.

The circuit should look like Figure \ref{fig:02-02}.

\begin{figure}[H]
	\centering
	\includegraphics[width=\maxwidth{.95\linewidth}]{gfx/02-02}
	\caption{Equation 1 And-Or Gates}
	\label{fig:02-02}
\end{figure}

Next, the inputs for the \texttt{AND} gates should be set to match the Boolean expression. The top \texttt{AND} gate will match the first group of inputs, $ (A'BC') $, so inputs \textit{A} and \textit{C} should be negated. To negate those two inputs, click the \texttt{AND} gate and in the properties panel set the \textit{Negate} item for the top and bottom input to ``Yes.'' When that is done, the two inputs on the \texttt{AND} gate should include a small ``negate'' circle.

In the same way, the middle and bottom input for the second \texttt{AND} gate should also be negated. The circuit should look like Figure \ref{fig:02-03}.

\begin{figure}[H]
	\centering
	\includegraphics[width=\maxwidth{.95\linewidth}]{gfx/02-03}
	\caption{Equation 1 And Gate Inputs Set}
	\label{fig:02-03}
\end{figure}

Finally, connect all gates with wires, like Figure \ref{fig:02-04}. 

\begin{figure}[H]
	\centering
	\includegraphics[width=\maxwidth{.95\linewidth}]{gfx/02-04}
	\caption{Equation 1 Circuit Completed}
	\label{fig:02-04}
\end{figure}

Test the circuit by selecting the \textit{poke} tool in the tool bar (it looks like a pointing finger) and setting various combinations of 1 and 0 on the three inputs. The output pin should go high only when the inputs are set to $ (A'BC') $, $ (AB'C') $, or $ (ABC) $.

\subsection{Subcircuit: Equation 2}

A new subcircuit can be added to a circuit by clicking \textsc{Project -> Add Circuit}. Name the new circuit \lstinline[columns=fixed]|Equation_2|. Open the new subcircuit by double-clicking its name in the Explorer Pane. 

Because this is a new subcircuit, the drawing canvas is blank. To start this subcircuit, write the equation for the circuit near the top of the drawing canvas by clicking the ``A'' button on the Toolbar and then clicking near the top of the drawing canvas and typing the following:

\[ (A'B'CD')+(A'BCD)+(AB'CD')+(ABCD') \]

It will save time to take a few minutes and analyze the expression. 

\begin{itemize}
	\item There are only four variables used in the entire expression: \textit{A}, \textit{B}, \textit{C}, and \textit{D}. Therefore, there would be four inputs into the circuit.
	\item There are four groups of variables and within each group the variables are joined with an \texttt{AND}. Therefore, the circuit must include four \texttt{AND} gates with four inputs for each gate.
	\item The four groups of variables are joined with an \texttt{OR}. Therefore, the circuit must include an \texttt{OR} gate with four inputs.
	\item While the expression does not name an output variable, it is reasonable to assume that the circuit would output a logic 1 or 0. Therefore, a one-bit output variable must be specified.
\end{itemize}

Design the subcircuit using these names for the inputs: \textit{In2A}, \textit{In2B}, \textit{In2C}, and \textit{In2D}. Also include an output named \textit{Out2}. Set the \texttt{AND} gates so the their inputs are negated properly and then wire the entire subcircuit. Finally, test the circuit to ensure the output goes high only when the four specified combinations of inputs are present.

\subsection{Main Circuit}

Make the \lstinline[columns=fixed]|main| circuit active by double-clicking its name in the Explorer Panel. Click once on the \lstinline[columns=fixed]|Equation_1| circuit and the cursor will change into an image of that circuit as it will appear on the drawing canvas. Click on the drawing canvas to drop that subcircuit. The circuit can later be moved by clicking it and dragging it to a new location. Wire the three inputs and output as shown in Figure \ref{fig:02-05}. Notice that the input/output pins do not need to be named the same as in the subcircuit; for example, the output for \lstinline[columns=fixed]|Equation_1| is labeled \textit{Out1} but it is connected to an output pin labeled \textit{True1}.

\begin{figure}[H]
	\centering
	\includegraphics[width=\maxwidth{.95\linewidth}]{gfx/02-05}
	\caption{Main Circuit}
	\label{fig:02-05}
\end{figure}

Add the \lstinline[columns=fixed]|Equation_2| circuit in the same way and wire four inputs and one output to that circuit. The inputs should be labeled \textit{A2}, \textit{B2}, \textit{C2}, and \textit{D2} and the output labeled \textit{True2}.

\subsection{Testing the Circuit}

One way to test this circuit is to use the \textit{poke} tool and click various input combinations for both subcircuits. If the subcircuits are correct then the output will only go high when the correct combination is set on the inputs. However, as digital logic circuits become more complex it is important to automate the testing process so no input combinations are overlooked. \textit{Logisim-evolution} includes a \textsc{Simulate -> Test Vector} feature that is used for automating circuit testing.

The first step in using automatic testing is to create a \textit{Test Vector} file. This is a simple \textit{.txt} file that can be created in any text processor, like \textit{Notepad}. \marginpar{Do not use a word processor to create the Test Vector since that would add unneeded codes for things like fonts.} The format for a test vector is fairly simple.

\begin{itemize}
	\item Every line is a single test of the circuit, except the first line.
	\item The first line defines the various inputs and outputs being tested.
	\item Any line that starts with a hash mark (\#) is a comment and is ignored.
\end{itemize}

Following is the test vector file used to test the \lstinline[columns=fixed]|Equation_1| subcircuit.

\begin{Verbatim}[frame=lines,
								 numbers=left,
								 xleftmargin=10mm,
								 xrightmargin=10mm]
# Test vector for Lab 2
# Equation 1
A1 B1 C1 True1
0   0  0     0
0   0  1     0
0   1  0     1
0   1  1     0
1   0  0     1
1   0  1     0
1   1  0     0
1   1  1     1
\end{Verbatim}

Following is an explanation for the \textit{Test vector for Lab 2} file.

\begin{description}
	\item[Line 1] This is just the title of the file. Because this line starts with a hash (\#) it is a comment and will be ignored by \textit{Logisim-evolution}.
	\item[Line 2] This is another descriptor line and is ignored by \textit{Logisim-evolution}.
	\item[Line 3] This line lists all of the inputs and outputs in the circuit under test. In this case, there are three inputs, \textit{A1}, \textit{B1}, and \textit{C1}, along with one output, \textit{True1}. \textit{Logisim-evolution} is able to determine whether the pin is an input or output from its properties. NOTE: each of the inputs and outputs in this circuit are single bits. If an input or output has more than one bit then that must be specified on this line. For example, if \textit{True1} was actually a four-bit output then that pin would be listed as \textit{True1[4]}.
	\item[Line 4] This line contains the first test for the circuit. This line specifies that Logisim-evolution make \textit{A1}, \textit{B1}, and \textit{C1} equal to zero and then check to be certain that \textit{True1} is also zero.
	\item[Other Lines] All other lines set the three input bits and specify the expected response in the output bit.
\end{description}

The test vector for Equation 2 would look like this:

\begin{Verbatim}[frame=lines,
								 numbers=left,
								 xleftmargin=10mm,
								 xrightmargin=10mm]
# Test vector for Lab 2
# Equation 2
A2 B2 C2 D2 True2
0   0  0  0     0
0   0  0  1     0
0   0  1  0     1
0   0  1  1     0
0   1  0  0     0
0   1  0  1     0
0   1  1  0     0
0   1  1  1     1
1   0  0  0     0
1   0  0  1     0
1   0  1  0     1
1   0  1  1     0
1   1  0  0     0
1   1  0  1     0
1   1  1  0     1
1   1  1  1     0
\end{Verbatim}

In practice, a circuit designer would usually not create two different test vectors but would, instead, create just one file to test all parts of the circuit. Combining the \textit{Equation 1} test and the \textit{Equation 2} test is not quite as easy as appending one after the other since all input and output pins for both circuits must be specified at the top of the file. Following is the test vector for a circuit that combines \textit{Equation 1} and \textit{Equation 2}. Notice that all input and output pins are defined on line three then each line beginning with line four tests both of the equation circuits. Because only eight tests are needed to fully exercise \textit{Equation 1} but 16 are needed for Equation 2, the \textit{Equation 1} tests are repeated starting on Line 12.

\begin{Verbatim}[frame=lines,
								 numbers=left,
								 xleftmargin=10mm,
								 xrightmargin=10mm]
# Test vector for Lab 2
# Equation 1   - Equation 2
A1 B1 C1 True1   A2 B2 C2 D2 True2
0   0  0     0    0  0  0  0     0
0   0  1     0    0  0  0  1     0
0   1  0     1    0  0  1  0     1
0   1  1     0    0  0  1  1     0
1   0  0     1    0  1  0  0     0
1   0  1     0    0  1  0  1     0
1   1  0     0    0  1  1  0     0
1   1  1     1    0  1  1  1     1
0   0  0     0    1  0  0  0     0
0   0  1     0    1  0  0  1     0
0   1  0     1    1  0  1  0     1
0   1  1     0    1  0  1  1     0
1   0  0     1    1  1  0  0     0
1   0  1     0    1  1  0  1     0
1   1  0     0    1  1  1  0     1
1   1  1     1    1  1  1  1     0
\end{Verbatim}

To start a test, click \textsc{Simulate -> Test Vector}. The window illustrated in Figure \ref{fig:02-06} opens. 

\begin{figure}[H]
	\centering
	\includegraphics[width=\maxwidth{.95\linewidth}]{gfx/02-06}
	\caption{Test Vector Window}
	\label{fig:02-06}
\end{figure}

Click the \textit{Load Vector} button at the bottom of the window and load the test vector file. The test will automatically start and Logisim-evolution will report the results, like in Figure \ref{fig:02-07}.

\begin{figure}[H]
	\centering
	\includegraphics[width=\maxwidth{.95\linewidth}]{gfx/02-07}
	\caption{Test Completed}
	\label{fig:02-07}
\end{figure}

The test indicates all 16 lines passed and zero failed so it could be reasonably concluded that the circuits are functioning properly. Figure \ref{fig:02-08} illustrates a failed test. The circuit designer would then need to troubleshoot to determine what went wrong with the circuit.

\begin{figure}[H]
	\centering
	\includegraphics[width=\maxwidth{.95\linewidth}]{gfx/02-08}
	\caption{Test Failure}
	\label{fig:02-08}
\end{figure}

\section{Deliverable}

\marginpar{It is important to name all inputs and outputs as specified in the lab since they are checked with a Test Vector file that depends on those names.}To receive a grade for this lab, complete the \lstinline[columns=fixed]|main| circuit and both subcircuits. Be sure the standard identifying information is at the top left of the \lstinline[columns=fixed]|main| circuit, similar to: 

\bigskip
% The minipage environment keeps the three lines together - no page break.
\begin{minipage}{\linewidth}
	\begin{verbatim}
	George Self
	Lab 02: Boolean Equations
	February 18, 2018
	\end{verbatim}
\end{minipage}
\bigskip

Save the file with this name: \emph{\texttt{Lab02\_Bool}} and submit that file for grading.

 
%\include{Chapters/03_Encoder}

% *********************************************************
% Part 3: Combinational Circuits
% *********************************************************
% use \cleardoublepage here to avoid problems with pdfbookmark
\cleardoublepage  %  >>>>> Include <<<<<
\ctparttext{\textsc{Combinational Logic} is the bedrock for all digital logic circuits. A combinational circuit's output is determined only by the status of the various inputs and an external clock signal is not necessary as in sequential circuits. All of the circuits completed so far in this manual have been combinational and the two labs in this part of the manual are designed to further develop the concepts of combinational digital logic with two relatively complex examples.}  
\part{Combinational Circuits}  
\include{Chapters/04_ALU}  
%%***************************************
% Lab 05: Vending Machine
%***************************************
\chapter{Vending Machine}

\section{Purpose}

One of the important benefits of working with \textit{Logisim-evoluation} is being able to simulate real-world circuits before they are physically built. This lab simulates a vending machine that meets these requirements:

\begin{enumerate}
	\item The customer can input the following coins: 5-cent, 10-cent, 25-cent.
	\item When 75 cents is input, the machine will activate the dispenser and permit the customer to select a product.
	\item When at least 75 cents is input no more coins will be accepted.
	\item Change will be returned to the customer if more than 75 cents is deposited.
	\item A reset button will return the customer's money.
	\item When a product is dispensed, 75 cents will be added to the machine's ``Total Money Collected'' register.
	\item No product is dispensed if less than 75 cents is deposited.
	\item The current number of items available for each product is stored in a counter.
	\item When a service technician restocks the machine the item count for each product is set to 15, which is the maximum number of items that can be stocked.
	\item If the number of products available is zero for any one product the machine will light a ``sold out'' light and no action will be taken if that product is selected.
\end{enumerate}

This circuit uses only combinational logic and is an example of a reasonably complex system. 

\section{Procedure}

The starter circuit for this lab is almost complete, but three of the requirements have not been met.

\begin{itemize}
	\item Requirement three is that the coin input will stop once 75 cents is reached but this is not working so customers can continue depositing coins into the machine.
	\item When a product is dispensed, the coins deposited and change returned is not reset back to zero. This means that a customer could deposit 75 cents and then keep selecting products until the machine is empty.
	\item Requirement six is that the machine totals all of the money collected but that is not functional.
\end{itemize}

\subsection{Testing the Circuit}

To test the circuit:

\begin{enumerate}
	\item Ensure simulation is enabled at \textsc{Simulate -> Simulation Enabled}.
	\item Poke the \textit{Ena} input pin to enable the vending machine simulator.
	\item Notice that the \textit{SoldOut1}, \textit{SoldOut2}, and \textit{SoldOut3} LEDs are lit, indicating that those products are sold out.
	\item Restock products by poking the \textit{Restock1} and \textit{Restock2} buttons. For this test, do not poke \textit{Restock3} to keep that product empty. As a product is restocked the ``SoldOut'' LED for that product goes out.
	\item Poke the \textit{In5}, \textit{In10}, and \textit{In25} buttons to deposit coins. The total deposited is displayed and any amount over 75 cents is shown as change. Notice that the deposit circuit is not disabled after 75 cents is reached so customers can continue depositing coins.
	\item Once at least 75 cents is deposited, poke \textit{Vend1} to vend that product. The number of items available for that product decreases. Notice that once a product is dispensed the amount of money deposited is not reset and the machine can dispense additional products without additional money being deposited.
	\item Poke \textit{Vend3} and notice that nothing happens since that product is sold out.
	\item Poke \textit{Reset} to reset the amount of money deposited.
\end{enumerate}

\subsection{Subcircuit Descriptions}

This simulator contains five subcircuits in addition to the \lstinline[columns=fixed]|main| circuit and this section describes all of those components.

\subsubsection{main}

The \lstinline[columns=fixed]|main| circuit is the interface between a human customer and the simulator, as shown in Figure \ref{fig:05-01}. 

\begin{figure}[H]
	\centering
	\includegraphics[width=\maxwidth{.95\linewidth}]{gfx/05-01}
	\caption{Vending Machine Main Circuit}
	\label{fig:05-01}
\end{figure}

The \lstinline[columns=fixed]|main| circuit includes the following components.

\begin{itemize}
	\item Numeric displays for the amount deposited, the change returned, and the number of items available for each of three products.
	\item An \textit{Ena} (\textit{Enable}) input so a technician can disable the machine for servicing.
	\item Buttons to simulate depositing coins, vending products, and restocking the machine.
	\item LEDs to indicate when products are sold out and dispensed.
\end{itemize}

\subsubsection{Activator}

The \lstinline[columns=fixed]|Activator| subcircuit receives a signal from the \lstinline[columns=fixed]|Bank| subcircuit that indicates how much money has been collected. The \lstinline[columns=fixed]|Activator| returns the \ac{BCD} Total and Change values and sets a signal to activate the \lstinline[columns=fixed]|Dispenser| subcircuit once 75 cents has been deposited. Figure \ref{fig:05-02} illustrates the \lstinline[columns=fixed]|Activator| subcircuit.

\begin{figure}[H]
	\centering
	\includegraphics[width=\maxwidth{.95\linewidth}]{gfx/05-02}
	\caption{Activator Subcircuit}
	\label{fig:05-02}
\end{figure}

The \lstinline[columns=fixed]|Activator| subcircuit has only one input, \textit{InCash}. That input is connected to the \lstinline[columns=fixed]|Bank| subcircuit output and contains the total amount of cash deposited. That input is connected to a Bin2BCD (\textit{BFH mega functions} library) device and is then output as a \ac{BCD} number on the \textit{DepositedBCD} output pin.

The \textit{InCash} input is also sent to a comparator where the amount is compared to 75. If the amount in the bank is equal to or greater than 75 then the \lstinline[columns=fixed]|Activate| output goes high.

Finally, the \textit{InCash} input is sent to a mux that outputs 75 until the comparator indicates that more than 75 is in the bank, then the mux passes the \textit{InCash} amount to a subtractor where 75 is subtracted from it and the result sent to the \textit{ChangeBCD} output.

\subsubsection{Bank}

The \lstinline[columns=fixed]|Bank| subcircuit keeps a running total of the amount deposited and sends that total to the \lstinline[columns=fixed]|Activator| subcircuit. Figure \ref{fig:05-03} illustrates the \lstinline[columns=fixed]|Bank| subcircuit.

\begin{figure}[H]
	\centering
	\includegraphics[width=\maxwidth{.95\linewidth}]{gfx/05-03}
	\caption{Bank Subcircuit}
	\label{fig:05-03}
\end{figure}

The \lstinline[columns=fixed]|Bank| subcircuit has five inputs. \textit{In5}, \textit{In10}, and \textit{In25} indicate the value of the coin dropped into the machine. When high, the \textit{Ena} input enables the \lstinline[columns=fixed]|Bank|. When high, the \textit{Rst} input resets the total to zero.

The \lstinline[columns=fixed]|Bank| subcircuit has only one output, \textit{OutAcc}, that makes the total cash accumulated available to the \lstinline[columns=fixed]|Activator| subcircuit.

For this description, imagine that a 5-cent coin is deposited. \textit{In5} goes high which changes the output of the priority encoder from zero to one. That output is sent to a mux control where the number five, on mux input one, is passed to an adder. The output of the adder is sent to a register where it is remembered. The output of the register is sent to the \textit{OutAcc} pin but is also looped back to the adder so each new coin is added to the previous total. Thus, the register keeps a running total of the money deposited.

The final logic function in this subcircuit is a three-input \texttt{OR} gate where each of the coin input pins are sent to the clock input of the register. As coins are dropped into the machine the register is clocked in order to capture each new deposit. It is important to note that \textit{the register is set to activate on a falling edge} in order to give the input signal enough time to propagate through the priority encoder, mux, and adder.

\subsubsection{Dispenser}

The \lstinline[columns=fixed]|Dispenser| subcircuit dispenses the three products available in the machine. Figure \ref{fig:05-04} illustrates the \lstinline[columns=fixed]|Dispenser| subcircuit.

\begin{figure}[H]
	\centering
	\includegraphics[width=\maxwidth{.95\linewidth}]{gfx/05-04}
	\caption{Dispenser Subcircuit}
	\label{fig:05-04}
\end{figure}

The Dispenser subcircuit has seven inputs and nine outputs.

Inputs:

\begin{itemize}
	\item \textbf{Activate}. A high input on this pin permits a product to be dispensed. This signal is generated in the \lstinline[columns=fixed]|Activator| subcircuit.
	\item \textbf{Vend}. These inputs cause one of three products to be dispensed.
	\item \textbf{Restock}. This resets the product count to 15, simulating a service technician restocking the machine.
\end{itemize}

Outputs:

\begin{itemize}
	\item \textbf{Avail}. This is an 8-bit number (not \ac{BCD}) that shows how many items each of the products have available for sale.
	\item \textbf{Empty}. This LED goes high when any product is sold out.
	\item \textbf{Disp}. This LED goes high when an item is dispensed.
\end{itemize}

Overall, this is a rather simple subcircuit. When one of the \textit{Vend} inputs goes high the priority encoder sends the number for that input to the demux control port. Thus, if a customer selects product one then the priority encoder transmits a one to the demux.

The demux will transmit the value present on the \textit{Activate} input to one of three \lstinline[columns=fixed]|Product| subcircuits. When \textit{Activate} is low then a zero is transmitted to the \lstinline[columns=fixed]|Product| subcircuit which effectively disables the dispenser function. However, if \textit{Activate} is high then a one is transmitted to one of the \lstinline[columns=fixed]|Product| subcircuits and that will cause a product to be dispensed.

\subsubsection{Product}

The \lstinline[columns=fixed]|Product| subcircuit keeps count of the number of items available for a product. There are two inputs and three outputs.

Inputs:

\begin{itemize}
	\item \textbf{Restock}. This resets the count of the item to 15. It is designed to simulate a service technician restocking the machine.
	\item \textbf{Vend}. When this goes high a single item is dispensed.
\end{itemize}

Outputs:

\begin{itemize}
	\item \textbf{AvailBCD}. This is a count, in \ac{BCD}, of the number of items available for sale.
	\item \textbf{Empty}. This goes high when there are no items available for sale.
	\item \textbf{Dispensed}. This goes high when an item is dispensed. It represents an item physically dropping out of the machine for the customer to retrieve.
\end{itemize}

\begin{figure}[H]
	\centering
	\includegraphics[width=\maxwidth{.95\linewidth}]{gfx/05-05}
	\caption{Product Subcircuit}
	\label{fig:05-05}
\end{figure}

This subcircuit is nothing more than a counter with a few controlling signals. The counter has a constant zero input on the \textit{M3} port. That sets the counter to decrement the count on each clock pulse.

The \textit{Restock} input is wired to the counter's reset port and a high input will reset the counter to 15. Note, the counter's properties are pre-set for a maximum count of 15.

The \textit{Vend} input is wired to the counter's clock port so when an item is sold the count will decrease. This input is also wired to the \textit{Dispensed} output to indicate that an item was sold.

The counter has two outputs. The \textit{3CT=0xF} output goes high when the count reaches zero (the item is sold out). That signal is used to disable the counter so no further sales are made. The second counter output is the count it contains and that is wired to a Bin2BCD (\textit{BFH mega functions} library) device. The output of that device is sent to the \textit{AvailBCD} port for other subcircuits to use.

\subsubsection{Vending}

The \lstinline[columns=fixed]|Vending| subcircuit consolidates the other subcircuits into an \ac{IC} that is used in the \lstinline[columns=fixed]|main| circuit. Figure \ref{fig:05-06} illustrates the \lstinline[columns=fixed]|Vending| subcircuit.

\begin{figure}[H]
	\centering
	\includegraphics[width=\maxwidth{.95\linewidth}]{gfx/05-06}
	\caption{Vending Subcircuit}
	\label{fig:05-06}
\end{figure}

No further explanation is given for this subcircuit since it only wires the other subcircuits together and introduces no new logic.

\section{Challenge}

The Vending Machine simulator has three vital flaws that must be corrected.

\begin{itemize}
	\item Requirement three is that the coin input will stop once 75 cents is reached but this is not working so customers can continue depositing coins into the machine.
	\item When a product is dispensed, the coins deposited and change returned is not reset back to zero. This means that a customer could deposit 75 cents and then keep selecting products until the machine is empty.
	\item Requirement six is that the machine totals all of the money collected but that is not functional.
\end{itemize}


\section{Deliverable}

To receive a grade for this lab, correct all three flaws identified in the Challenge. Be sure the standard identifying information is at the top left of the \textit{main} circuit, similar to: 

\bigskip
% The minipage environment keeps the three lines together - no page break.
\begin{minipage}{\linewidth}
	\begin{verbatim}
	George Self
	Lab 05: Vending Machine
	February 16, 2018
	\end{verbatim}
\end{minipage}
\bigskip

Save the file with this name: \emph{\texttt{Lab05\_Vend}} and submit that file for grading.

 

% *********************************************************
% Part 4: Sequential Circuits
% *********************************************************
% use \cleardoublepage here to avoid problems with pdfbookmark
\cleardoublepage  
\ctparttext{\textsc{Sequential Logic} circuits develop the concepts of clock-driven logic while creating several practical counters and memory circuits. These labs also introduce the \textit{Logisim-evolution} \textit{Chronogram}, which builds timing diagrams for sequential logic circuits.}  
\part{Sequential Circuits}  
%%**************************************************************
% Lab 06: Counter
%**************************************************************
\chapter{Counters}

Counters are perhaps the most commonly-used circuits in electronic devices. They are found in virtually all electronics systems, from the simplest embedded computers to massive mainframes. Counters are designed to cycle through a specific predefined sequence of binary numbers when an input pulse is applied. Typically, counters simply count up or down from given start and end numbers, but they can be designed to produce unique output patterns for special uses. 

Counters, though, are used for more than simple counting. They can measure time so devices like alarm clocks and watches include counters. They are used as frequency dividers so a fast input frequency can be output at a slower rate. In devices with memory they are used to increment memory addresses as a program steps through some process. They can activate a series of subcircuits in sequence as part of a complex process. They are, in short, one of the most important workhorses of the digital logic world.

\section{Purpose}

This lab has two goals: 

\begin{enumerate}
	\item Develop several different common counters using \textit{D} flip-flops. Because there are two main families of counters, asynchronous and synchronous, this lab includes examples of both. 
	\item Introduce the \LE chronogram feature that generates a timing diagram as a sequential circuit functions. 
\end{enumerate}

\section{Procedure}

\subsection{Asynchronous Up Counter}

A counter is built from a series of flip-flops and where the output from each flip-flop is combined to create the counter output, trigger the next flip-flop, or both. Each flip-flop is considered a ``stage'' of the counter. A counter is triggered by a clock signal that is typically supplied by a timer with a regularly-recurring pattern of high/low levels, but it can also be triggered by an event of some sort, like the press of a button or the completion of a process.

\marginpar{In all Counter circuits in this manual flip-flop U0 provides the Least Significant Bit to the output and U3 provides the Most Significant Bit.}One of the simplest counters is illustrated in Figure \ref{fig:06-01}. This is an asynchronous four-stage up counter. A counter is is considered ``asynchronous'' if the input clock signal is applied to only the first stage and then that signal ripples through each flip-flop in turn. Thus, an asynchronous counter is frequently called a ``ripple'' counter.

\begin{figure}[H]
	\centering
	\includegraphics[width=\maxwidth{.95\linewidth}]{gfx/06-01}
	\caption{Asynchronous Up Counter}
	\label{fig:06-01}
\end{figure}

The following list describes the operation of the counter in Figure \ref{fig:06-01}. Students should open the counter circuit with \LE then use the ``poke'' tool to set the clock high then low (one complete clock cycle) as they follow the description below.

\begin{description}
	\item [Reset is Activated] All flip-flops are reset so \textit{Q} is low and \textit{Q'} is high.

	%%%%%%%%%%%%%%%
	\item [Tick 1] \textit{U0} clocked: \textit{Q0} \textuparrow \: \textemdash \: \textit{Q'0} \textdownarrow

	%%%%%%%%%%%%%%%
	\item [Tick 2] \textit{U0} clocked: \textit{Q0} \textdownarrow \: \textemdash \: \textit{Q'0} \textuparrow
	
	\hspace{14pt}\textit{U1} clocked: \textit{Q1} \textuparrow \: \textemdash \: \textit{Q'1} \textdownarrow
	
	%%%%%%%%%%%%%%%
	\item [Tick 3] \textit{U0} clocked: \textit{Q0} \textuparrow \: \textemdash \: \textit{Q'0} \textdownarrow

	%%%%%%%%%%%%%%%
	\item [Tick 4] \textit{U0} clocked: \textit{Q0} \textdownarrow \: \textemdash \: \textit{Q'0} \textuparrow

	\hspace{14pt}\textit{U1} clocked: \textit{Q1} \textdownarrow \: \textemdash \: \textit{Q'1} \textuparrow

	\hspace{14pt}\textit{U2} clocked: \textit{Q2} \textuparrow \: \textemdash \: \textit{Q'2} \textdownarrow

	%%%%%%%%%%%%%%%
	\item [Tick 5] \textit{U0} clocked: \textit{Q0} \textuparrow \: \textemdash \: \textit{Q'0} \textdownarrow

	%%%%%%%%%%%%%%%
	\item [Tick 6] \textit{U0} clocked: \textit{Q0} \textdownarrow \: \textemdash \: \textit{Q'0} \textuparrow

	\hspace{14pt}\textit{U1} clocked: \textit{Q1} \textuparrow \: \textemdash \: \textit{Q'1} \textdownarrow
	
	%%%%%%%%%%%%%%%
	\item [Tick 7] \textit{U0} clocked: \textit{Q0} \textuparrow \: \textemdash \: \textit{Q'0} \textdownarrow

	%%%%%%%%%%%%%%%
	\item [Tick 8] \textit{U0} clocked: \textit{Q0} \textdownarrow \: \textemdash \: \textit{Q'0} \textuparrow

	\hspace{14pt}\textit{U1} clocked: \textit{Q1} \textdownarrow \: \textemdash \: \textit{Q'1} \textuparrow
	
	\hspace{14pt}\textit{U2} clocked: \textit{Q2} \textdownarrow \: \textemdash \: \textit{Q'2} \textuparrow

	\hspace{14pt}\textit{U3} clocked: \textit{Q3} \textuparrow \: \textemdash \: \textit{Q'3} \textdownarrow

\end{description}

As the clock continues the counter would cycle through the binary values 1001 - 1111. The following table lists the \textit{Up} counter output as indicated by the \textit{Q} values at each tick listed above.

\begin{table}[H]
	\sffamily
	\newcommand{\head}[1]{\textcolor{white}{\textbf{#1}}}		
	\begin{center}
		\rowcolors{2}{gray!10}{white} % Color every other line a light gray
		\begin{tabular}{cc} 
			\rowcolor{black!75}
			\head{Tick} & \head{Output} \\
			Reset & 0000 \\
			1 & 0001 \\
			2 & 0010 \\
			3 & 0011 \\
			4 & 0100 \\
			5 & 0101 \\
			6 & 0110 \\
			7 & 0111 \\
			8 & 1000
		\end{tabular}
	\end{center}
	\caption{Up Counter Output}
	\label{tab0601}
\end{table}

\subsection{Asynchronous Down Counter}

The asynchronous down counter illustrated in Figure \ref{fig:06-02} is very similar to the up counter in Figure \ref{fig:06-01} except the stages are triggered from the \textit{Q} output of the preceding stage rather than \textit{Q'} and the \textit{Reset} signal is applied to the flip-flop \textit{S} input rather than \textit{R}.

\begin{figure}[H]
	\centering
	\includegraphics[width=\maxwidth{.95\linewidth}]{gfx/06-02}
	\caption{Asynchronous Down Counter}
	\label{fig:06-02}
\end{figure}

The following list describes the operation of the counter in Figure \ref{fig:06-02}. Students should open the counter circuit with \LE then use the ``poke'' tool to set the clock high then low (one complete clock cycle) as they follow the description below.

\begin{description}
	\item [Reset is activated] All flip-flops are set so \textit{Q} is high and \textit{Q'} is low.
	
	\item [Tick 1] \textit{U0} clocked: \textit{Q0} \textdownarrow \: \textemdash \: \textit{Q'0} \textuparrow

	%%%%%%%%%%%%%%%
	\item [Tick 2] \textit{U0} clocked: \textit{Q0} \textuparrow \: \textemdash \: \textit{Q'0} \textdownarrow
	
	\hspace{14pt}\textit{U1} clocked: \textit{Q1} \textdownarrow \: \textemdash \: \textit{Q'1} \textuparrow
	
	%%%%%%%%%%%%%%%
	\item [Tick 3] \textit{U0} clocked: \textit{Q0} \textdownarrow \: \textemdash \: \textit{Q'0} \textuparrow

	%%%%%%%%%%%%%%%
	\item [Tick 4] \textit{U0} clocked: \textit{Q0} \textuparrow \: \textemdash \: \textit{Q'0} \textdownarrow
	
	\hspace{14pt}\textit{U1} clocked: \textit{Q1} \textuparrow \: \textemdash \: \textit{Q'1} \textdownarrow
	
	\hspace{14pt}\textit{U2} clocked: \textit{Q2} \textdownarrow \: \textemdash \: \textit{Q'2} \textuparrow

	%%%%%%%%%%%%%%%
	\item [Tick 5] \textit{U0} clocked: \textit{Q0} \textdownarrow \: \textemdash \: \textit{Q'0} \textuparrow

	%%%%%%%%%%%%%%%	
	\item [Tick 6] \textit{U0} clocked: \textit{Q0} \textuparrow \: \textemdash \: \textit{Q'0} \textdownarrow
	
	\hspace{14pt}\textit{U1} clocked: \textit{Q1} \textdownarrow \: \textemdash \: \textit{Q'1} \textuparrow

	%%%%%%%%%%%%%%%
	\item [Tick 7] \textit{U0} clocked: \textit{Q0} \textdownarrow \: \textemdash \: \textit{Q'0} \textuparrow

	%%%%%%%%%%%%%%%
	\item [Tick 8] \textit{U0} clocked: \textit{Q0} \textuparrow \: \textemdash \: \textit{Q'0} \textdownarrow
	
	\hspace{14pt}\textit{U1} clocked: \textit{Q1} \textuparrow \: \textemdash \: \textit{Q'1} \textdownarrow
	
	\hspace{14pt}\textit{U2} clocked: \textit{Q2} \textuparrow \: \textemdash \: \textit{Q'2} \textdownarrow
	
	\hspace{14pt}\textit{U3} clocked: \textit{Q3} \textdownarrow \: \textemdash \: \textit{Q'3} \textuparrow
	
\end{description}

As the clock continues the counter would cycle through the binary values 0110 - 0000. The following table lists the \textit{Down} counter output as indicated by the \textit{Q} values at each tick listed above.

\begin{table}[H]
	\sffamily
	\newcommand{\head}[1]{\textcolor{white}{\textbf{#1}}}		
	\begin{center}
		\rowcolors{2}{gray!10}{white} % Color every other line a light gray
		\begin{tabular}{cc} 
			\rowcolor{black!75}
			\head{Tick} & \head{Output} \\
			Reset & 1111 \\
			1 & 1110 \\
			2 & 1101 \\
			3 & 1100 \\
			4 & 1011 \\
			5 & 1010 \\
			6 & 1001 \\
			7 & 1000 \\
			8 & 0111
		\end{tabular}
	\end{center}
	\caption{Down Counter Output}
	\label{tab0602}
\end{table}


\subsection{Asynchronous Decade Counter}

Binary counters, like those considered in Figure \ref{fig:06-01} and Figure \ref{fig:06-02} are only able to count to a value that is a power of two but it is often necessary to build a counter that stops at some other value. These types of counters are called ``mod'' counters (short for ``modulus'') since they count up to a preset value then reset and start over, like modulus math. One of the most common mod counters is one that has ten states (it counts from zero to nine) and then resets, and that type of counter is generally referred to as a decade counter. Decade counters are found in any application that has to count in decimal for easy human interpretation.

The logic of a  mod counter is to add an \texttt{AND} gate on the flip-flop outputs such that the output of the \texttt{AND} gate is high when the flip-flop outputs equal the mod number. For example, the \texttt{AND} gate for a decade counter would go high when the count reaches ten and that signal would immediately reset the counter back to zero.

\begin{figure}[H]
	\centering
	\includegraphics[width=\maxwidth{.95\linewidth}]{gfx/06-03}
	\caption{Asynchronous Decade Counter}
	\label{fig:06-03}
\end{figure}

The following list describes the operation of the counter in Figure \ref{fig:06-03}:

\begin{description}
	\item [Reset is activated] All flip-flops are reset so \textit{Q} is low and \textit{Q'} is high.
	
	%%%%%%%%%%%%%%%
	\item [Tick 1] \textit{U0} clocked: \textit{Q0} \textuparrow \: \textemdash \: \textit{Q'0} \textdownarrow
	
	%%%%%%%%%%%%%%%
	\item [Tick 2] \textit{U0} clocked: \textit{Q0} \textdownarrow \: \textemdash \: \textit{Q'0} \textuparrow
	
	\hspace{14pt}\textit{U1} clocked: \textit{Q1} \textuparrow \: \textemdash \: \textit{Q'1} \textdownarrow
	
	%%%%%%%%%%%%%%%
	\item [Tick 3] \textit{U0} clocked: \textit{Q0} \textuparrow \: \textemdash \: \textit{Q'0} \textdownarrow
	
	%%%%%%%%%%%%%%%
	\item [Tick 4] \textit{U0} clocked: \textit{Q0} \textdownarrow \: \textemdash \: \textit{Q'0} \textuparrow
	
	\hspace{14pt}\textit{U1} clocked: \textit{Q1} \textdownarrow \: \textemdash \: \textit{Q'1} \textuparrow
	
	\hspace{14pt}\textit{U2} clocked: \textit{Q2} \textuparrow \: \textemdash \: \textit{Q'2} \textdownarrow
	
	%%%%%%%%%%%%%%%
	\item [Tick 5] \textit{U0} clocked: \textit{Q0} \textuparrow \: \textemdash \: \textit{Q'0} \textdownarrow
	
	%%%%%%%%%%%%%%%
	\item [Tick 6] \textit{U0} clocked: \textit{Q0} \textdownarrow \: \textemdash \: \textit{Q'0} \textuparrow
	
	\hspace{14pt}\textit{U1} clocked: \textit{Q1} \textuparrow \: \textemdash \: \textit{Q'1} \textdownarrow
	
	%%%%%%%%%%%%%%%
	\item [Tick 7] \textit{U0} clocked: \textit{Q0} \textuparrow \: \textemdash \: \textit{Q'0} \textdownarrow
	
	%%%%%%%%%%%%%%%
	\item [Tick 8] \textit{U0} clocked: \textit{Q0} \textdownarrow \: \textemdash \: \textit{Q'0} \textuparrow
	
	\hspace{14pt}\textit{U1} clocked: \textit{Q1} \textdownarrow \: \textemdash \: \textit{Q'1} \textuparrow
	
	\hspace{14pt}\textit{U2} clocked: \textit{Q2} \textdownarrow \: \textemdash \: \textit{Q'2} \textuparrow
	
	\hspace{14pt}\textit{U3} clocked: \textit{Q3} \textuparrow \: \textemdash \: \textit{Q'3} \textdownarrow

	%%%%%%%%%%%%%%%
	\item [Tick 9] \textit{U0} clocked: \textit{Q0} \textuparrow \: \textemdash \: \textit{Q'0} \textdownarrow

	%%%%%%%%%%%%%%%
	\item [Tick 10] \textit{U1} clocked: \textit{Q0} \textuparrow \: \textemdash \: \textit{Q'0} \textdownarrow
	
	\noindent Both inputs for the \texttt{AND} gate are momentarily high and that sends a reset signal that causes all outputs to go low.
	
\end{description}

As the clock continues the counter would cycle through the binary values 0000 - 1001. The following table lists the \textit{Decade} counter output as indicated by the \textit{Q} values at each tick listed above.

\begin{table}[H]
	\sffamily
	\newcommand{\head}[1]{\textcolor{white}{\textbf{#1}}}		
	\begin{center}
		\rowcolors{2}{gray!10}{white} % Color every other line a light gray
		\begin{tabular}{cc} 
			\rowcolor{black!75}
			\head{Tick} & \head{Output} \\
			Reset & 0000 \\
			1 & 0001 \\
			2 & 0010 \\
			3 & 0011 \\
			4 & 0100 \\
			5 & 0101 \\
			6 & 0110 \\
			7 & 0111 \\
			8 & 1000 \\
			9 & 1001 \\
		 10 & 0000
		\end{tabular}
	\end{center}
	\caption{Decade Counter Output}
	\label{tab0603}
\end{table}

\subsection{Synchronous Ring Counter}

In a ring counter the high bit is shifted through all of the bits one at a time. This counter is very useful in controlling subcircuits since the high bit in the counter can activate the next subcircuit in the sequence.

The ring counter presented here is also a synchronous circuit; that is, each clock pulse is applied to all of the flip-flops instead of just the first stage. The \textit{Q} output from each flip-flop is used but \textit{Q'} is not needed at all. Also, there is a feedback line from \textit{U3} to the data input port of \textit{U0} so when the \textit{Q} output of \textit{U3} goes high that is made available to \textit{U0} and loop that value back through the circuit.

\begin{figure}[H]
	\centering
	\includegraphics[width=\maxwidth{.95\linewidth}]{gfx/06-04}
	\caption{Synchronous Ring Counter}
	\label{fig:06-04}
\end{figure}

The following list describes the operation of the counter in Figure \ref{fig:06-04}. Students should open the counter circuit with \LE then use the ``poke'' tool to set the clock high then low (one complete clock cycle) as they follow the description below.

\begin{description}
	\item [Reset is activated] \textit{U0} is set and \textit{U1}-\textit{U3} are reset so the counter is seeded with a single high bit to shift.
	
	%%%%%%%%%%%%%%%
	\item [Tick 1] \textit{Q0} \textdownarrow \: \textemdash \: \textit{Q1} \textuparrow
	
	%%%%%%%%%%%%%%%
	\item [Tick 2] \textit{Q1} \textdownarrow \: \textemdash \: \textit{Q2} \textuparrow
	
	%%%%%%%%%%%%%%%
	\item [Tick 3] \textit{Q2} \textdownarrow \: \textemdash \: \textit{Q3} \textuparrow
	
	%%%%%%%%%%%%%%%
	\item [Tick 4] \textit{Q3} \textdownarrow \: \textemdash \: \textit{Q1} \textuparrow
	
\end{description}

As the clock continues the counter would cycle through the binary values 0001 - 1000. The following table lists the \textit{ring} counter output as indicated by the \textit{Q} values at each tick listed above.

\begin{table}[H]
	\sffamily
	\newcommand{\head}[1]{\textcolor{white}{\textbf{#1}}}		
	\begin{center}
		\rowcolors{2}{gray!10}{white} % Color every other line a light gray
		\begin{tabular}{cc} 
			\rowcolor{black!75}
			\head{Tick} & \head{Output} \\
			Reset & 0001 \\
			1 & 0010 \\
			2 & 0100 \\
			3 & 1000 \\
			4 & 0001 \\
			5 & 0010 \\
			6 & 0100 \\
			7 & 1000 \\
			8 & 0001 
		\end{tabular}
	\end{center}
	\caption{Ring Counter Output}
	\label{tab0604}
\end{table}

\subsection{Synchronous Johnson Counter}

A Johnson Counter is similar to a ring counter in that a high bit value is shifted through the entire binary word. The difference is that the feedback loop comes from the \textit{Q'} output of the last stage rather than the \textit{Q} output. This type of counter is sometimes called a ``twisted tail'' counter since the \textit{Q'} output is fedback.

\begin{figure}[H]
	\centering
	\includegraphics[width=\maxwidth{.95\linewidth}]{gfx/06-05}
	\caption{Synchronous Johnson Counter}
	\label{fig:06-05}
\end{figure}

The following list describes the operation of the counter in Figure \ref{fig:06-05}. Students should open the counter circuit with \LE then use the ``poke'' tool to set the clock high then low (one complete clock cycle) as they follow the description below.

\begin{description}
	\item [Reset is activated] \textit{U0} is set and \textit{U1}-\textit{U3} are reset so the counter is seeded with a single high bit to shift.
	
	%%%%%%%%%%%%%%%
	\item [Tick 1] \textit{Q1} \textuparrow 
	
	%%%%%%%%%%%%%%%
	\item [Tick 2] \textit{Q2} \textuparrow
	
	%%%%%%%%%%%%%%%
	\item [Tick 3] \textit{Q3} \textuparrow 
	
	%%%%%%%%%%%%%%%
	\item [Tick 4] \textit{Q0} \textdownarrow

	%%%%%%%%%%%%%%%
	\item [Tick 5] \textit{Q1} \textdownarrow

	%%%%%%%%%%%%%%%
	\item [Tick 6] \textit{Q2} \textdownarrow
	
	%%%%%%%%%%%%%%%
	\item [Tick 7] \textit{Q3} \textdownarrow

	%%%%%%%%%%%%%%%
	\item [Tick 8] \textit{Q0} \textuparrow

\end{description}

As the clock continues the counter would cycle through the binary values 0000 - 1111. The following table lists the \textit{Johnson} counter output as indicated by the \textit{Q} values at each tick listed above.

\begin{table}[H]
	\sffamily
	\newcommand{\head}[1]{\textcolor{white}{\textbf{#1}}}		
	\begin{center}
		\rowcolors{2}{gray!10}{white} % Color every other line a light gray
		\begin{tabular}{cc} 
			\rowcolor{black!75}
			\head{Tick} & \head{Output} \\
			Reset & 0001 \\
			1 & 0011 \\
			2 & 0111 \\
			3 & 1111 \\
			4 & 1110 \\
			5 & 1100 \\
			6 & 1000 \\
			7 & 0000 \\
			8 & 0001 
		\end{tabular}
	\end{center}
	\caption{Johnson Counter Output}
	\label{tab0605}
\end{table}

\subsection{Main}

The \lstinline[columns=fixed]|main| circuit provides a human interface to try out each of the counters by dropping them in place of the \textit{Up} counter.

\begin{figure}[H]
	\centering
	\includegraphics[width=\maxwidth{.95\linewidth}]{gfx/06-06}
	\caption{Main Circuit}
	\label{fig:06-06}
\end{figure}

Notice that there are two clocks in the \lstinline[columns=fixed]|main| circuit. \textit{Clk} is linked to the counter being tested and is used within the counter circuit to advance the count. \textit{Sysclk} is used by the \LE chronogram as described in the next section of this document.

\subsection{Chronogram}

\LE can generate a timing diagram, called a \textit{chronogram}, for a sequential circuit. That is a representation of the various signals in a circuit and how those signals change over time. Figure \ref{fig:06-07} is the timing diagram for an Up counter.

\begin{figure}[H]
	\centering
	\includegraphics[width=\maxwidth{.95\linewidth}]{gfx/06-07}
	\caption{Timing Diagram for Up Counter}
	\label{fig:06-07}
\end{figure}

At the top of Figure \ref{fig:06-07} is a scale that indicates the number of seconds that the counter has been operating. The first trace is the input clk signal. The clock goes high at the start of each second and then goes low at the half-second mark. Under the clock is the ``Probe1'' signal. Because that is a four-bit number \LE displays the number, but under that number is a breakout of the four bits that make up that number. Thus, at time zero ``Probe1'' is 0001 and ``Probe1\_s\_0'' (that stands for ``Probe 1, Signal 0'') is high while the other bits are low. The \LE \textit{chronogram} includes a cursor indicated by a red line (found just before the five second tick in Figure \ref{fig:06-07}) that can be placed anywhere along the diagram. The cursor sets the values of each signal in the area on the left edge of the diagram, so the cursor in Figure \ref{fig:06-07} is pointing to a spot where the \textit{clk} is low, \textit{Probe1} is at 0101, and so forth.

Follow the next steps to use the chronogram. Notes: the chronogram will only check subcircuits that are found on the  \lstinline[columns=fixed]|main| subcircuit. Therefore, in order to create a timing diagram all subcircuits need to be combined on \lstinline[columns=fixed]|main|. The labs completed in this manual have been designed to use the \lstinline[columns=fixed]|main| subcircuit as the human interface so the chronogram feature will work well with these circuits. 

\begin{enumerate}
	\item In the \lstinline[columns=fixed]|main| subcircuit, add a ``sampling clock'' labeled \textit{sysclk} (this name is important, do not change it to something else). The sampling clock is only used by the \textit{chronogram} and will not show up in the timing diagram. It should not be connected to any other components and can be placed anywhere on \lstinline[columns=fixed]|main|. Set the properties for \textit{sysclk} to a 1 Tick high duration and a 1 Tick low duration (this is the default). 
	\item Add a circuit master clock labeled \textit{clk}. This is the clock that will be used to trigger all components in the circuit. Set the properties for \textit{clk} to a 4 Tick high duration and a 4 Tick low duration.
	\item Set \textsc{Simulate -> Tick Frequency} to 4 Hertz. This will simulate a clock that ticks once per second, as in Figure \ref{fig:06-07}. While the actual tick frequency can be changed later to ``speed up'' the circuit, a one-second tick is useful for learning how the \textit{chronogram} works.
	\item Click \textsc{Simulate -> Chronogram} to set up the \textit{chronogram}. Figure \ref{fig:06-08} illustrates the initial setup screen for the \textit{chronogram}.

	\begin{figure}[H]
		\centering
		\includegraphics[width=\maxwidth{.95\linewidth}]{gfx/06-08}
		\caption{Set Up Chronogram}
		\label{fig:06-08}
	\end{figure}

	\item Click \textit{sysclk} in the left panel and then click \textit{Add >>} to add that signal to the \textit{chronogram}. The ``-2'' following the \textit{sysclk} name in the right panel indicates that it is a binary signal.\marginpar{NOTE: \textit{sysclk} must be added to the \textit{chronogram} or it will not sample the circuit; however, the \textit{sysclk} signal will not actually show up in the timing diagram.} It is probably best to add the \textit{sysclk} signal first so it is not overlooked.
	\item Click \textit{clk} in the left panel and then click \textit{Add >>} to add that signal to the \textit{chronogram}.
	\item Click \textit{Probe1} in the left panel and then click \textit{Add >>} to add that signal to the \textit{chronogram}.
	\item Click ``Enable time selection'' and chose \textit{clk} as the clock with a frequency of 1 Hertz.
	\item The \textit{chronogram} setup should look like Figure \ref{fig:06-09}.
	
	\begin{figure}[H]
		\centering
		\includegraphics[width=\maxwidth{.95\linewidth}]{gfx/06-09}
		\caption{Chronogram Ready}
		\label{fig:06-09}
	\end{figure}

	\item Click \textit{Start Chronogram} and the screen illustrated in Figure \ref{fig:06-10} pops up.
	
	\begin{figure}[H]
		\centering
		\includegraphics[width=\maxwidth{.95\linewidth}]{gfx/06-10}
		\caption{Chronogram Starting}
		\label{fig:06-10}
	\end{figure}
	
	\item Right-click on the \textit{Probe1} signal and set the format for binary. The format can be set for any radix but to match this lab binary numbers should be specified.
	\item Right-click on the \textit{Probe1} signal and enable \textit{Expand} to see all four signals that create \textit{Probe1}.
	\item At this point, the chronogram should look like Figure \ref{fig:06-11}.
	
	\begin{figure}[H]
		\centering
		\includegraphics[width=\maxwidth{.95\linewidth}]{gfx/06-11}
		\caption{Chronogram At Zero Time}
		\label{fig:06-11}
	\end{figure}
	
	\item The \textit{chronogram} has five buttons that control the simulator.
	
	\begin{figure}[H]
		\centering
		\includegraphics[width=\maxwidth{.95\linewidth}]{gfx/06-12}
		\caption{Chronogram Controls}
		\label{fig:06-12}
	\end{figure}
	
	\begin{itemize}
		\item Button One: Start/Stop the simulation.
		\item Button Two: Simulate one step.
		\item Button Three: Start/Stop \textit{sysclk}. This will ``turn on'' the chronogram and begin creating a timing diagram.
		\item Button Four: Step one \textit{sysclk} tick. This will tick the \textit{sysclk} one time. Since this lab set up the \textit{sysclk} for four ticks per second this button would need to be clicked four times to extend the timing diagram one second.
		\item Button Five: Step one \textit{clk} tick. This extends the timing diagram by one complete clock tick, or one second in this circuit.
	\end{itemize}
	
	\item Click button three to start the \textit{chronogram} and watch the timing diagram unfold. After a few seconds click that button a second time to stop the \textit{chronogram}.
	
	\item The following can be done once the timing diagram is complete.
	
	\begin{itemize}
		\item Click on the timing diagram to set the cursor (indicated by a red line). Once the cursor is set the values for each signal at the cursor's location are printed next to the signal's label on the left edge of the timing diagram.
		\item Hover the mouse over the timing diagram and roll the mouse wheel to zoom the timing diagram appearance.
		\item Click ``Export'' to save the timing diagram signal levels in a text file. That file can later be loaded to reevaluate the timing diagram.
		\item Click ``Export as image'' to save the timing diagram as a PNG file.
	\end{itemize}
	
\end{enumerate}

\section{Challenge}

This lab includes several different timers. Place all of them on a single subcircuit named \lstinline[columns=fixed]|Universal| that includes an output mux so a user can select the type of counter output desired. Place the \lstinline[columns=fixed]|Universal| circuit on \lstinline[columns=fixed]|main| and wire appropriate inputs and outputs.

Set up the chronogram for the ring counter and create a ten-second timing diagram for that counter. Save the timing diagram as a PNG image named ``RingCounter.''

\section{Deliverable}

To receive a grade for this lab, complete the Challenge. Be sure the standard identifying information is at the top left of the \lstinline{main} circuit: 

\bigskip
% The minipage environment keeps the three lines together - no page break.
\begin{minipage}{\linewidth}
	\begin{verbatim}
	George Self
	Lab 06: Counters
	March 17, 2018
	\end{verbatim}
\end{minipage}
\bigskip

Save the circuit with this name: \textit{Lab07\_counter} and submit that along with \textit{RingCounter.PNG} for grading.

 
%\include{Chapters/07_Timer} 
%%*****************************************
% Lab 08: Reaction Timer
%*****************************************
\chapter{Reaction Timer}

\section{Purpose}

This lab continues the exploration of timing circuits and is intended to provide additional practice with sequential circuit design. The project is to build a circuit that times a user's reaction speed. When complete, the \lstinline[columns=fixed]|main| circuit should look something like Figure \ref{fig:08-01}.

\begin{figure}[H]
	\centering
	\includegraphics[width=\maxwidth{.95\linewidth}]{gfx/08-01}
	\caption{Reaction Timer}
	\label{fig:08-01}
\end{figure}

In operation:

\begin{enumerate}
	\item The user clicks \textit{start}. 
	\item An unseen timer begins and counts down a random length of time while the ``Waiting'' LED is lit. 
	\item When the unseen timer reaches zero the ``Waiting'' LED turns off and the numbers on the two hex displays begin to increase.
	\item The user clicks the \textit{Stop} button to stop the timer.
	\item The reaction time is displayed on the two hex displays.
\end{enumerate}

\section{Procedure}

The design of this circuit is left to the student, but the timer built in Lab 7 would be a good starter for this lab. As a tip, \textit{Logisim-evolution} includes a Random Generator (\textit{Memory} library) that can be used to create a random countdown for the ``Waiting'' subcircuit. Finally, the \textsc{Simulate -> Tick Frequency} can be set to a low number (maybe 4 Hz) to build and troubleshoot the circuit for convenience but it should then be set somewhat faster to actually measure a user's reaction time.

\section{Deliverable}

To receive a grade for this lab, complete the circuit. Be sure the standard identifying information is at the top left of the \lstinline{main} circuit, similar to: 

\bigskip
% The minipage environment keeps the three lines together - no page break.
\begin{minipage}{\linewidth}
	\begin{verbatim}
	George Self
	Lab 08: React
	March 11, 2018
	\end{verbatim}
\end{minipage}
\bigskip

Save the file with this name: \emph{\texttt{Lab08\_React}} and submit that file for grading.

 
%%**************************************************************
% Lab 09: ROM
%**************************************************************
\chapter{ROM}\label{Lab09}

\section{Purpose}

This lab introduces students to \acf{ROM} and builds a fun application: The Magic 8-Ball. This was a toy that was developed in the 1950s and was popular throughout the 1960s. It was a small sphere with the markings of an 8-ball. If the user ``asked it a question'' and then turned the ball upside down the answer would magically appear in a small window on the bottom of the ball.

\section{Procedure}

Start a new \textit{Logisim-evolution} project and create a subcircuit named \lstinline[columns=fixed]|Magic_8_Ball|. Open that circuit and place a ROM (\textit{Memory} library) device near the center of the drawing canvas. Set the ROM properties for an \textit{Address Bit Width} of 12 and a \textit{Data Bit Width} of 8.

\begin{figure}[H]
	\centering
	\includegraphics[width=\maxwidth{.95\linewidth}]{gfx/09-01}
	\caption{Placing ROM}
	\label{fig:09-01}
\end{figure}

A ROM stores data that is accessed by setting an address on the inputs at the top left of the device and then reading the contents of that address on the 8-bit bus on the right side of the device. By attaching a counter to the ROM address port several consecutive addresses can be ``stepped through'' to output a message. Attach a Counter (\textit{Memory} library) with 12 Data Bits to the address port of the ROM, as in Figure \ref{fig:09-02}.

\begin{figure}[H]
	\centering
	\includegraphics[width=\maxwidth{.95\linewidth}]{gfx/09-02}
	\caption{ROM With Counter}
	\label{fig:09-02}
\end{figure}

According to Wikipedia\footnote{\url{https://en.wikipedia.org/wiki/Magic_8-Ball}}, the Magic 8-Ball featured 20 sayings: 

\begin{verbatim}
 1 001 It is certain
 2 00f It is decidedly so
 3 022 Without a doubt
 4 032 Yes definitely
 5 041 You may rely on it
 6 054 As I see it yes
 7 064 Most likely
 8 070 Outlook good
 9 07d Yes
10 081 Signs point to yes
11 094 Reply hazy try again
12 0a9 Ask again later
13 0b8 Better not tell you now
14 0d1 Cannot predict now
15 0e4 Concentrate and ask again
16 0fe Do not count on it
17 111 My reply is no
18 120 My sources say no
19 132 Outlook not so good
20 146 Very doubtful
\end{verbatim}

The Magic 8-Ball simulator built in this lab uses those same 20 saying. In the above chart, each saying is numbered and the start point in ROM for each saying is also noted. Thus, saying one starts on ROM byte 000, saying two starts on ROM byte 00f, saying three starts on ROM byte 022, and so forth.

The content of the ROM device must be loaded before it can be used and that content is provided in \emph{\texttt{Lab09\_ROM.txt}} accompanying this lab. To load the ROM device, click it one time and then click the ``(click to edit)'' link in its properties panel. In the ROM editor window that pops up, click the ``open'' button and navigate to the ROM memory file. Click ``close window'' to load the ROM device and make it ready for service.

The start point for each saying, as indicated on the above table, is stored in Constants (\textit{Wiring} library) and a Mux (\textit{Plexers} library) with five select bits is used to channel the start byte in ROM Memory for a specific message to the counter. Figure \ref{fig:09-03} illustrates the circuit at this point.

\begin{figure}[H]
	\centering
	\includegraphics[width=\maxwidth{.95\linewidth}]{gfx/09-03}
	\caption{ROM Filter Mux}
	\label{fig:09-03}
\end{figure}

A five-bit Random Generator (\textit{Memory} library) is used to select a message at random. Figure \ref{fig:09-04} illustrates the placement of the random generator.

\begin{figure}[H]
	\centering
	\includegraphics[width=\maxwidth{.95\linewidth}]{gfx/09-04}
	\caption{Random Generator Added}
	\label{fig:09-04}
\end{figure}

Next, a one-shot generator is added to the circuit. This is a simple subcircuit that is designed such that when activated it will output a high signal for a single clock pulse and then return to a low. In this circuit, when the reset signal goes high the first flip-flop changes so \textit{Q} is high. On the next clock pulse, the second flip-flop changes and \textit{Q} goes high. On the next pulse both flip-flops return to their quiescent state. The circuit has also added a clock that links to a Tunnel (\textit{Wiring} library) and then to the input of the one-shot. There are two other tunnels connected to the one-shot and they will be linked shortly.

\begin{figure}[H]
	\centering
	\includegraphics[width=\maxwidth{.95\linewidth}]{gfx/09-05}
	\caption{One-Shot Added}
	\label{fig:09-05}
\end{figure}

To complete the circuit, a few odds-and-ends were added.

\begin{itemize}
	\item Four signals were added to control the counter. Those signals are made available from tunnels and are actually generated elsewhere in the circuit.
	\item Tho signals were added to the one-shot subcircuit. The enable is linked through an \texttt{AND} gate to the clock of the random generator. Thus, whenever a reset signal is received the random generator will choose another 8-ball saying at random. Also, \textit{ttyClr} is generated to clear the teletype device on the \lstinline[columns=fixed]|main| circuit.
	\item The output of the ROM device is connected to the \textit{ttyOut} port in order to drive the teletype device on the \lstinline[columns=fixed]|main| circuit.
	\item Note that at the output of the ROM device is a splitter. ASCII letters are only seven bits wide so this splitter passes bits 0-6 to the \textit{ttyOut} port but bit 7 (the most significant bit) is simply discarded.
	\item Near the output of the ROM device, an \texttt{AND} gate feeds the \textit{ttyClk} signal, which is used on the \lstinline[columns=fixed]|main| circuit to clock the teletype device.
	\item The Bit Finder (\textit{Arithmetic} library) attached to the output of the ROM device is used to find the lowest-order one in the ROM byte. If the ROM byte includes at least one one then the south port is high but it goes low if the ROM byte is all zeros. That is the \textit{ena} signal that enables the clock and, when it goes low, permits a \textit{rst} signal (generated on the \lstinline[columns=fixed]|main| circuit when the user ``asks another question'') to create a new answer.
\end{itemize}

Figure \ref{fig:09-06} illustrates the complete Magic 8-Ball circuit.

\begin{figure}[H]
	\centering
	\includegraphics[width=\maxwidth{.95\linewidth}]{gfx/09-06}
	\caption{Complete Magic 8-Ball Circuit}
	\label{fig:09-06}
\end{figure}

The only remaining step is to create the \lstinline[columns=fixed]|main| circuit. As in all labs in this manual, the \lstinline[columns=fixed]|main| circuit does nothing more than provide a user interface for the Magic 8-Ball Circuit. Figure \ref{fig:09-07} illustrates the \lstinline[columns=fixed]|main| circuit.

\begin{figure}[H]
	\centering
	\includegraphics[width=\maxwidth{.95\linewidth}]{gfx/09-07}
	\caption{Magic 8-Ball Main Circuit}
	\label{fig:09-07}
\end{figure}

\subsection{Testing the Circuit}

Before the circuit can be tested the ROM device must be loaded. The ROM was loaded earlier in the lab but in case it does not have any content (it is filled with zeros), then load it with \emph{\texttt{Lab09\_ROM.txt}}, which was provided with the lab. To load the ROM device, click it one time and then click the ``(click to edit)'' link in its properties panel. In the ROM editor window that pops up, click the ``open'' button and find the ROM memory file. Click ``close window'' to load the ROM device and make it ready for service.

The circuit should be tested by enabling the simulator clock at a frequency of 16K Hertz. Every time the \textit{Reset} button is pressed a new random message will be displayed on the teletype screen.

\section{Deliverable}

To receive a grade for this lab, build this circuit. Be sure the standard identifying information is at the top left of the \lstinline{main} circuit, similar to: 

\bigskip
% The minipage environment keeps the three lines together - no page break.
\begin{minipage}{\linewidth}
	\begin{verbatim}
	George Self
	Lab 09: ROM
	February 16, 2018
	\end{verbatim}
\end{minipage}
\bigskip

Save the file with this name: \textit{Lab09\_ROM} and submit that file for grading.

 
%%**************************************************************
% Lab 10: RAM
%**************************************************************
\chapter{RAM}

\section{Purpose}

This lab is used to demonstrate how a \acf{RAM} device operates. 

\section{Procedure}

A RAM (\textit{Memory} library) device is similar to a ROM device as used in Lab \ref{Lab09}, \nameref{Lab09}. A RAM device has an address input port, a data port, and several control ports. An address is loaded in the Address Port then on the next clock signal the device either reads the data at that address and outputs it on the data port or inputs whatever is on the data port and writes it to that address. Figure \ref{fig:10-01} illustrates a counter connected to a RAM address port so as the counter outputs an increasing value the RAM will ``step through'' memory locations.

\begin{figure}[H]
	\centering
	\includegraphics[width=\maxwidth{.95\linewidth}]{gfx/10-01}
	\caption{RAM Basics}
	\label{fig:10-01}
\end{figure}

In operation, a high signal on RAM port M1 enables the write function and the RAM device will store whatever is present on the data ports into the address pointed to on the address port. A high signal on port M2 enables the output function (a ``read'' function) and the RAM device will send whatever is present in the address pointed to on the address port to the data ports.

Notice that the data ports have both an in and out pointing arrow to indicate that those ports are designed for both input and output, depending on the setting of M1 and M2.

Figure \ref{fig:10-02} shows a RAM device with the various control signals.

\begin{figure}[H]
	\centering
	\includegraphics[width=\maxwidth{.95\linewidth}]{gfx/10-02}
	\caption{RAM With Control Signals}
	\label{fig:10-02}
\end{figure}

\marginpar{To simplify the circuit wiring, tunnels are used to transport various signals around the circuit.}At the top left of the subcircuit a button is used to generate a clock pulse. By using a button students can pulse the circuit slowly and observe how the RAM device operates. In an actual circuit that button would be replaced by a Clock (\textit{Wiring} library).

At the top of the circuit is a T Flip-Flop (\textit{Memory} library) that is used to control whether the RAM device is reading or writing data. Because it is important that M1 and M2, the two control ports on the RAM device, are never both high at one time a flip-flop is the perfect controller. The T input on the flip-flop is tied to a constant high so whenever the rd\_wrt button is pressed the RAM device toggles between read and write functions.

The Counter has a Reset button attached that will reset its count to zero so the RAM device will always either read or write from its lowest memory location. In actual practice the counter would need a much more complex circuit to set a specific start point for the RAM device to read or write but for this simple demonstration circuit it is enough to always start read/write operations from the lowest memory location.

The next step is to set up the data bus on the east side of the RAM device. It is important that the bus does not attempt to carry data out of the RAM device at the same time that data are being sent to the RAM device. Thus, control buffers are used to determine the direction of data flow between the RAM device and the data bus. Figure \ref{fig:10-03} shows the data bus with the control buffers.

\begin{figure}[H]
	\centering
	\includegraphics[width=\maxwidth{.95\linewidth}]{gfx/10-03}
	\caption{Data Bus}
	\label{fig:10-03}
\end{figure}

Notice that the outputs of the read/write flip-flop are being used to control the direction of the data flow for the RAM device.

To complete the demonstration circuit, a Keyboard (\textit{Input/Output} library) device is added to write ASCII characters into RAM memory and a TTY (\textit{Input/Output} library) device is used to display ASCII characters read from RAM memory. Figure \ref{fig:10-04} shows the input/output devices.

\begin{figure}[H]
	\centering
	\includegraphics[width=\maxwidth{.95\linewidth}]{gfx/10-04}
	\caption{RAM With Input/Output Devices}
	\label{fig:10-04}
\end{figure}

To operate the keyboard device, click it and enter some text from the computer's keyboard. Then as that device is clocked one ASCII character at a time will be sent to the output port at its south-east corner. As in ASCII devices used in earlier labs, a splitter is used for both the keyboard and TTY display to strip the most significant bit from the data bus since the bus is eight bits wide but ASCII is only a seven-bit code.

Finally, two indicator LEDS have been added to make it clear whether data are being written to RAM or read from RAM.

\subsection{Testing the Circuit}

To test the complete circuit:

\begin{enumerate}
	\item Click Reset to set the counter to zero.
	\item Click the ``rd\_wrt'' button until the ``Write\_to\_RAM'' LED is on.
	\item Click the keyboard device and enter some text.
	\item Click the ``clk'' button to stream the text from the keyboard into RAM. Notice how the RAM device display changes to indicate the ASCII codes that have been stored.
	\item Click Reset to set the counter to zero.
	\item Click the ``clr'' button on the TTY device to clear that display.
	\item Click the ``rd\_wrt'' button until the ``Read\_from\_RAM'' LED is on.
	\item Click the ``clk'' button to stream text from RAM to the TTY device. Notice that this does not remove the text from RAM so it is still available for another reading if desired.
\end{enumerate}

\section{Challenge}

Build the circuit as described in this Lab and ensure that it operates as expected.

\section{Deliverable}

To receive a grade for this lab, complete the Challenge. Be sure the standard identifying information is at the top left of the \lstinline{main} circuit, similar to: 

\bigskip
% The minipage environment keeps the three lines together - no page break.
\begin{minipage}{\linewidth}
	\begin{verbatim}
	George Self
	Lab 10: RAM
	February 16, 2018
	\end{verbatim}
\end{minipage}
\bigskip

Save the file with this name: \textit{Lab10\_RAM} and submit that file for grading.
 

% *********************************************************
% Part 5: Simulation
% *********************************************************
% use \cleardoublepage here to avoid problems with pdfbookmark
\cleardoublepage  
\ctparttext{\textsc{Simulation} is the most complex topic covered in this lab manual. Included in this manual are a simple processor, designed to teach the foundations of a Central Processing Unit, and an elevator simulator, designed to be a capstone project.}  % >>>>> Include <<<<<
\part{Simulation}  % >>>>> Include <<<<<
%%**************************************************************
% Lab 11: Processor
%**************************************************************
\chapter{Processor}

\section{Purpose}

A \acf{CPU} is arguably one of the most important digital logic devices. \acp{CPU} are found in all computers and many other embedded logic devices. They are versatile circuits that can be used to control many processes and peripheral devices. The purpose of this lab is to lay the foundation of \ac{CPU} operation.

\subsection{A Definition} 

When asked to define ``\ac{CPU}'' many students offer poetic definitions like ``it is the brain of the computer.'' This may be somewhat artistic but is not very helpful in defining \ac{CPU} for digital logic purposes. Here is a much better definition:

\begin{quote}
	A \acf{CPU} is a hardware device that is designed to translate binary codes stored in software into signals that control hardware. Thus, a \ac{CPU} is the interface between software and hardware.
\end{quote}

The purpose of this lab is to demonstrate how binary codes can be used to manipulate hardware devices, like registers and adders, to move data through a circuit and accomplish a purpose. While the circuit developed in this lab is not a practical start for a \ac{CPU} is does serve as an introduction to the concept of hardware manipulation by software codes. 

\section{Procedure}

This processor contains only three subcircuits connected by several bus lines and each of the three subcircuits are reasonably simple to understand.

\subsection{Arithmetic-Logic Unit}

This processor starts with a simple \ac{ALU}, as in Figure \ref{fig:11-01}.

\begin{figure}[H]
	\centering
	\includegraphics[width=\maxwidth{.95\linewidth}]{gfx/11-01}
	\caption{Simple ALU}
	\label{fig:11-01}
\end{figure}

To be sure, this \ac{ALU} is not very complex but uses the same principles developed in Lab \ref{lab04}, \nameref{lab04}. It contains only three arithmetic functions, increment, add, and negate, four logic functions, \texttt{AND}, \texttt{OR}, \texttt{XOR}, \texttt{NOT}, and one constant zero output. There are two data input ports but note that some of the functions only use the lower input, and one output port. The multiplexer determines which of the functions will be connected to the output and that is controlled by a signal named \textit{ALUCtl}.

The \ac{ALU} is then expanded somewhat to make it usable in a \ac{CPU}.

\begin{figure}[H]
	\centering
	\includegraphics[width=\maxwidth{.95\linewidth}]{gfx/11-02}
	\caption{Full ALU}
	\label{fig:11-02}
\end{figure}

The simple \ac{ALU} functions are found in the center of Figure \ref{fig:11-02}. However, what started as \textit{DataInA} has been replaced by a register named \textit{ALUBuffer}.\footnote{IMPORTANT NOTE: All registers in this Processor circuit are triggered on the Falling Edge of the clock. The reason for this will become evident when the circuit is tested.} The \textit{ALUBuffer's} inputs are from Tunnels (\textit{Wiring} library) because those inputs are used in more than one location in the subcircuit.\footnote{Tunnels are used extensively in this circuit to simplify the diagrams and aid in tracing signals.}

The \ac{ALU} output is routed through a register named \textit{Acc}, for \textit{Accumulator}, which is the commonly-used name for the \ac{ALU} output in a \ac{CPU} circuit.

On the left side of the subcircuit are the three input ports. \textit{DataIn} is an eight-bit number that is sent to both the \textit{ALUBuffer} and the lower \textit{DataIn} bus. The \textit{ALUCtl} signal is split into two components. Bits 0-2 are sent to the multiplexer to select which of the eight functions will be output. Bit 3 of the \textit{ALUCtl} signal is sent to the \textit{AccEna} tunnel and when that is high the \textit{Acc} register will be enabled but when that signal is low then the \textit{ALUBuffer} register will be enabled. Finally, the clock input is sent to both registers.

\subsection{General Registers}

A \ac{CPU} must have several general registers available to hold data while an instruction is being carried out. For example, to temporarily hold the \textit{Acc} output until it is needed in a later step that value can be stored in a register and then recovered when needed. 

The processor circuit being built in this lab has four general registers. Figure \ref{fig:11-03} illustrates the \lstinline[columns=fixed]|GenReg| subcircuit.

\begin{figure}[H]
	\centering
	\includegraphics[width=\maxwidth{.95\linewidth}]{gfx/11-03}
	\caption{General Registers}
	\label{fig:11-03}
\end{figure}

The \lstinline[columns=fixed]|GenReg| subcircuit does not require any novel digital logic concepts. Starting on the left side of the circuit:

\begin{itemize}
	\item \textit{DataIn} is connected to the data bus and is the main input port for the registers. Note that \textit{DataIn} is connected to the \textit{Data} port on all four registers. 
	\item The register that actually stores the input data is determined by the Decoder (\textit{Plexers} library) in the lower left corner of the subcircuit. The two low-order bits from the \textit{RegSel} signal activate one of the output lines from the Decoder and that line is tied to the Write Enable port of the register. On the next clock pulse that register will lock in the data present on the \textit{DataIn} port.
	\item The outputs from all of the registers are wired to a Multiplexer (\textit{Plexers} library). The select bits from the Decoder that are used to select the storage register are also used to select the register output line which is, in turn, wired to the \textit{DataOut} port.
	\item The high-order bit from the \textit{RegSel} control signal is used to determine if data are stored to or read from a register. When that bit is high the decoder is active and will select a storage register but when that bit is low the output multiplexer will be activated and send a register's stored value to the output port.
\end{itemize}

\subsection{Control}

The \lstinline[columns=fixed]|Control| subcircuit in this device is very simple and could, in all actuality, be eliminated. However, in a true \ac{CPU} the \lstinline[columns=fixed]|Control| subcircuit is rather complex and critical to the operation of the circuit so a \lstinline[columns=fixed]|Control| subcircuit is included in this lab as an example. Figure \ref{fig:11-04} illustrates the \lstinline[columns=fixed]|Control| subcircuit.

\begin{figure}[H]
	\centering
	\includegraphics[width=\maxwidth{.95\linewidth}]{gfx/11-04}
	\caption{Control Subcircuit}
	\label{fig:11-04}
\end{figure}

The \lstinline[columns=fixed]|Control| subcircuit includes a nine-bit input named \textit{mCode} (for ``Microcode''). That input is latched by a register\footnote{Note, as an exception to the other registers in the Processor circuit, the register in the control subcircuit must be set to trigger on the leading edge of the clock rather than the falling edge.} and the output of that register is split into three components.

\begin{description}
	\item[Bits 0-3] These are the \ac{ALU} control bits and they are sent to the \lstinline[columns=fixed]|ALU| subcircuit.
	\item[Bits 4-6] These are the register control bits and are sent to that subcircuit.
	\item[Bits 7-8] These are the \textit{dBus} (``Data Bus'') control bits. The data bus is found in the \lstinline[columns=fixed]|main| circuit and carries the data to each of the subcircuits. The dBus control is just a multiplexer that controls which subcircuit's output has control of the data bus.
\end{description}

\subsection{Main}

The \lstinline[columns=fixed]|main| circuit ties the three subcircuits together with three control busses and one data bus. Figure \ref{fig:11-05} illustrates the \lstinline[columns=fixed]|main| circuit.

\begin{figure}[H]
	\centering
	\includegraphics[width=\maxwidth{.95\linewidth}]{gfx/11-05}
	\caption{Main Circuit}
	\label{fig:11-05}
\end{figure}

There are no novel digital logic functions used in this circuit. The first input is \textit{mCode} which is the microcode used to control the flow of data in the dBus (``data bus''). the other input, \textit{LdImm} (``Load Immediate'') can contain an eight-bit number that is to be loaded into one of the registers for processing. In a full \ac{CPU} that input would be wired to a \ac{RAM} device.

\subsection{Testing the Circuit}

The circuit should be tested by inputting these signals and observing the output.

\subsubsection{Copy LdImm To R0}

Enter some value in the \textit{LdImm} input port, set the \textit{mCode} input to 101001000 (the first three values in the table below), and then pulse the \textit{clk}. When completed, the \textit{dBus} and \textit{R0} should both contain the value of the \textit{LdImm} port.

\begin{table}[H]
	\sffamily
	\newcommand{\head}[1]{\textcolor{white}{\textbf{#1}}}		
	\begin{center}
		\rowcolors{2}{gray!10}{white} % Color every other line a light gray
		\begin{tabular}{ccccl} 
			\textbf{dBus} & \textbf{Reg} & \textbf{ALU} & \textbf{dBus} & \textbf{Notes} \\
			10 & 100 & 1000 & LdImm & R0 <- LdImm \\
		\end{tabular}
	\end{center}
	\caption{R0 <- LdImm}
	\label{tab:11-01}
\end{table}

\subsubsection{Copy LdImm To R1}

Enter some value in the \textit{LdImm} input port, set the \textit{mCode} input to 101011000 (the first three values in the table below), and then pulse the \textit{clk}. When completed, the \textit{dBus} and \textit{R1} should both contain the value of the \textit{LdImm} port.

\begin{table}[H]
	\sffamily
	\newcommand{\head}[1]{\textcolor{white}{\textbf{#1}}}		
	\begin{center}
		\rowcolors{2}{gray!10}{white} % Color every other line a light gray
		\begin{tabular}{ccccl} 
			\textbf{dBus} & \textbf{Reg} & \textbf{ALU} & \textbf{dBus} & \textbf{Notes} \\
			10 & 101 & 1000 & LdImm & R1 <- LdImm
		\end{tabular}
	\end{center}
	\caption{R1 <- LdImm}
	\label{tab:11-02}
\end{table}

\subsubsection{Copy LdImm To ALU}

Enter some value in the \textit{LdImm} input port, set the \textit{mCode} input to 100110000 (the first three values in the table below), and then pulse the \textit{clk}. When completed, the \textit{dBus} and \textit{ALU} should both contain the value of the \textit{LdImm} port.

\begin{table}[H]
	\sffamily
	\newcommand{\head}[1]{\textcolor{white}{\textbf{#1}}}		
	\begin{center}
		\rowcolors{2}{gray!10}{white} % Color every other line a light gray
		\begin{tabular}{ccccl} 
			\textbf{dBus} & \textbf{Reg} & \textbf{ALU} & \textbf{dBus} & \textbf{Notes} \\
			10 & 011 & 0000 & LdImm & ALU <- LdImm
		\end{tabular}
	\end{center}
	\caption{ALU <- LdImm}
	\label{tab:11-03}
\end{table}

\subsubsection{Increment dBus}

Set the \textit{mCode} input to 000001000 (the first three values in the table below), and then pulse the \textit{clk}. When completed, the original value of the \textit{dBus} will be incremented by one.

\begin{table}[H]
	\sffamily
	\newcommand{\head}[1]{\textcolor{white}{\textbf{#1}}}		
	\begin{center}
		\rowcolors{2}{gray!10}{white} % Color every other line a light gray
		\begin{tabular}{ccccl} 
			\textbf{dBus} & \textbf{Reg} & \textbf{ALU} & \textbf{dBus} & \textbf{Notes} \\
			00 & 000 & 1000 & dBus+1 & dBus <- Inc(dBus)
		\end{tabular}
	\end{center}
	\caption{dBus <- Inc(dBus)}
	\label{tab:11-04}
\end{table}

\subsubsection{Add R0 And R1, Store In R0}

\marginpar{Use the LdImm function to initialize R0 and R1.}Adding the values of \textit{R0} and \textit{R1} and storing the result in \textit{R0} requires three steps. Set the \textit{mCode} input to the first three values in the table below and pulse the \textit{clk} for each of the steps. When completed, the sum of the original values of \textit{R0} and \textit{R1} will be stored in \textit{R0}.

\begin{table}[H]
	\sffamily
	\newcommand{\head}[1]{\textcolor{white}{\textbf{#1}}}		
	\begin{center}
		\rowcolors{2}{gray!10}{white} % Color every other line a light gray
		\begin{tabular}{ccccl} 
			\textbf{dBus} & \textbf{Reg} & \textbf{ALU} & \textbf{dBus} & \textbf{Notes} \\
			01 & 001 & 0001 & R1 & ALU <- R1 \\
			01 & 000 & 1001 & R0 & Acc <- R0 + R1 \\
			00 & 100 & 0001 & Acc & R0 <- Acc 
		\end{tabular}
	\end{center}
	\caption{R0 <- R0 + R1}
	\label{tab:11-05}
\end{table}

\subsubsection{Subtract R1 From R0, Store In R0}

\marginpar{Use the LdImm function to initialize R0 and R1.}Subtracting the value of \textit{R1} from \textit{R0} and storing the result in \textit{R0} requires four steps. Set the \textit{mCode} input to the first three values in the table below and pulse the \textit{clk} for each of the steps. When completed, the difference of the original values of \textit{R0} and \textit{R1} will be stored in \textit{R0}.

\begin{table}[H]
	\sffamily
	\newcommand{\head}[1]{\textcolor{white}{\textbf{#1}}}		
	\begin{center}
		\rowcolors{2}{gray!10}{white} % Color every other line a light gray
		\begin{tabular}{ccccl} 
			\textbf{dBus} & \textbf{Reg} & \textbf{ALU} & \textbf{dBus} & \textbf{Notes} \\
			01 & 000 & 0010 & R0 & ALU <- R0 \\
			01 & 001 & 1010 & R1 & Acc <- \textasciitilde R1 \\
			00 & 100 & 1001 & R0-R1 & dBus <- Acc \\
			00 & 100 & 0111 & dBus+1 & R0 <- R0 - R1
		\end{tabular}
	\end{center}
	\caption{R0 <- R0 - R1}
	\label{tab:11-06}
\end{table}

\subsubsection{Copy R0 to R1}

\marginpar{Use the LdImm function to initialize R0.}Copying the value of \textit{R0} to \textit{R1} requires four steps. Set the \textit{mCode} input to the first three values in the table below and pulse the \textit{clk} for each of the steps. When completed, the value of \textit{R0} will be stored in \textit{R1}.

\begin{table}[H]
	\sffamily
	\newcommand{\head}[1]{\textcolor{white}{\textbf{#1}}}		
	\begin{center}
		\rowcolors{2}{gray!10}{white} % Color every other line a light gray
		\begin{tabular}{ccccl} 
			\textbf{dBus} & \textbf{Reg} & \textbf{ALU} & \textbf{dBus} & \textbf{Notes} \\
			00 & 000 & 1111 & 0 & dBus <- 0 \\
			00 & 000 & 0100 & 0 & ALU <- dBus \\
			01 & 000 & 1100 & Acc & Acc <- ALU OR R0 \\
			00 & 101 & 0111 & Acc & R1 <- Acc
		\end{tabular}
	\end{center}
	\caption{R1 <- R0}
	\label{tab:11-07}
\end{table}

\subsubsection{Swap R0 And R1}

\marginpar{Use the LdImm function to initialize R0 and R1.}Swapping the values of \textit{R0} and \textit{R1} requires 12 steps. Set the \textit{mCode} input to the first three values in the table below and pulse the \textit{clk} for each of the steps. When completed, the values of \textit{R0} and \textit{R1} will exchanged.

\begin{table}[H]
	\sffamily
	\newcommand{\head}[1]{\textcolor{white}{\textbf{#1}}}		
	\begin{center}
		\rowcolors{2}{gray!10}{white} % Color every other line a light gray
		\begin{tabular}{ccccl} 
			\textbf{dBus} & \textbf{Reg} & \textbf{ALU} & \textbf{dBus} & \textbf{Notes} \\
			00 & 000 & 1111 & 0 & dBus <- 0 (Move R0 to R2)\\
			00 & 000 & 0100 & 0 & ALU <- dBus \\
			01 & 000 & 1100 & Acc & Acc <- ALU OR R0 \\
			00 & 110 & 0111 & Acc & R2 <- Acc \\
 			& & & & \\ %Empty line

			00 & 000 & 1111 & 0 & dBus <- 0 (Move R1 to R0)\\
			00 & 000 & 0100 & 0 & ALU <- dBus \\
			01 & 001 & 1100 & Acc & Acc <- ALU OR R1 \\
			00 & 100 & 0111 & Acc & R0 <- Acc \\
 			& & & & \\ %Empty line

			00 & 000 & 1111 & 0 & dBus <- 0 (Move R2 to R1)\\
			00 & 000 & 0100 & 0 & ALU <- dBus \\
			01 & 010 & 1100 & Acc & Acc <- ALU OR R2 \\
			00 & 101 & 0111 & Acc & R1 <- Acc
		\end{tabular}
	\end{center}
	\caption{R0 <-> R1}
	\label{tab:11-08}
\end{table}

\section{About Programming Languages}

The codes that were input for the last example (swap \textit{R0} and \textit{R1}) would create the following program.

\begin{Verbatim}[frame=lines,
xleftmargin=10mm,
xrightmargin=10mm]
000001111
000000100
010001100
001100111
000001111
000000100
010011100
001000111
000001111
000000100
010101100
001010111
\end{Verbatim}

This group of instructions would be considered ``CPU Microcode,'' which is a very highly specialized form of programming. It is the code that is built into a \ac{CPU} circuit and it determines what gates, registers, and other devices are active for each step of the code. When Intel, AMD, Motorola, or other manufacturers create a new \ac{CPU}, one of their main challenges is creating the microcode that will, for example, ``add the contents of register one to the contents of register two and store the result in register zero.'' The microcode must be able to activate and deactivate various devices within the \ac{CPU} so data appear on the appropriate bus at the right time in order to achieve the objective. Normally, microcode steps must be executed over several clock cycles in order to do a single job. For example, in one clock cycle the contents of register one may be placed on the data bus, the next clock cycle will load that data into the ALU register, and so forth until the entire process is complete.

Microcode is usually stored in \ac{ROM} that is built into the \ac{CPU}. This is typically called ``firmware'' since it is a string of ones and zeros, like software, but it cannot be changed, like hardware.

It is important to keep in mind the difference between instructions contained in a software program (like Word) and those contained in microcode. A single instruction in software is interpreted and executed by the \ac{CPU} using, perhaps, dozens of microcode steps. As an example, the software may want to move a single byte from \ac{RAM} to the video card. The \ac{CPU} may process that instruction by first moving the byte from \ac{RAM} to register one and then moving it from there to the video card's input register. Those moves may require several clock cycles as various multiplexers and other devices are activated in the correct sequence to move the data to its destination. 

In summary, the CPU's Control Unit controls all of the registers, buffers, buses, and other devices in the CPU in order to execute the instructions requested by the software program.

A computer program, as contained in software, is nothing more than a series of ones and zeros, organized into groups of 32 (or 64). Each group of bits forms a single ``word'' of information; or a single instruction which would then be used to trigger a microcode sequence within the \ac{CPU}. When viewed at the level of ones and zeros, a program is said to be in ``machine code,'' and could look something like this:

\begin{Verbatim}[frame=lines,
xleftmargin=10mm,
xrightmargin=10mm]
10010100101100101001101011001010
01101001101011000111101011101011
00011011110010000111010111100101
\end{Verbatim}

If a programmer could master machine code, then those programs would be as concise and efficient as possible since they would be written in machine code the \ac{CPU} can execute directly. Of course, as it is easy to imagine, no one actually writes machine code due to its complexity.

The next level higher than machine code is called ``Assembly'' code. Assembly uses easy-to-remember abbreviations to represent the various \ac{CPU} instructions available; and it looks something like this: 

\begin{Verbatim}[frame=lines,
xleftmargin=10mm,
xrightmargin=10mm]
INP
STA FIRST 
INP
STA SECOND 
LDA FIRST 
SUB SECOND
OUT
HLT
FIRST DAT
SECOND DAT
\end{Verbatim}

Once the program has been written in Assembly, it must be ``assembled'' into machine code before it can be executed. An assembler is a fairly simply program that converts a file containing assembly codes into machine codes that can be executed by the \ac{CPU}.

Many programming languages have been developed that are considered ``higher'' than Assembly; for example, C++, Java, and Visual Basic. These languages tend to be easy to master and can enable a programmer to quickly create very complex programs. Programs written in each of these languages must be compiled, or changed into machine code, before they can be executed. Here is an example Java program:

\begin{Verbatim}[frame=lines,
xleftmargin=10mm,
xrightmargin=10mm]
public class HelloWorldExample{
  public static void main(String args[]){
  System.out.println("Hello World !"); 
  }
}
\end{Verbatim}

In the end, while there are dozens of different programming languages, they are all designed to be reduced into a series of machine codes, which the \ac{CPU} can then execute.

\section{Challenge}

Using the examples in the ``Testing the Circuit'' section, create the microcode necessary to carry out these functions:

\begin{enumerate}
	\item Store the value contained in \textit{LdImm} in \textit{R2} (\textit{R2} <- \textit{LdImm}). (Assume that \textit{LdImm} is pre-loaded with the value to store.)
	\item Store the value contained in \textit{LdImm} in \textit{R3} (\textit{R3} <- \textit{LdImm}). (Assume that \textit{LdImm} is pre-loaded with the value to store.)
	\item Store the 2s complement of the value in \textit{R0} back into \textit{R0} (\textit{R0} <- \textasciitilde \textit{R0}). The subtraction example will help with this function.
	\item Store the bitwise NOT of the value in \textit{R0} back into \textit{R0} (\textit{R0} <- \textit{R0'}).
\end{enumerate}

\section{Deliverable}

To receive a grade for this lab, build the Processor circuit and then complete the Challenge. Be sure the standard identifying information is at the top left of the Processor \lstinline{main} circuit, similar to: 

\bigskip
% The minipage environment keeps the three lines together - no page break.
\begin{minipage}{\linewidth}
	\begin{verbatim}
	George Self
	Lab 11: Processor
	April 5, 2018
	\end{verbatim}
\end{minipage}
\bigskip

Save the Processor circuit in a file with this name: \textit{Lab11\_Processor}. Complete the code required in the Challenge and store that in a text file with the name \textit{Lab11\_Code.txt}. Submit both files for grading.
 
%%%%% Do Not Use Yet - it ain't ready! %**************************************************************
% Lab 12: CPU
%**************************************************************
\chapter{Central Processing Unit}\label{lab12}

\section{Introduction}

\subsection{First Steps}

When building a \ac{CPU}, the first few steps do not involve slinging gates into a circuit. As in most things, planning will save time when executing.

\subsubsection{Purpose}

The first question to answer when building a \ac{CPU} is its purpose. It makes a great deal of difference whether the \ac{CPU} is intended for a single purpose of some sort or a more generalized application. The better the purpose can be defined the simpler (and easier) the designing job becomes. For example, there is no need to design a \ac{CPU} with a lot of ``bells and whistles'' that is intended to be embedded in a microwave oven. A simple 4-bit \ac{CPU} can do that job.

\subsubsection{Instruction Set}

After the \acp{CPU} purpose has been defined, the instruction set must be designed. In general, the fewer instructions that are needed for the \ac{CPU} to do its job, the simpler the design will be. Of course, the designer must include enough instructions to ensure the \ac{CPU} can be effective.\footnote{As an aside, there has been some theoretical work concerning the smallest possible instruction set that can still be used to create a working \ac{CPU}. The current record is one: one instruction is all that is necessary to create a functional \ac{CPU}. While some would argue which instruction is best; in general, computer scientists agree that ``Subtract and Branch if Less Than or Equal to Zero'' is most efficient. To find out more about this type of \ac{CPU}, search for ``SUBLEQ.''}

\subsubsection{States}

The next step is to define the various states the \ac{CPU} can enter and what should happen in each state. While that discussion is beyond the scope of this lesson, in general the \acp{CPU} various functions are mapped out so the designer can create circuits to match each function.

A \acp{CPU} states can be divided into three broad categories: Fetch, Decode, Execute. During the Fetch state, an instruction is fetched from memory and brought into the \ac{CPU}. The instruction is next decoded so the \ac{CPU} knows what type of instruction was fetched. Finally, the microcode steps required by the instruction are executed. The entire cycle repeats until the program is completed. Here is a generic diagram of the instruction cycle:

\bigskip

\smartdiagram[circular diagram:clockwise]{%
	Fetch,Decode,Execute
}

\section{Purpose}

%%%%%%%%%%%%%%%%%%%%%%%%%%%%%%%%%%%%%%%%%%%%%%%%%%%%%%%%%%%%%%
% After I create the CPU I need to start here and update this part of the lab.
%%%%%%%%%%%%%%%%%%%%%%%%%%%%%%%%%%%%%%%%%%%%%%%%%%%%%%%%%%%%%%

A \ac{CPU} contains several components that are tied together with bus lines. Figure \ref{fig:11-01} is a block diagram that shows the main components of the \ac{CPU} being studied in this book: 

\begin{figure}[H]
	\centering
	\includegraphics[width=\maxwidth{.95\linewidth}]{gfx/11-01}
	\caption{Simple Processor}
	\label{fig:11-01}
\end{figure}

Here is the purpose for each of these components (from the top of the diagram):

\begin{itemize}
	\item The Ctrl circuit controls all functions within the \ac{CPU}. The output of the Ctrl circuit is placed on the Control Bus where it can then be used to turn on/off various multiplexors and control buffers in order to control the data flow between the components on the data bus. In operation, the \ac{CPU} “fetches” the next program instruction from memory, and then the control circuit interprets that instruction and opens or closes data paths as appropriate.
	\item The Arithmetic Logic Unit (ALU) is responsible for all data manipulation, such as adding two numbers. The ALU sends the results of its work to a register called the Accumulator (which is internal to the ALU on this \ac{CPU}). 
	\item The Gen Regs (General Registers) are a number of registers that are used to temporarily store information while it is being processed. These registers may be considered the “scratch pad” of the \ac{CPU}.
	\item The Pgm Cnter (Program Counter) keeps track of the address of the memory location that contains the next program instruction to be executed. That instruction is then “fetched” and processed by the Ctrl circuit.
	\item The Addr Reg (Address Register) contains a RAM address that is important for the current operation; perhaps, for example, a single byte of an ASCII encoded message that needs to be displayed on the screen. The output of the Address Register is sent directly to RAM through the Address Bus. 
	\item RAM contains the program that is currently being executed. Most of the information traveling on the Data Bus either comes from or is going to RAM.
	\item Peripherals are devices that the \ac{CPU} must control; such as a hard drive or monitor.
\end{itemize}

\footnote{Much of the material in this unit was adapted from a similar lab written by Dr. Lawlor for TKGate, another logic simulator. Here is the URL for that lab: \url{http://www.cs.uaf.edu/2008/fall/cs441/lecture/09_09_\ac{CPU}_construction.html}}

\section{Procedure}










\subsection{Testing the Circuit}

The circuit should be tested by ...

\section{Challenge}

Whatever

\section{Deliverable}

To receive a grade for this lab, build the \ac{CPU} circuit and then complete the Challenge. Be sure the standard identifying information is at the top left of the \ac{CPU} \lstinline{main} circuit, similar to: 

\bigskip
% The minipage environment keeps the three lines together - no page break.
\begin{minipage}{\linewidth}
	\begin{verbatim}
	George Self
	Lab 12: CPU
	April 30, 2018
	\end{verbatim}
\end{minipage}
\bigskip

Save the \ac{CPU} circuit in a file with this name: \textit{Lab12\_CPU}. Complete the code required in the Challenge and store that in a Word or Text file with the name \textit{Lab12\_Code}. Submit both files for grading.

 
%\include{Chapters/13_Elevator} 

% *********************************************************
% Backmatter
% *********************************************************
\appendix  
%\renewcommand{\thechapter}{\alph{chapter}}
\cleardoublepage 
\part{Appendix}  
%********************************************************************
% Appendix
%*******************************************************
% If problems with the headers: get headings in appendix etc. right
%\markboth{\spacedlowsmallcaps{Appendix}}{\spacedlowsmallcaps{Appendix}}

%********************************************************************
% TTL Reference
%********************************************************************
\chapter{TTL Reference}

\LE includes a number of \ac{TTL} \acp{IC}. These are pre-packaged digital logic circuits that perform specific, well-defined functions. There are, literally, hundreds of \ac{TTL} \acp{IC} available for purchase from electronics warehouses but \LE includes only $ 35 $ of the most commonly-used devices. Figure \ref{fig:50-74HC595} shows three surface-mounted \acp{IC} on a circuit board.

\begin{figure}[H]
	\centering
	\includegraphics[width=\maxwidth{.75\linewidth}]{gfx/50-74HC595}
	\caption{Three Surface-Mounted Integrated Circuits}
	\label{fig:50-74HC595}
\end{figure}

\section{7400: Quad 2-Input NAND Gate}

This device contains four independent 2-input NAND gates. Figure \ref{fig:50-7400} is a logic diagram of one of the four circuits.

\begin{figure}[H]
	\centering
	\includegraphics{gfx/50-7400}
	\caption{7400: Single NAND Gate Circuit}
	\label{fig:50-7400}
\end{figure}

The 7400 device in \LE uses the wiring connections indicated in Table \ref{tab:50-7400}.

\begin{table}[H]
	\sffamily
	\newcommand{\head}[1]{\textcolor{white}{\textbf{#1}}}		
	\begin{center}
		\rowcolors{2}{gray!10}{white} % Color every other line a light gray
		\begin{tabular}{rl} 
			\rowcolor{black!75}
			\head{Logisim Label} & \head{Function} \\
			Input: 1   & In 1A  \\
			Input: 2   & In 1B  \\
			Output: 3  & Out 1Y \\
			Input: 4   & In 2A  \\
			Input: 5   & In 2B  \\
			Output: 6  & Out 2Y \\
			Output: 8  & Out 3Y \\
			Input: 9   & In 3A  \\
			Input: 10  & In 3B  \\
			Output: 11 & Out 4Y \\
			Input: 12  & In 4A  \\
			Input: 13  & In 4B  \\
		\end{tabular}
	\end{center}
	\caption{Pinout For 7400}
	\label{tab:50-7400}
\end{table}

\section{7402: Quad 2-Input NOR Gate}

This device contains four independent 2-input NOR gates. Figure \ref{fig:50-7402} is a logic diagram of one of the four circuits.

\begin{figure}[H]
	\centering
	\includegraphics{gfx/50-7402}
	\caption{7402: Single NOR Gate Circuit}
	\label{fig:50-7402}
\end{figure}

The 7402 device in \LE uses the wiring connections indicated in Table \ref{tab:50-7402}.

\begin{table}[H]
	\sffamily
	\newcommand{\head}[1]{\textcolor{white}{\textbf{#1}}}		
	\begin{center}
		\rowcolors{2}{gray!10}{white} % Color every other line a light gray
		\begin{tabular}{rl} 
			\rowcolor{black!75}
			\head{Logisim Label} & \head{Function} \\
			Input: 1   & In 1A  \\
			Input: 2   & In 1B  \\
			Output: 3  & Out 1Y \\
			Input: 4   & In 2A  \\
			Input: 5   & In 2B  \\
			Output: 6  & Out 2Y \\
			Output: 8  & Out 3Y \\
			Input: 9   & In 3A  \\
			Input: 10  & In 3B  \\
			Output: 11 & Out 4Y \\
			Input: 12  & In 4A  \\
			Input: 13  & In 4B  \\
		\end{tabular}
	\end{center}
	\caption{Pinout For 7402}
	\label{tab:50-7402}
\end{table}

\section{7404: Hex Inverter}

This device contains six independent inverters. Figure \ref{fig:50-7404} is a logic diagram of one of the six circuits.

\begin{figure}[H]
	\centering
	\includegraphics{gfx/50-7404}
	\caption{7404: Single Inverter Circuit}
	\label{fig:50-7404}
\end{figure}

The 7404 device in \LE uses the wiring connections indicated in Table \ref{tab:50-7404}.

\begin{table}[H]
	\sffamily
	\newcommand{\head}[1]{\textcolor{white}{\textbf{#1}}}		
	\begin{center}
		\rowcolors{2}{gray!10}{white} % Color every other line a light gray
		\begin{tabular}{rl} 
			\rowcolor{black!75}
			\head{Logisim Label} & \head{Function} \\
			Input: 1   & In 1  \\
			Output: 2  & Out 1  \\
			Input: 3   & In 2 \\
			Output: 4  & Out 2  \\
			Input: 5   & In 3  \\
			Output: 6  & Out 3 \\
			Output: 8  & Out 4  \\
			Input: 9   & In 4  \\
			Output: 10 & Out 5  \\
			Input: 11  & In 5  \\
			Output: 12 & Out 6 \\
			Input: 13  & In 6  \\
		\end{tabular}
	\end{center}
	\caption{Pinout For 7404}
	\label{tab:50-7404}
\end{table}

\section{7408: Quad 2-Input AND Gate}

This device contains four independent 2-input AND gates. Figure \ref{fig:50-7408} is a logic diagram of one of the four circuits.

\begin{figure}[H]
	\centering
	\includegraphics{gfx/50-7408}
	\caption{7408: Single AND Gate Circuit}
	\label{fig:50-7408}
\end{figure}

The 7408 device in \LE uses the wiring connections indicated in Table \ref{tab:50-7408}.

\begin{table}[H]
	\sffamily
	\newcommand{\head}[1]{\textcolor{white}{\textbf{#1}}}		
	\begin{center}
		\rowcolors{2}{gray!10}{white} % Color every other line a light gray
		\begin{tabular}{rl} 
			\rowcolor{black!75}
			\head{Logisim Label} & \head{Function} \\
			Input: 1   & In 1A  \\
			Input: 2   & In 1B  \\
			Output: 3  & Out 1Y \\
			Input: 4   & In 2A  \\
			Input: 5   & In 2B  \\
			Output: 6  & Out 2Y \\
			Output: 8  & Out 3Y \\
			Input: 9   & In 3A  \\
			Input: 10  & In 3B  \\
			Output: 11 & Out 4Y \\
			Input: 12  & In 4A  \\
			Input: 13  & In 4B  \\
		\end{tabular}
	\end{center}
	\caption{Pinout For 7408}
	\label{tab:50-7408}
\end{table}

\section{7410: Triple 3-Input NAND Gate}

This device contains three independent 3-input NAND gates. Figure \ref{fig:50-7410} is a logic diagram of one of the three circuits.

\begin{figure}[H]
	\centering
	\includegraphics{gfx/50-7410}
	\caption{7410: Single 3-Input NAND Gate Circuit}
	\label{fig:50-7410}
\end{figure}

The 7410 device in \LE uses the wiring connections indicated in Table \ref{tab:50-7410}.

\begin{table}[H]
	\sffamily
	\newcommand{\head}[1]{\textcolor{white}{\textbf{#1}}}		
	\begin{center}
		\rowcolors{2}{gray!10}{white} % Color every other line a light gray
		\begin{tabular}{rl} 
			\rowcolor{black!75}
			\head{Logisim Label} & \head{Function} \\
			Input: 1   & In 1A  \\
			Input: 2   & In 1B  \\
			Input: 3   & In 2A \\
			Input: 4   & In 2B  \\
			Input: 5   & In 2C  \\
			Output: 6  & Out 2Y \\
			Output: 8  & Out 3Y \\
			Input: 9   & In 3A  \\
			Input: 10  & In 3B  \\
			Input: 11  & In 3C \\
			Output: 12 & Out 1Y  \\
			Input: 13  & In 1C  \\
		\end{tabular}
	\end{center}
	\caption{Pinout For 7410}
	\label{tab:50-7410}
\end{table}

\section{7411: Triple 3-Input AND Gate}

This device contains three independent 3-input AND gates. Figure \ref{fig:50-7411} is a logic diagram of one of the three circuits.

\begin{figure}[H]
	\centering
	\includegraphics{gfx/50-7411}
	\caption{7411: Single 3-Input AND Gate Circuit}
	\label{fig:50-7411}
\end{figure}

The 7411 device in \LE uses the wiring connections indicated in Table \ref{tab:50-7411}.

\begin{table}[H]
	\sffamily
	\newcommand{\head}[1]{\textcolor{white}{\textbf{#1}}}		
	\begin{center}
		\rowcolors{2}{gray!10}{white} % Color every other line a light gray
		\begin{tabular}{rl} 
			\rowcolor{black!75}
			\head{Logisim Label} & \head{Function} \\
			Input: 1   & In 1A  \\
			Input: 2   & In 1B  \\
			Input: 3   & In 2A \\
			Input: 4   & In 2B  \\
			Input: 5   & In 2C  \\
			Output: 6  & Out 2Y \\
			Output: 8  & Out 3Y \\
			Input: 9   & In 3A  \\
			Input: 10  & In 3B  \\
			Input: 11  & In 3C \\
			Output: 12 & Out 1Y  \\
			Input: 13  & In 1C  \\
		\end{tabular}
	\end{center}
	\caption{Pinout For 7411}
	\label{tab:50-7411}
\end{table}

\section{7413: Dual 4-Input NAND Gate (Schmitt-Trigger)}

This device contains two independent 4-input NAND gates. Schmitt-triggers are a special type of device that are used to filter out spurious noise on a circuit. They are designed to change from low-to-high or high-to-low only when the input voltage reaches a preset level but not if the voltage randomly fluctuates without crossing the set-points. This device is essentially the same as the 7418. Figure \ref{fig:50-7413} is a logic diagram of one of the two circuits.

\begin{figure}[H]
	\centering
	\includegraphics{gfx/50-7413}
	\caption{7413: Single 4-Input NAND Gate Circuit}
	\label{fig:50-7413}
\end{figure}

The 7413 device in \LE uses the wiring connections indicated in Table \ref{tab:50-7413}.

\begin{table}[H]
	\sffamily
	\newcommand{\head}[1]{\textcolor{white}{\textbf{#1}}}		
	\begin{center}
		\rowcolors{2}{gray!10}{white} % Color every other line a light gray
		\begin{tabular}{rl} 
			\rowcolor{black!75}
			\head{Logisim Label} & \head{Function} \\
			Input: 1    & In A0  \\
			Input: 2    & In B0  \\
			Pin 3: NC   & Not Connected \\
			Input: 4    & In C0  \\
			Input: 5    & In D0  \\
			Output: 6   & Out Y0 \\
			Output: 8   & Out Y1 \\
			Input: 9    & In D1  \\
			Input: 10   & In C1  \\
			Pin 11: NC  & Not Connected \\
			Input: 12   & In B1  \\
			Input: 13    & In A1  \\
		\end{tabular}
	\end{center}
	\caption{Pinout For 7413}
	\label{tab:50-7413}
\end{table}

\section{7414: Hex Inverter (Schmitt-Trigger)}

This device contains six independent inverters. Schmitt-triggers are a special type of device that are used to filter out spurious noise on a circuit. They are designed to change from low-to-high or high-to-low only when the input voltage reaches a preset level but not if the voltage randomly fluctuates without crossing the set-points. This device is essentially the same as the 7419. Figure \ref{fig:50-7414} is a logic diagram of one of the six circuits.

\begin{figure}[H]
	\centering
	\includegraphics{gfx/50-7404}
	\caption{7414: Single Inverter Circuit}
	\label{fig:50-7414}
\end{figure}

The 7414 device in \LE uses the wiring connections indicated in Table \ref{tab:50-7414}.

\begin{table}[H]
	\sffamily
	\newcommand{\head}[1]{\textcolor{white}{\textbf{#1}}}		
	\begin{center}
		\rowcolors{2}{gray!10}{white} % Color every other line a light gray
		\begin{tabular}{rl} 
			\rowcolor{black!75}
			\head{Logisim Label} & \head{Function} \\
			Input: 1   & In 1  \\
			Output: 2  & Out 1  \\
			Input: 3   & In 2 \\
			Output: 4  & Out 2  \\
			Input: 5   & In 3  \\
			Output: 6  & Out 3 \\
			Output: 8  & Out 4  \\
			Input: 9   & In 4  \\
			Output: 10 & Out 5  \\
			Input: 11  & In 5  \\
			Output: 12 & Out 6 \\
			Input: 13  & In 6  \\
		\end{tabular}
	\end{center}
	\caption{Pinout For 7414}
	\label{tab:50-7414}
\end{table}

\section{7418: Dual 4-Input NAND Gate (Schmitt-Trigger Inputs)}

This device contains two independent 4-input NAND gates. Schmitt-triggers are a special type of device that are used to filter out spurious noise on a circuit. They are designed to change from low-to-high or high-to-low only when the input voltage reaches a preset level but not if the voltage randomly fluctuates without crossing the set-points. This device is essentially the same as the 7413. Figure \ref{fig:50-7418} is a logic diagram of one of the two circuits.

\begin{figure}[H]
	\centering
	\includegraphics{gfx/50-7413}
	\caption{7418: Single 4-Input NAND Gate Circuit}
	\label{fig:50-7418}
\end{figure}

The 7418 device in \LE uses the wiring connections indicated in Table \ref{tab:50-7418}.

\begin{table}[H]
	\sffamily
	\newcommand{\head}[1]{\textcolor{white}{\textbf{#1}}}		
	\begin{center}
		\rowcolors{2}{gray!10}{white} % Color every other line a light gray
		\begin{tabular}{rl} 
			\rowcolor{black!75}
			\head{Logisim Label} & \head{Function} \\
			Input: 1   & In A0  \\
			Input: 2   & In B0  \\
			Pin 3 NC   & Not Connected \\
			Input: 4   & In C0  \\
			Input: 5   & In D0  \\
			Output: 6  & Out Y0 \\
			Output: 8  & Out Y1 \\
			Input: 9   & In D1  \\
			Input: 10  & In C1  \\
			Pin 11 NC  & Not Connected \\
			Input: 12 & In B1  \\
			Input: 13  & In A1  \\
		\end{tabular}
	\end{center}
	\caption{Pinout For 7418}
	\label{tab:50-7418}
\end{table}

\section{7419: Hex Inverter (Schmitt-Trigger)}

This device contains six independent inverters. Schmitt-triggers are a special type of device that are used to filter out spurious noise on a circuit. They are designed to change from low-to-high or high-to-low only when the input voltage reaches a preset level but not if the voltage randomly fluctuates without crossing the set-points. This device is essentially the same as the 7414. Figure \ref{fig:50-7419} is a logic diagram of one of the six circuits.

\begin{figure}[H]
	\centering
	\includegraphics{gfx/50-7404}
	\caption{7419: Single Inverter Circuit}
	\label{fig:50-7419}
\end{figure}

The 7419 device in \LE uses the wiring connections indicated in Table \ref{tab:50-7419}.

\begin{table}[H]
	\sffamily
	\newcommand{\head}[1]{\textcolor{white}{\textbf{#1}}}		
	\begin{center}
		\rowcolors{2}{gray!10}{white} % Color every other line a light gray
		\begin{tabular}{rl} 
			\rowcolor{black!75}
			\head{Logisim Label} & \head{Function} \\
			Input: 1   & In 1  \\
			Output: 2  & Out 1  \\
			Input: 3   & In 2 \\
			Output: 4  & Out 2  \\
			Input: 5   & In 3  \\
			Output: 6  & Out 3 \\
			Output: 8  & Out 4  \\
			Input: 9   & In 4  \\
			Output: 10 & Out 5  \\
			Input: 11  & In 5  \\
			Output: 12 & Out 6 \\
			Input: 13  & In 6  \\
		\end{tabular}
	\end{center}
	\caption{Pinout For 7419}
	\label{tab:50-7419}
\end{table}

\section{7420: Dual 4-Input NAND Gate}

This device contains two independent 4-input NAND gates. Figure \ref{fig:50-7420} is a logic diagram of one of the two circuits.

\begin{figure}[H]
	\centering
	\includegraphics{gfx/50-7413}
	\caption{7420: Single 4-Input NAND Gate Circuit}
	\label{fig:50-7420}
\end{figure}

The 7420 device in \LE uses the wiring connections indicated in Table \ref{tab:50-7420}.

\begin{table}[H]
	\sffamily
	\newcommand{\head}[1]{\textcolor{white}{\textbf{#1}}}		
	\begin{center}
		\rowcolors{2}{gray!10}{white} % Color every other line a light gray
		\begin{tabular}{rl} 
			\rowcolor{black!75}
			\head{Logisim Label} & \head{Function} \\
			Input: 1   & In A0  \\
			Input: 2   & In B0  \\
			Pin 3 NC   & Not Connected \\
			Input: 4   & In C0  \\
			Input: 5   & In D0  \\
			Output: 6  & Out Y0 \\
			Output: 8  & Out Y1 \\
			Input: 9   & In D1  \\
			Input: 10  & In C1  \\
			Pin 11 NC  & Not Connected \\
			Input: 12 & In B1  \\
			Input: 13  & In A1  \\
		\end{tabular}
	\end{center}
	\caption{Pinout For 7420}
	\label{tab:50-7420}
\end{table}

\section{7421: Dual 4-Input AND Gate}

This device contains two independent 4-input AND gates. Figure \ref{fig:50-7421} is a logic diagram of one of the two circuits.

\begin{figure}[H]
	\centering
	\includegraphics{gfx/50-7421}
	\caption{7421: Single 4-Input AND Gate Circuit}
	\label{fig:50-7421}
\end{figure}

The 7421 device in \LE uses the wiring connections indicated in Table \ref{tab:50-7421}.

\begin{table}[H]
	\sffamily
	\newcommand{\head}[1]{\textcolor{white}{\textbf{#1}}}		
	\begin{center}
		\rowcolors{2}{gray!10}{white} % Color every other line a light gray
		\begin{tabular}{rl} 
			\rowcolor{black!75}
			\head{Logisim Label} & \head{Function} \\
			Input: 1   & In A0  \\
			Input: 2   & In B0  \\
			Pin 3 NC   & Not Connected \\
			Input: 4   & In C0  \\
			Input: 5   & In D0  \\
			Output: 6  & Out Y0 \\
			Output: 8  & Out Y1 \\
			Input: 9   & In D1  \\
			Input: 10  & In C1  \\
			Pin 11 NC  & Not Connected \\
			Input: 12 & In B1  \\
			Input: 13  & In A1  \\
		\end{tabular}
	\end{center}
	\caption{Pinout For 7421}
	\label{tab:50-7421}
\end{table}

\section{7424: Quad 2-Input NAND Gate (Schmitt-Trigger)}

This device contains four independent 2-input NAND gates. Schmitt-triggers are a special type of device that are used to filter out spurious noise on a circuit. They are designed to change from low-to-high or high-to-low only when the input voltage reaches a preset level but not if the voltage randomly fluctuates without crossing the set-points. This device is essentially the same as the 7400.  Figure \ref{fig:50-7424} is a logic diagram of one of the four circuits.

\begin{figure}[H]
	\centering
	\includegraphics{gfx/50-7400}
	\caption{7424: Single NAND Gate Circuit}
	\label{fig:50-7424}
\end{figure}

The 7424 device in \LE uses the wiring connections indicated in Table \ref{tab:50-7424}.

\begin{table}[H]
	\sffamily
	\newcommand{\head}[1]{\textcolor{white}{\textbf{#1}}}		
	\begin{center}
		\rowcolors{2}{gray!10}{white} % Color every other line a light gray
		\begin{tabular}{rl} 
			\rowcolor{black!75}
			\head{Logisim Label} & \head{Function} \\
			Input: 1   & In 1A  \\
			Input: 2   & In 1B  \\
			Output: 3  & Out 1Y \\
			Input: 4   & In 2A  \\
			Input: 5   & In 2B  \\
			Output: 6  & Out 2Y \\
			Output: 8  & Out 3Y \\
			Input: 9   & In 3A  \\
			Input: 10  & In 3B  \\
			Output: 11 & Out 4Y \\
			Input: 12  & In 4A  \\
			Input: 13  & In 4B  \\
		\end{tabular}
	\end{center}
	\caption{Pinout For 7424}
	\label{tab:50-7424}
\end{table}

\section{7427: Triple 3-Input NOR Gate}

This device contains three independent 3-input NOR gates. Figure \ref{fig:50-7427} is a logic diagram of one of the three circuits.

\begin{figure}[H]
	\centering
	\includegraphics{gfx/50-7427}
	\caption{7411: Single 3-Input NOR Gate Circuit}
	\label{fig:50-7427}
\end{figure}

The 7427 device in \LE uses the wiring connections indicated in Table \ref{tab:50-7427}.

\begin{table}[H]
	\sffamily
	\newcommand{\head}[1]{\textcolor{white}{\textbf{#1}}}		
	\begin{center}
		\rowcolors{2}{gray!10}{white} % Color every other line a light gray
		\begin{tabular}{rl} 
			\rowcolor{black!75}
			\head{Logisim Label} & \head{Function} \\
			Input: 1   & In 1A  \\
			Input: 2   & In 1B  \\
			Input: 3   & In 2A \\
			Input: 4   & In 2B  \\
			Input: 5   & In 2C  \\
			Output: 6  & Out 2Y \\
			Output: 8  & Out 3Y \\
			Input: 9   & In 3A  \\
			Input: 10  & In 3B  \\
			Input: 11  & In 3C \\
			Output: 12 & Out 1Y  \\
			Input: 13  & In 1C  \\
		\end{tabular}
	\end{center}
	\caption{Pinout For 7427}
	\label{tab:50-7427}
\end{table}

\section{7430: Single 8-Input NAND Gate}

This device contains a single 8-input NAND gate. The logic for this gate is $ Y = \overline{A \cdot B \cdot C \cdot D \cdot E \cdot F \cdot G \cdot H} $. Figure \ref{fig:50-7430} is a logic diagram of the circuit.

\begin{figure}[H]
	\centering
	\includegraphics{gfx/50-7430}
	\caption{7430: Single 8-Input NAND Gate}
	\label{fig:50-7430}
\end{figure}

The 7430 device in \LE uses the wiring connections indicated in Table \ref{tab:50-7430}.

\begin{table}[H]
	\sffamily
	\newcommand{\head}[1]{\textcolor{white}{\textbf{#1}}}		
	\begin{center}
		\rowcolors{2}{gray!10}{white} % Color every other line a light gray
		\begin{tabular}{rl} 
			\rowcolor{black!75}
			\head{Logisim Label} & \head{Function} \\
			Input: 1    & In A \\
			Input: 2    & In B \\
			Input: 3    & In C \\
			Input: 4    & In D \\
			Input: 5    & In E \\
			Input: 6    & In F \\
			Output: 8   & Out Y \\
			Pin 9: NC   & Not Connected \\
			Pin 10: NC  & Not Connected \\
			Input: 11   & In G \\
			Input: 12   & In H  \\
			Pin 13: NC  & Not Connected  \\
		\end{tabular}
	\end{center}
	\caption{Pinout For 7430}
	\label{tab:50-7430}
\end{table}

\section{7432: Quad 2-Input OR Gate}

This device contains four independent 2-input OR gates. Figure \ref{fig:50-7432} is a logic diagram of one of the four circuits.

\begin{figure}[H]
	\centering
	\includegraphics{gfx/50-7432}
	\caption{7432: Single OR Gate Circuit}
	\label{fig:50-7432}
\end{figure}

The 7432 device in \LE uses the wiring connections indicated in Table \ref{tab:50-7432}.

\begin{table}[H]
	\sffamily
	\newcommand{\head}[1]{\textcolor{white}{\textbf{#1}}}		
	\begin{center}
		\rowcolors{2}{gray!10}{white} % Color every other line a light gray
		\begin{tabular}{rl} 
			\rowcolor{black!75}
			\head{Logisim Label} & \head{Function} \\
			Input: 1   & In 1A  \\
			Input: 2   & In 1B  \\
			Output: 3  & Out 1Y \\
			Input: 4   & In 2A  \\
			Input: 5   & In 2B  \\
			Output: 6  & Out 2Y \\
			Output: 8  & Out 3Y \\
			Input: 9   & In 3A  \\
			Input: 10  & In 3B  \\
			Output: 11 & Out 4Y \\
			Input: 12  & In 4A  \\
			Input: 13  & In 4B  \\
		\end{tabular}
	\end{center}
	\caption{Pinout For 7432}
	\label{tab:50-7432}
\end{table}

\section{7436: Quad 2-Input NOR Gate}

This device contains four independent 2-input NOR gates. This device is essentially the same as the 7402. Figure \ref{fig:50-7436} is a logic diagram of one of the four circuits.

\begin{figure}[H]
	\centering
	\includegraphics{gfx/50-7402}
	\caption{7436: Single NOR Gate Circuit}
	\label{fig:50-7436}
\end{figure}

The 7436 device in \LE uses the wiring connections indicated in Table \ref{tab:50-7436}.

\begin{table}[H]
	\sffamily
	\newcommand{\head}[1]{\textcolor{white}{\textbf{#1}}}		
	\begin{center}
		\rowcolors{2}{gray!10}{white} % Color every other line a light gray
		\begin{tabular}{rl} 
			\rowcolor{black!75}
			\head{Logisim Label} & \head{Function} \\
			Input: 1   & In 1A  \\
			Input: 2   & In 1B  \\
			Output: 3  & Out 1Y \\
			Input: 4   & In 2A  \\
			Input: 5   & In 2B  \\
			Output: 6  & Out 2Y \\
			Output: 8  & Out 3Y \\
			Input: 9   & In 3A  \\
			Input: 10  & In 3B  \\
			Output: 11 & Out 4Y \\
			Input: 12  & In 4A  \\
			Input: 13  & In 4B  \\
		\end{tabular}
	\end{center}
	\caption{Pinout For 7436}
	\label{tab:50-7436}
\end{table}

\section{7442: BCD to Decimal Decoder}

This device takes a BDC input and deactivates a single line corresponding to the input number. It is often called a ``One-Of-Ten'' decoder. As an example, if $ 0111_{BCD} $ is input then line 7-of-10 will go low while all other outputs will remain high. Figure \ref{fig:50-7442} illustrates a 7442 \ac{IC} in a very simple circuit.

\begin{figure}[H]
	\centering
	\includegraphics[width=\maxwidth{.95\linewidth}]{gfx/50-7442}
	\caption{7442: BCD to Decimal Decoder}
	\label{fig:50-7442}
\end{figure}

Table \ref{tab:50-7442a} is the truth table for this device. Any BCD input greater than $ 1001 $ is ignored and all outputs will be high for those inputs.

\begin{table}[H]
	\sffamily
	\newcommand{\head}[1]{\textcolor{white}{\textbf{#1}}}		
	\begin{center}
		%\rowcolors{2}{gray!10}{white} % Color every other line light gray - Note: makes the vline vanish
		\begin{tabular}{cccc | cccccccccc} 
			\rowcolor{black!75}
			\multicolumn{4}{c}{\head{Inputs}} & \multicolumn{10}{c}{\head{Output}} \\
			\textbf{A} & \textbf{B} & \textbf{C} & \textbf{D} & \textbf{0} & \textbf{1} & \textbf{2} & \textbf{3} & \textbf{4} & \textbf{5} & \textbf{6} & \textbf{7} & \textbf{8} & \textbf{9} \\
			\hline
			0 & 0 & 0 & 0  & 0 & 1 & 1 & 1 & 1 & 1 & 1 & 1 & 1 & 1 \\
			0 & 0 & 0 & 1  & 1 & 0 & 1 & 1 & 1 & 1 & 1 & 1 & 1 & 1 \\
			0 & 0 & 1 & 0  & 1 & 1 & 0 & 1 & 1 & 1 & 1 & 1 & 1 & 1 \\
			0 & 0 & 1 & 1  & 1 & 1 & 1 & 0 & 1 & 1 & 1 & 1 & 1 & 1 \\
			0 & 1 & 0 & 0  & 1 & 1 & 1 & 1 & 0 & 1 & 1 & 1 & 1 & 1 \\
			0 & 1 & 0 & 1  & 1 & 1 & 1 & 1 & 1 & 0 & 1 & 1 & 1 & 1 \\
			0 & 1 & 1 & 0  & 1 & 1 & 1 & 1 & 1 & 1 & 0 & 1 & 1 & 1 \\
			0 & 1 & 1 & 1  & 1 & 1 & 1 & 1 & 1 & 1 & 1 & 0 & 1 & 1 \\
			1 & 0 & 0 & 0  & 1 & 1 & 1 & 1 & 1 & 1 & 1 & 1 & 0 & 1 \\
			1 & 0 & 0 & 1  & 1 & 1 & 1 & 1 & 1 & 1 & 1 & 1 & 1 & 0 \\
		\end{tabular}
	\end{center}
	\caption{Truth Table For The 7442 Circuit}
	\label{tab:50-7442a}
\end{table}

The 7442 device in \LE uses the wiring connections indicated in Table \ref{tab:50-7442b}.

\begin{table}[H]
	\sffamily
	\newcommand{\head}[1]{\textcolor{white}{\textbf{#1}}}		
	\begin{center}
		\rowcolors{2}{gray!10}{white} % Color every other line a light gray
		\begin{tabular}{rl} 
			\rowcolor{black!75}
			\head{Logisim Label} & \head{Function} \\
			Output 1: O0  & Out 0 \\
			Output 2: O1  & Out 1 \\
			Output 3: O2  & Out 2 \\
			Output 4: O3  & Out 3 \\
			Output 5: O4  & Out 4 \\
			Output 6: O5  & Out 5 \\
			Output 7: O6  & Out 6 \\
			Output 8: O7  & Out 7 \\
			Output 10: O8 & Out 8 \\
			Output 11: O9 & Out 9 \\
			Input 12: D   & In D  \\
			Input 13: C   & In C  \\
			Input 14: B   & In B  \\
			Input 15: A   & In A  \\
		\end{tabular}
	\end{center}
	\caption{Pinout For 7442}
	\label{tab:50-7442b}
\end{table}

\section{7443: Excess-3 to Decimal Decoder}

This device takes an Excess-3 input and deactivates a single line corresponding to the input number. It is often called a ``One-Of-Ten'' decoder. As an example, if $ 0011_{Ex3} $ is input then line 0-of-10 will go low while all other outputs will remain high. This is wired in exactly the same way as the 7442 \ac{IC} illustrated in Figure \ref{fig:50-7442}.

Table \ref{tab:50-7443a} is the truth table for this device. Any input numbers other than those found in the truth table are ignored and all outputs will be high for those inputs.

\begin{table}[H]
	\sffamily
	\newcommand{\head}[1]{\textcolor{white}{\textbf{#1}}}		
	\begin{center}
		%\rowcolors{2}{gray!10}{white} % Color every other line light gray - Note: makes the vline vanish
		\begin{tabular}{cccc | cccccccccc} 
			\rowcolor{black!75}
			\multicolumn{4}{c}{\head{Inputs}} & \multicolumn{10}{c}{\head{Output}} \\
			\textbf{A} & \textbf{B} & \textbf{C} & \textbf{D} & \textbf{0} & \textbf{1} & \textbf{2} & \textbf{3} & \textbf{4} & \textbf{5} & \textbf{6} & \textbf{7} & \textbf{8} & \textbf{9} \\
			\hline
			0 & 0 & 1 & 1  & 0 & 1 & 1 & 1 & 1 & 1 & 1 & 1 & 1 & 1 \\
			0 & 1 & 0 & 0  & 1 & 0 & 1 & 1 & 1 & 1 & 1 & 1 & 1 & 1 \\
			0 & 1 & 0 & 1  & 1 & 1 & 0 & 1 & 1 & 1 & 1 & 1 & 1 & 1 \\
			0 & 1 & 1 & 0  & 1 & 1 & 1 & 0 & 1 & 1 & 1 & 1 & 1 & 1 \\
			0 & 1 & 1 & 1  & 1 & 1 & 1 & 1 & 0 & 1 & 1 & 1 & 1 & 1 \\
			1 & 0 & 0 & 0  & 1 & 1 & 1 & 1 & 1 & 0 & 1 & 1 & 1 & 1 \\
			1 & 0 & 0 & 1  & 1 & 1 & 1 & 1 & 1 & 1 & 0 & 1 & 1 & 1 \\
			1 & 0 & 1 & 0  & 1 & 1 & 1 & 1 & 1 & 1 & 1 & 0 & 1 & 1 \\
			1 & 0 & 1 & 1  & 1 & 1 & 1 & 1 & 1 & 1 & 1 & 1 & 0 & 1 \\
			1 & 1 & 0 & 0  & 1 & 1 & 1 & 1 & 1 & 1 & 1 & 1 & 1 & 0 \\
		\end{tabular}
	\end{center}
	\caption{Truth Table For The 7443 Circuit}
	\label{tab:50-7443a}
\end{table}

The 7443 device in \LE uses the wiring connections indicated in Table \ref{tab:50-7443b}.

\begin{table}[H]
	\sffamily
	\newcommand{\head}[1]{\textcolor{white}{\textbf{#1}}}		
	\begin{center}
		\rowcolors{2}{gray!10}{white} % Color every other line a light gray
		\begin{tabular}{rl} 
			\rowcolor{black!75}
			\head{Logisim Label} & \head{Function} \\
			Output 1: O0  & Out 0 \\
			Output 2: O1  & Out 1 \\
			Output 3: O2  & Out 2 \\
			Output 4: O3  & Out 3 \\
			Output 5: O4  & Out 4 \\
			Output 6: O5  & Out 5 \\
			Output 7: O6  & Out 6 \\
			Output 8: O7  & Out 7 \\
			Output 10: O8 & Out 8 \\
			Output 11: O9 & Out 9 \\
			Input 12: D   & In D  \\
			Input 13: C   & In C  \\
			Input 14: B   & In B  \\
			Input 15: A   & In A  \\
		\end{tabular}
	\end{center}
	\caption{Pinout For 7443}
	\label{tab:50-7443b}
\end{table}

\section{7444: Gray to Decimal Decoder}

This device takes a Gray Excess Code, which is a combination of Gray and Excess-3 Codes, input and deactivates a single line corresponding to the input number. It is often called a ``One-Of-Ten'' decoder. As an example, if $ 1100_{GrayEx3} $ is input then line 5-of-10 will go low while all other outputs will remain high. This is wired in exactly the same way as the 7442 \ac{IC} illustrated in Figure \ref{fig:50-7442}.

Table \ref{tab:50-7444a} is the truth table for this device. Any input numbers other than those found in the truth table are ignored and all outputs will be high for those inputs.

\begin{table}[H]
	\sffamily
	\newcommand{\head}[1]{\textcolor{white}{\textbf{#1}}}		
	\begin{center}
		%\rowcolors{2}{gray!10}{white} % Color every other line light gray - Note: makes the vline vanish
		\begin{tabular}{cccc | cccccccccc} 
			\rowcolor{black!75}
			\multicolumn{4}{c}{\head{Inputs}} & \multicolumn{10}{c}{\head{Output}} \\
			\textbf{A} & \textbf{B} & \textbf{C} & \textbf{D} & \textbf{0} & \textbf{1} & \textbf{2} & \textbf{3} & \textbf{4} & \textbf{5} & \textbf{6} & \textbf{7} & \textbf{8} & \textbf{9} \\
			\hline
			0 & 0 & 1 & 0  & 0 & 1 & 1 & 1 & 1 & 1 & 1 & 1 & 1 & 1 \\
			0 & 1 & 1 & 0  & 1 & 0 & 1 & 1 & 1 & 1 & 1 & 1 & 1 & 1 \\
			0 & 1 & 1 & 1  & 1 & 1 & 0 & 1 & 1 & 1 & 1 & 1 & 1 & 1 \\
			0 & 1 & 0 & 1  & 1 & 1 & 1 & 0 & 1 & 1 & 1 & 1 & 1 & 1 \\
			0 & 1 & 0 & 0  & 1 & 1 & 1 & 1 & 0 & 1 & 1 & 1 & 1 & 1 \\
			1 & 1 & 0 & 0  & 1 & 1 & 1 & 1 & 1 & 0 & 1 & 1 & 1 & 1 \\
			1 & 1 & 0 & 1  & 1 & 1 & 1 & 1 & 1 & 1 & 0 & 1 & 1 & 1 \\
			1 & 1 & 1 & 1  & 1 & 1 & 1 & 1 & 1 & 1 & 1 & 0 & 1 & 1 \\
			1 & 1 & 1 & 0  & 1 & 1 & 1 & 1 & 1 & 1 & 1 & 1 & 0 & 1 \\
			1 & 0 & 1 & 0  & 1 & 1 & 1 & 1 & 1 & 1 & 1 & 1 & 1 & 0 \\
		\end{tabular}
	\end{center}
	\caption{Truth Table For The 7444 Circuit}
	\label{tab:50-7444a}
\end{table}

The 7443 device in \LE uses the wiring connections indicated in Table \ref{tab:50-7444b}.

\begin{table}[H]
	\sffamily
	\newcommand{\head}[1]{\textcolor{white}{\textbf{#1}}}		
	\begin{center}
		\rowcolors{2}{gray!10}{white} % Color every other line a light gray
		\begin{tabular}{rl} 
			\rowcolor{black!75}
			\head{Logisim Label} & \head{Function} \\
			Output 1: O0  & Out 0 \\
			Output 2: O1  & Out 1 \\
			Output 3: O2  & Out 2 \\
			Output 4: O3  & Out 3 \\
			Output 5: O4  & Out 4 \\
			Output 6: O5  & Out 5 \\
			Output 7: O6  & Out 6 \\
			Output 8: O7  & Out 7 \\
			Output 10: O8 & Out 8 \\
			Output 11: O9 & Out 9 \\
			Input 12: D   & In D  \\
			Input 13: C   & In C  \\
			Input 14: B   & In B  \\
			Input 15: A   & In A  \\
		\end{tabular}
	\end{center}
	\caption{Pinout For 7444}
	\label{tab:50-7444b}
\end{table}

\section{7447: BCD to 7-Segment Decoder}

This device takes a BCD Code input and activates a combination of outputs such that a 7-segment display will correctly indicate the input number. Figure \ref{fig:50-7447} illustrates a 7447 \ac{IC} in a very simple circuit.

\begin{figure}[H]
	\centering
	\includegraphics{gfx/50-7447}
	\caption{7447: BCD to 7-Segment Decoder}
	\label{fig:50-7447}
\end{figure}

Table \ref{tab:50-7447} is the truth table for this device. 

\begin{table}[H]
	\sffamily
	\newcommand{\head}[1]{\textcolor{white}{\textbf{#1}}}		
	\begin{center}
		%\rowcolors{2}{gray!10}{white} % Color every other line light gray - Note: makes the vline vanish
		\begin{tabular}{cccc | ccccccc} 
			\rowcolor{black!75}
			\multicolumn{4}{c}{\head{Inputs}} & \multicolumn{7}{c}{\head{Output}} \\
			\textbf{A} & \textbf{B} & \textbf{C} & \textbf{D} & \textbf{a} & \textbf{b} & \textbf{c} & \textbf{d} & \textbf{e} & \textbf{f} & \textbf{g} \\
			\hline
			0 & 0 & 0 & 0  & 1 & 1 & 1 & 1 & 1 & 1 & 0 \\
			0 & 0 & 0 & 1  & 0 & 1 & 1 & 0 & 0 & 0 & 0 \\
			0 & 0 & 1 & 0  & 1 & 1 & 0 & 1 & 1 & 0 & 1 \\
			0 & 0 & 1 & 1  & 1 & 1 & 1 & 1 & 0 & 0 & 1 \\
			0 & 1 & 0 & 0  & 0 & 1 & 1 & 0 & 0 & 1 & 1 \\
			0 & 1 & 0 & 1  & 1 & 0 & 1 & 1 & 0 & 1 & 1 \\
			0 & 1 & 1 & 0  & 1 & 0 & 1 & 1 & 1 & 1 & 1 \\
			0 & 1 & 1 & 1  & 1 & 1 & 1 & 0 & 0 & 0 & 0 \\
			1 & 0 & 0 & 0  & 1 & 1 & 1 & 1 & 1 & 1 & 1 \\
			1 & 0 & 0 & 1  & 1 & 1 & 1 & 0 & 0 & 1 & 1 \\
			1 & 0 & 1 & 0  & 1 & 1 & 1 & 0 & 1 & 1 & 1 \\
			1 & 0 & 1 & 1  & 0 & 0 & 1 & 1 & 1 & 1 & 1 \\
			1 & 1 & 0 & 0  & 1 & 0 & 0 & 1 & 1 & 1 & 0 \\
			1 & 1 & 0 & 1  & 0 & 1 & 1 & 1 & 1 & 0 & 1 \\
			1 & 1 & 1 & 0  & 1 & 0 & 0 & 1 & 1 & 1 & 1 \\
			1 & 1 & 1 & 1  & 1 & 0 & 0 & 0 & 1 & 1 & 1 \\
		\end{tabular}
	\end{center}
	\caption{Truth Table For The 7447 Circuit}
	\label{tab:50-7447}
\end{table}

The 7447 device in \LE uses the wiring connections indicated in Table \ref{tab:50-7447a}.

\begin{table}[H]
	\sffamily
	\newcommand{\head}[1]{\textcolor{white}{\textbf{#1}}}		
	\begin{center}
		\rowcolors{2}{gray!10}{white} % Color every other line a light gray
		\begin{tabular}{rl} 
			\rowcolor{black!75}
			\head{Logisim Label} & \head{Function} \\
			Input 1: B   & B   \\
			Input 2: C   & C   \\
			Input 3: LT  & LT  \\
			Input 4: BI  & BI  \\
			Input 5: RBI & RBI \\
			Input 6: D   & D   \\
			Input 7: A   & A   \\
			Output 8: e  & e   \\
			Output 10: d & d   \\
			Output 11: c & c   \\
			Output 12: b & b   \\
			Output 13: a & a   \\
			Output 14: g & g   \\
			Output 15: f & f   \\
		\end{tabular}
	\end{center}
	\caption{Pinout For 7447}
	\label{tab:50-7447a}
\end{table}

\section{7451: Dual AND-OR-INVERT Gate}

This device contains two independent AND-OR-INVERT gates. Figure \ref{fig:50-7451} is a logic diagram of one of the two circuits.

\begin{figure}[H]
	\centering
	\includegraphics{gfx/50-7451}
	\caption{7451: Single AND-OR-INVERT Gate Circuit}
	\label{fig:50-7451}
\end{figure}

The 7451 device in \LE uses the wiring connections indicated in Table \ref{tab:50-7451}.

\begin{table}[H]
	\sffamily
	\newcommand{\head}[1]{\textcolor{white}{\textbf{#1}}}		
	\begin{center}
		\rowcolors{2}{gray!10}{white} % Color every other line a light gray
		\begin{tabular}{rl} 
			\rowcolor{black!75}
			\head{Logisim Label} & \head{Function} \\
			Input 1: A1   & In A1         \\
			Input 2: A2   & In A2         \\
			Input 3: B2   & In B2         \\
			Input 4: C2   & In C2         \\
			Input 5: D2   & In D2         \\
			Output 6: Y2  & Out Y2        \\
			Output 8: Y1  & Out Y1        \\
			Input 9: C1   & In C1         \\
			Input 10: D1  & In D1         \\
			Pin 11: NC    & Not Connected \\
			Pin 12: NC    & Not Connected \\
			Input 13: B1  & In B1         \\
		\end{tabular}
	\end{center}
	\caption{Pinout For 7451}
	\label{tab:50-7451}
\end{table}

\section{7454: Four Wide AND-OR-INVERT Gate}

This device contains a single four-wide AND-OR-INVERT gate. Figure \ref{fig:50-7454} is a logic diagram of the circuit.

\begin{figure}[H]
	\centering
	\includegraphics{gfx/50-7454}
	\caption{7454: Four Wide AND-OR-INVERT Gate Circuit}
	\label{fig:50-7454}
\end{figure}

The 7454 device in \LE uses the wiring connections indicated in Table \ref{tab:50-7454}.

\begin{table}[H]
	\sffamily
	\newcommand{\head}[1]{\textcolor{white}{\textbf{#1}}}		
	\begin{center}
		\rowcolors{2}{gray!10}{white} % Color every other line a light gray
		\begin{tabular}{rl} 
			\rowcolor{black!75}
			\head{Logisim Label} & \head{Function} \\
			Input 1: A   & In A          \\
			Input 2: C   & In C          \\
			Input 3: D   & In D          \\
			Input 4: E   & In E          \\
			Input 5: F   & In F          \\
			Pin 6: NC    & Not Connected \\
			Output 8: Y  & Out Y         \\
			Input 9: G   & In G          \\
			Input 10: H  & In H          \\
			Pin 11: NC   & Not Connected \\
			Pin 12: NC   & Not Connected \\
			Input 13: B  & In B          \\
		\end{tabular}
	\end{center}
	\caption{Pinout For 7454}
	\label{tab:50-7454}
\end{table}

\section{7458: Dual AND-OR Gate}

This device contains a two AND-OR gates. One has three-input AND gates and the other has two-input AND gates. Figure \ref{fig:50-7458} is a logic diagram of the circuit.

\begin{figure}[H]
	\centering
	\includegraphics{gfx/50-7458}
	\caption{7458: Dual AND-OR Gate Circuit}
	\label{fig:50-7458}
\end{figure}

The 7458 device in \LE uses the wiring connections indicated in Table \ref{tab:50-7458}.

\begin{table}[H]
	\sffamily
	\newcommand{\head}[1]{\textcolor{white}{\textbf{#1}}}		
	\begin{center}
		\rowcolors{2}{gray!10}{white} % Color every other line a light gray
		\begin{tabular}{rl} 
			\rowcolor{black!75}
			\head{Logisim Label} & \head{Function} \\
			Input 1: A0   & In A0  \\
			Input 2: A1   & In A1  \\
			Input 3: B1   & In B1  \\
			Input 4: C1   & In C1  \\
			Input 5: D1   & In D1  \\
			Output 6: Y1  & Out Y1 \\
			Output 8: Y0  & Out Y0 \\
			Input 9: D0   & In D0  \\
			Input 10: E0  & In E0 \\
			Input 11: F0  & In F0 \\
			Input 12: B0  & In B0 \\
			Input 13: C0  & In C0 \\
		\end{tabular}
	\end{center}
	\caption{Pinout For 7458}
	\label{tab:50-7458}
\end{table}

\section{7464: 4-2-3-2 AND-OR-INVERT Gate}

This device contains four AND gates of different input sizes that feed a NOR gate. Figure \ref{fig:50-7464} is a logic diagram of the circuit.

\begin{figure}[H]
	\centering
	\includegraphics{gfx/50-7464}
	\caption{7464: 4-2-3-2 AND-OR-INVERT Gate Circuit}
	\label{fig:50-7464}
\end{figure}

The 7464 device in \LE uses the wiring connections indicated in Table \ref{tab:50-7464}.

\begin{table}[H]
	\sffamily
	\newcommand{\head}[1]{\textcolor{white}{\textbf{#1}}}		
	\begin{center}
		\rowcolors{2}{gray!10}{white} % Color every other line a light gray
		\begin{tabular}{rl} 
			\rowcolor{black!75}
			\head{Logisim Label} & \head{Function} \\
			Input 1: A   & In A  \\
			Input 2: E   & In E  \\
			Input 3: F   & In F  \\
			Input 4: G   & In G  \\
			Input 5: H   & In H  \\
			Input 6: I   & In I  \\
			Output 8: Y  & Out Y \\
			Input 9: J   & In J  \\
			Input 10: K  & In K  \\
			Input 11: B  & In B  \\
			Input 12: C  & In C  \\
			Input 13: D  & In D  \\
		\end{tabular}
	\end{center}
	\caption{Pinout For 7464}
	\label{tab:50-7464}
\end{table}

\section{7474: Dual D-Flipflops with Preset and Clear}

This device contains two D-Flipflops, each with its own preset and clear. The 7474 device in \LE uses the wiring connections indicated in Table \ref{tab:50-7474}.

\begin{table}[H]
	\sffamily
	\newcommand{\head}[1]{\textcolor{white}{\textbf{#1}}}		
	\begin{center}
		\rowcolors{2}{gray!10}{white} % Color every other line a light gray
		\begin{tabular}{rl} 
			\rowcolor{black!75}
			\head{Logisim Label} & \head{Function} \\
			Input 1: nCLR1  & On low, clear FF1 \\
			Input 2: D1     & FF1 data input    \\
			Input 3: CLK1   & FF1 clock         \\
			Input 4: nPRE1  & On low, set FF1   \\
			Output 5: Q1    & FF1 Q-out         \\
			Output 6: nQ1   & FF1 Q-not-out     \\
			Output 8: nQ2   & FF2 Q-not-out     \\
			Output 9: Q2    & FF2 Q-out         \\
			Input 10: nPRE2 & On low, set FF2   \\
			Input 11: CLK2  & FF2 clock         \\
			Input 12: D2    & FF2 data input    \\
			Input 13: nCLR2 & On low, clear FF2 \\
		\end{tabular}
	\end{center}
	\caption{Pinout For 7474}
	\label{tab:50-7474}
\end{table}

\section{7485: 4-Bit Magnitude Comparator}

This device compares two 4-bit numbers and outputs one of three values: $ A>B $, $ A=B $, or $ A<B $. It is also designed to be cascaded by including an input port for each of the three values. The 7485 device in \LE uses the wiring connections indicated in Table \ref{tab:50-7485}.

\begin{table}[H]
	\sffamily
	\newcommand{\head}[1]{\textcolor{white}{\textbf{#1}}}		
	\begin{center}
		\rowcolors{2}{gray!10}{white} % Color every other line a light gray
		\begin{tabular}{rl} 
			\rowcolor{black!75}
			\head{Logisim Label} & \head{Function} \\
			Input 1: B3   & Bit B3                 \\
			Input 2: A<B  & Value from prior stage \\
			Input 3: A=B  & Value from prior stage \\
			Input 4: A>B  & Value from prior stage \\
			Output 5: A>B & High if A>B            \\
			Output 6: A=B & High if A=B            \\
			Output 7: A<B & High if A<B            \\
			Input 9: B0   & Bit B0                 \\
			Input 10: A0  & Bit A0                 \\
			Input 11: B1  & Bit B1                 \\
			Input 12: A1  & Bit A1                 \\
			Input 13: A2  & Bit A2                 \\
			Input 14: B2  & Bit B2                 \\
			Input 15: A3  & Bit A3                 \\
		\end{tabular}
	\end{center}
	\caption{Pinout For 7485}
	\label{tab:50-7485}
\end{table}

\section{7486: Quad 2-Input XOR Gate}

This device contains four independent 2-input XOR gates. Figure \ref{fig:50-7486} is a logic diagram of one of the four circuits.

\begin{figure}[H]
	\centering
	\includegraphics{gfx/50-7486}
	\caption{7486: Single XOR Gate Circuit}
	\label{fig:50-7486}
\end{figure}

The 7486 device in \LE uses the wiring connections indicated in Table \ref{tab:50-7486}.

\begin{table}[H]
	\sffamily
	\newcommand{\head}[1]{\textcolor{white}{\textbf{#1}}}		
	\begin{center}
		\rowcolors{2}{gray!10}{white} % Color every other line a light gray
		\begin{tabular}{rl} 
			\rowcolor{black!75}
			\head{Logisim Label} & \head{Function} \\
			Input: 1   & In 1A  \\
			Input: 2   & In 1B  \\
			Output: 3  & Out 1Y \\
			Input: 4   & In 2A  \\
			Input: 5   & In 2B  \\
			Output: 6  & Out 2Y \\
			Output: 8  & Out 3Y \\
			Input: 9   & In 3A  \\
			Input: 10  & In 3B  \\
			Output: 11 & Out 4Y \\
			Input: 12  & In 4A  \\
			Input: 13  & In 4B  \\
		\end{tabular}
	\end{center}
	\caption{Pinout For 7486}
	\label{tab:50-7486}
\end{table}

\section{74125: Quad Bus Buffer, 3-State Gate}

This device contains four independent buffers. When each is enabled with a low on the enable line then the input is passed to the output, when not enabled then the output floats. Figure \ref{fig:50-74125} is a logic diagram of one of the four circuits.

\begin{figure}[H]
	\centering
	\includegraphics{gfx/50-74125}
	\caption{74125: Single Buffer Circuit}
	\label{fig:50-74125}
\end{figure}

The 74125 device in \LE uses the wiring connections indicated in Table \ref{tab:50-74125}.

\begin{table}[H]
	\sffamily
	\newcommand{\head}[1]{\textcolor{white}{\textbf{#1}}}		
	\begin{center}
		\rowcolors{2}{gray!10}{white} % Color every other line a light gray
		\begin{tabular}{rl} 
			\rowcolor{black!75}
			\head{Logisim Label} & \head{Function} \\
			Input: 1   & nEna 1 \\
			Input: 2   & In 1  \\
			Output: 3  & Out 1 \\
			Input: 4   & nEna 2 \\
			Input: 5   & In 2  \\
			Output: 6  & Out 2 \\
			Output: 8  & Out 3 \\
			Input: 9   & In 3  \\
			Input: 10  & nEna 3 \\
			Output: 11 & Out 4 \\
			Input: 12  & In 4  \\
			Input: 13  & nEna 4 \\
		\end{tabular}
	\end{center}
	\caption{Pinout For 74125}
	\label{tab:50-74125}
\end{table}

\section{74165: 8-Bit Parallel-to-Serial Shift Register}

This device can accept data in either parallel or serial form and shift it out in serial form. The 74165 device in \LE uses the wiring connections indicated in Table \ref{tab:50-74165}.

\begin{table}[H]
	\sffamily
	\newcommand{\head}[1]{\textcolor{white}{\textbf{#1}}}		
	\begin{center}
		\rowcolors{2}{gray!10}{white} % Color every other line a light gray
		\begin{tabular}{rl} 
			\rowcolor{black!75}
			\head{Logisim Label} & \head{Function} \\
			Input 1: Shift/Load     & Load when low, shift when high \\
			Input 2: Clock          & Clock                          \\
			Input 3: P4             & Input bit 4                    \\
			Input 4: P5             & Input bit 5                    \\
			Input 5: P6             & Input bit 6                    \\
			Input 6: P7             & Input bit 7                    \\
			Output 7: Q7n           & Complement of serial out       \\
			Output 9: Q7            & Serial out                     \\
			Input 10: Serial Input  & Serial data in                 \\
			Input 11: P0            & Input bit 0                    \\
			Input 12: P1            & Input bit 1                    \\
			Input 13: P2            & Input bit 2                    \\
			Input 14: P3            & Input bit 3                    \\
			Input 15: Clock Inhibit & Clock inhibit                  \\
		\end{tabular}
	\end{center}
	\caption{Pinout For 74165}
	\label{tab:50-74165}
\end{table}


\section{74175: Quad D-Flipflops with Sync Reset}

This device contains four D-Flipflops with a single clock and master reset. The 74175 device in \LE uses the wiring connections indicated in Table \ref{tab:50-74175}.

\begin{table}[H]
	\sffamily
	\newcommand{\head}[1]{\textcolor{white}{\textbf{#1}}}		
	\begin{center}
		\rowcolors{2}{gray!10}{white} % Color every other line a light gray
		\begin{tabular}{rl} 
			\rowcolor{black!75}
			\head{Logisim Label} & \head{Function} \\
			Input 1: nCLR  & On low, clear all FF \\
			Output 2: Q1   & FF1 Q-out            \\
			Output 3: nQ1  & FF1 Q-not-out        \\
			Input 4: D1    & FF1 data input       \\
			Input 5: D2    & FF2 data input       \\
			Output 6: nQ2  & FF2 Q-not-out        \\
			Output 7: Q2   & FF2 Q-out            \\
			Input 9: CLK   & Clock for all FF     \\
			Output 10: Q3  & FF3 Q-out            \\
			Output 11: nQ3 & FF3 Q-not-out        \\
			Input 12: D3   & FF3 data input       \\
			Input 13: D4   & FF4 data input       \\
			Output 14: nQ4 & FF4 Q-not-out        \\
			Output 15: Q4  & FF4 Q-out            \\
		\end{tabular}
	\end{center}
	\caption{Pinout For 74175}
	\label{tab:50-74175}
\end{table}

\section{74266: Quad 2-Input XNOR Gate}

This device contains four independent 2-input XNOR gates. Figure \ref{fig:50-74266} is a logic diagram of one of the four circuits.

\begin{figure}[H]
	\centering
	\includegraphics{gfx/50-74266}
	\caption{74266: Single XNOR Gate Circuit}
	\label{fig:50-74266}
\end{figure}

The 74266 device in \LE uses the wiring connections indicated in Table \ref{tab:50-74266}.

\begin{table}[H]
	\sffamily
	\newcommand{\head}[1]{\textcolor{white}{\textbf{#1}}}		
	\begin{center}
		\rowcolors{2}{gray!10}{white} % Color every other line a light gray
		\begin{tabular}{rl} 
			\rowcolor{black!75}
			\head{Logisim Label} & \head{Function} \\
			Input: 1   & In 1A  \\
			Input: 2   & In 1B  \\
			Output: 3  & Out 1Y \\
			Input: 4   & In 2A  \\
			Input: 5   & In 2B  \\
			Output: 6  & Out 2Y \\
			Output: 8  & Out 3Y \\
			Input: 9   & In 3A  \\
			Input: 10  & In 3B  \\
			Output: 11 & Out 4Y \\
			Input: 12  & In 4A  \\
			Input: 13  & In 4B  \\
		\end{tabular}
	\end{center}
	\caption{Pinout For 74266}
	\label{tab:50-74266}
\end{table}

\section{74273: Octal D-Flipflop with Clear}

This device contains a single 8-bit D-Flipflop with a single clock and master clear. The 74273 device in \LE uses the wiring connections indicated in Table \ref{tab:50-74273}.

\begin{table}[H]
	\sffamily
	\newcommand{\head}[1]{\textcolor{white}{\textbf{#1}}}		
	\begin{center}
		\rowcolors{2}{gray!10}{white} % Color every other line a light gray
		\begin{tabular}{rl} 
			\rowcolor{black!75}
			\head{Logisim Label} & \head{Function} \\
			Input 1: nCLR & On low, clear the FF \\
			Output 2: Q1  & data bit 1 output    \\
			Input 3: D1   & data bit 1 input     \\
			Input 4: D2   & data bit 2 input     \\
			Output 5: Q2  & data bit 2 output    \\
			Output 6: Q3  & data bit 3 output    \\
			Input 7: D3   & data bit 3 input     \\
			Input 8: D4   & data bit 4 input     \\
			Output 9: Q4  & data bit 4 output    \\
			Input 11: CLK & Clock                \\
			Output 12: Q5 & data bit 5 output    \\
			Input 13: D5  & data bit 5 input     \\
			Input 14: D6  & data bit 6 input     \\
			Output 15: Q6 & data bit 6 output    \\
			Output 16: Q7 & data bit 7 output    \\
			Input 17: D7  & data bit 7 input     \\
			Input 18: D8  & data bit 8 input     \\
			Output 19: Q8 & data bit 8 output    \\
		\end{tabular}
	\end{center}
	\caption{Pinout For 74273}
	\label{tab:50-74273}
\end{table}

\section{74283: 4-Bit Binary Full Adder}

This device contains a 4-bit adder with carry-in and carry-out bits. The 74283 device in \LE uses the wiring connections indicated in Table \ref{tab:50-74283}.

\begin{table}[H]
	\sffamily
	\newcommand{\head}[1]{\textcolor{white}{\textbf{#1}}}		
	\begin{center}
		\rowcolors{2}{gray!10}{white} % Color every other line a light gray
		\begin{tabular}{rl} 
			\rowcolor{black!75}
			\head{Logisim Label} & \head{Function}  \\
			Output 1: $ \sum $2  & Sum, bit 2       \\
			Input 2: B2          & Operand B, bit 2 \\
			Input 3: A2          & Operand A, bit 2 \\
			Output 4: $ \sum $1  & Sum, bit 1       \\
			Input 5: A1          & Operand A, bit 1 \\
			Input 6: B1          & Operand B, bit 1 \\
			Input 7: CIN         & Carry in bit     \\
			Output 9: C4         & Carry out bit    \\
			Output 10: $ \sum $4 & Sum, bit 4       \\
			Input 11: B4         & Operand B, bit 4 \\
			Input 12: A4         & Operand A, bit 4 \\
			Output 13: $ \sum $3 & Sum, bit 3       \\
			Input 14: A3         & Operand A, bit 3 \\
			Input 15: B3         & Operand B, bit 3 \\
		\end{tabular}
	\end{center}
	\caption{Pinout For 74283}
	\label{tab:50-74283}
\end{table}

\section{74377: Octal D-Flipflop with Enable}

This device contains a single 8-bit D-Flipflop with a single clock and enable. The 74377 device in \LE uses the wiring connections indicated in Table \ref{tab:50-74377}.

\begin{table}[H]
	\sffamily
	\newcommand{\head}[1]{\textcolor{white}{\textbf{#1}}}		
	\begin{center}
		\rowcolors{2}{gray!10}{white} % Color every other line a light gray
		\begin{tabular}{rl} 
			\rowcolor{black!75}
			\head{Logisim Label} & \head{Function} \\
			Input 1: nCLKen & On low, enable the clock \\
			Output 2: Q1    & data bit 1 output    \\
			Input 3: D1     & data bit 1 input     \\
			Input 4: D2     & data bit 2 input     \\
			Output 5: Q2    & data bit 2 output    \\
			Output 6: Q3    & data bit 3 output    \\
			Input 7: D3     & data bit 3 input     \\
			Input 8: D4     & data bit 4 input     \\
			Output 9: Q4    & data bit 4 output    \\
			Input 11: CLK   & Clock                \\
			Output 12: Q5   & data bit 5 output    \\
			Input 13: D5    & data bit 5 input     \\
			Input 14: D6    & data bit 6 input     \\
			Output 15: Q6   & data bit 6 output    \\
			Output 16: Q7   & data bit 7 output    \\
			Input 17: D7    & data bit 7 input     \\
			Input 18: D8    & data bit 8 input     \\
			Output 19: Q8   & data bit 8 output    \\
		\end{tabular}
	\end{center}
	\caption{Pinout For 74377}
	\label{tab:50-74377}
\end{table}


%*******************************************************
% New Appendix Here
%*******************************************************
%\chapter{New Appendix Here}
%\label{ap:ch:new_appendix_here}
 
% *********************************************************
% Other Stuff in the Back
% *********************************************************
\cleardoublepage\include{FrontBackmatter/Colophon}
%\cleardoublepage%*******************************************************
% Declaration
%*******************************************************
\refstepcounter{dummy}
\pdfbookmark[0]{Style Guide}{styleguide}
\chapter*{Style Guide}
\thispagestyle{empty}

\begin{itemize}

  \item \textsc{Boolean Expressions}. Use math equations for in-line Boolean expressions and equations.

	\item \textsc{Circuit/Subcircuit Name}. Use  \lstinline[columns=fixed]|Circuit_4|.

  \item \textsc{Figures and Tables}.
  
  \begin{itemize}
    \item Figures and Tables use an [H] specification
    \item Captions go at the end of the table block so it is printed under the table to match figures and listings. (Figures and Listings place the caption under by default)
  \end{itemize}

  \item \textsc{File Names}. File names should be in Typewriter and Emphasis styles: \emph{\texttt{Lab01\_Mux21}}.

	\item \textsc{Library Devices}. Devices (like counters) in regular font but with the italicized library name following, like: Counter (\textit{Memory} library).

	\item \textsc{Gates}. Should be in all caps and typewriter font: \texttt{AND}
  
  \item \textsc{Menu Items}. Use small caps: \textsc{Simulate -> Reset Simulation}.  

  \item \textsc{Numbers}. 

	\begin{itemize}
		\item Numbers use normal font, not math or some other special font.
		\item Spell out numbers up to ten; thus, not ``1s,'' but ones.
		\item Larger numbers with a suffix, like ``s,'' do not use an apostrophe: not ten's, but tens.
	\end{itemize}

  \item \textsc{Pin Names}. Pin names (like \textit{Q1}) are italicized. This is also true for variable names and types of flip-flops.

  \item \textsc{Properties}. Properties (like \textit{Facing}) are italicized.

	\item \textsc{Signals}. Should be italicized in typewriter font: \textit{\texttt{Activate}}

  \item \textsc{Tools}. Tools (like \textit{Poke}) are capitalized and italicized.

  \item \textsc{True/False}. The words \emph{True} and \emph{False} are capitalized and in an \lstinline[columns=fixed]|\emph{}| block.

	\item \textsc{Vocabulary}
	
  \begin{itemize}
	\item ``Flip-flop'' is hyphenated
	\item ``Logisim-evoluation'' is placed in italics with only ``Logisim'' capitalized: \textit{Logisim-evolution}
	\item ``Subcircuit'' is not hyphenated
\end{itemize}	

\end{itemize}



\blankpage
\blankpage
\blankpage
\blankpage

\end{document}
