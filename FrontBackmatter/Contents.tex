%*******************************************************
% Brief Table of Contents
%*******************************************************
%\phantomsection
\refstepcounter{dummy}
%\pdfbookmark[1]{\briefcontentsname}{brieftableofcontents}
%\setcounter{tocdepth}{0} % <-- 2 includes up to subsections in the ToC
%\setcounter{secnumdepth}{0} % <-- 3 numbers up to subsubsections
%\manualmark
%\markboth{\spacedlowsmallcaps{\contentsname}}{\spacedlowsmallcaps{\contentsname}}
\shorttoc{Brief Contents}{0} % Only Chapter Headings 

%*******************************************************
% Table of Contents
%*******************************************************
%\phantomsection
\refstepcounter{dummy}
\pdfbookmark[1]{\contentsname}{tableofcontents}
\setcounter{tocdepth}{2} % <-- 2 includes up to subsections in the ToC
\setcounter{secnumdepth}{3} % <-- 3 numbers up to subsubsections
\manualmark
\markboth{\spacedlowsmallcaps{\contentsname}}{\spacedlowsmallcaps{\contentsname}}
\tableofcontents 
\automark[section]{chapter}
\renewcommand{\chaptermark}[1]{\markboth{\spacedlowsmallcaps{#1}}{\spacedlowsmallcaps{#1}}}
\renewcommand{\sectionmark}[1]{\markright{\thesection\enspace\spacedlowsmallcaps{#1}}}
%*******************************************************
% List of Figures and of the Tables
%*******************************************************
\clearpage

\begingroup 
    \let\clearpage\relax
    \let\cleardoublepage\relax
    \let\cleardoublepage\relax
    %*******************************************************
    % List of Figures
    %*******************************************************    
    %\phantomsection 
    \refstepcounter{dummy}
    \addcontentsline{toc}{chapter}{\listfigurename}
    \pdfbookmark[1]{\listfigurename}{lof}
    \listoffigures

    \vspace{8ex}

    %*******************************************************
    % List of Tables
    %*******************************************************
    %\phantomsection 
    \refstepcounter{dummy}
    \addcontentsline{toc}{chapter}{\listtablename}
    \pdfbookmark[1]{\listtablename}{lot}
    \listoftables
        
    \vspace{8ex}
%   \newpage
    
    %*******************************************************
    % List of Listings
    %*******************************************************      
      %\phantomsection 
    \refstepcounter{dummy}
    \addcontentsline{toc}{chapter}{\lstlistlistingname}
    \pdfbookmark[1]{\lstlistlistingname}{lol}
    \lstlistoflistings 

    \vspace{8ex}
       
    %*******************************************************
    % Acronyms
    %*******************************************************
    %\phantomsection 
    \refstepcounter{dummy}
    \pdfbookmark[1]{Acronyms}{acronyms}
    \markboth{\spacedlowsmallcaps{Acronyms}}{\spacedlowsmallcaps{Acronyms}}
    \chapter*{Acronyms}
    % Notes: alphabitize this list as you create it.
    % Accronyms in this list are not printed unless they show up in the text somewhere
    \begin{acronym}[UMLX]
        \acro{ALU}{Arithmetic Logic Unit}
        \acro{ASCII}{American Standard Code for Information Interchange}
        \acro{ASIC}{Application-Specific Integrated Circuit}
        \acro{BCD}{Binary Coded Decimal}
        \acro{BDD}{Binary Decision Diagram}
        \acro{BIOS}{Basic Input/Output System}
        \acro{BLIF}{Berkeley Logic Interchange Format}
        \acro{CAT}{Computer-Aided Tools}
        \acro{CISC}{Complex Instruction Set Computer}
        \acro{CPU}{Central Processing Unit}
        \acro{DRAM}{Dynamic Random Access Memory}
        \acro{DUT}{Device Under Test}
        \acro{EBCDIC}{Extended Binary Coded Decimal Interchange Code}
        \acro{EDA}{Electronic Design Automation}
        \acro{FSM}{Finite State Machine}
        \acro{HDL}{Hardware Description Language}
        \acro{IC}{Integrated Circuit}
        \acro{IEEE}{Institute of Electrical and Electronics Engineers}
        \acro{KARMA}{KARnaugh MAp simplifier}
        \acro{LED}{Light Emitting Diode}
        \acro{LSB}{Least Significant Bit}
        \acro{LSN}{Least Significant Nibble}
        \acro{MSB}{Most Significant Bit}
        \acro{MSN}{Most Significant Nibble}
        \acro{NaN}{Not a Number}
        \acro{OER}{Open Educational Resource}
        \acro{PCB}{Printed Circuit Board}
        \acro{POS}{Product of Sums}
        \acro{PROM}{Programmable Read-Only Memory}
        \acro{RAM}{Random Access Memory}
        \acro{RISC}{Reduced Instruction Set Computer}
        \acro{ROM}{Read Only Memory}
        \acro{RPM}{Rotations Per Minute}
        \acro{RTL}{Register Transfer Language}
        \acro{SECDED}{Single Error Correction, Double Error Detection}
        \acro{SOP}{Sum of Products}
        \acro{SDRAM}{Synchronized Dynamic Random Access Memory}
        \acro{SRAM}{Static Random Access Memory}
        \acro{TTL}{Transistor-Transistor Logic}
        \acro{USB}{Universal Synchronous Bus}
        \acro{VHDL}{VHSIC Hardware Descriptive Language}
        \acro{VHSIC}{Very High Speed Integrated Circuit}
        \acro{VLIW}{Very Long Instruction Word}
    \end{acronym}                     
\endgroup
