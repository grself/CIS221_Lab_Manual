%*******************************************
% Lab 09: Priority Encoder
%*******************************************
\chapter{Priority Encoder}\label{encode}

\section{Purpose}

Often a circuit will receive data from several sources at one time and there must be a way to prioritize those inputs. This circuit creates a simple priority encoder for nine different inputs. This is a fairly simple circuit but is best explained by building and ``playing around'' with it rather than attempting to understand a printed text; thus, the explanation for this lab is somewhat limited.

\section{Procedure}

Start \LE and create a subcircuit named \lstinline[columns=fixed]|Encoder|. Open that subcircuit and place 12 \texttt{AND} gates as illustrated in Figure \ref{fig:encode-01}.

\begin{figure}[H]
	\centering
	\includegraphics[width=\maxwidth{.95\linewidth}]{gfx/encode-01}
	\caption{AND Gates}
	\label{fig:encode-01}
\end{figure}

The gates have one data bit and these properties:

\begin{itemize}
	\item \textbf{U1}: Five inputs, numbers two, three, and four negated.
	\item \textbf{U2}: Four inputs, numbers two and three negated.
	\item \textbf{U3}: Three inputs, number two negated.
	\item \textbf{U4}: Two inputs, none negated.
	\item \textbf{U5}: Four inputs, numbers two and three negated.
	\item \textbf{U6}: Four inputs, numbers one and two negated.
	\item \textbf{U7-U12}: Two inputs, none negated. 
\end{itemize}

Many of the output signals need to be combined with \texttt{OR} gates and those should be added next, as in Figure \ref{fig:encode-02}. Note: U16 is a \texttt{NOR} (\textit{Gates} library) gate.

\begin{figure}[H]
	\centering
	\includegraphics[width=\maxwidth{.95\linewidth}]{gfx/encode-02}
	\caption{OR Gates Added}
	\label{fig:encode-02}
\end{figure}

This encoder is designed to prioritize nine input lines so nine inputs must be added, as illustrated in Figure \ref{fig:encode-03}.

\begin{figure}[H]
	\centering
	\includegraphics[width=\maxwidth{.95\linewidth}]{gfx/encode-03}
	\caption{Inputs Added}
	\label{fig:encode-03}
\end{figure}

Wiring this circuit is the most challenging part of the build. As illustrated in Figure \ref{fig:encode-04}, the inputs are wired to several different \texttt{AND} gates.

\begin{figure}[H]
	\centering
	\includegraphics[width=\maxwidth{.95\linewidth}]{gfx/encode-04}
	\caption{Wiring the Encoder}
	\label{fig:encode-04}
\end{figure}

Finally, four output ports are added, as illustrated in Figure \ref{fig:encode-05}. 

\begin{figure}[H]
	\centering
	\includegraphics[width=\maxwidth{.95\linewidth}]{gfx/encode-05}
	\caption{Nine-line Priority Encoder}
	\label{fig:encode-05}
\end{figure}

This circuit is designed to output a \acf{BCD} number, so no further conversion is needed to be able to read the highest priority input line. At this point, the circuit is complete and the \textit{poke} tool can be used to change the inputs and observe how that high input bit drives the outputs.

To finish the project, open the \lstinline[columns=fixed]|main| circuit and drop the \lstinline[columns=fixed]|Encoder| on the drawing canvas. Add nine inputs and label them \textit{In1} through \textit{In9}. Place a four-bit output labeled \textit{PriOut} and wire the four outputs through a splitter to that output port. To make it easier to read the \ac{BCD} number, connect a Hex Digit Display (\textit{Input/Output} library) to the four-bit bus between the splitter and output port. The completed \lstinline[columns=fixed]|main| circuit is illustrated in Figure \ref{fig:encode-06}.

\begin{figure}[H]
	\centering
	\includegraphics[width=\maxwidth{.95\linewidth}]{gfx/encode-06}
	\caption{Main Circuit}
	\label{fig:encode-06}
\end{figure}

In Figure \ref{fig:encode-06}, notice that two inputs are selected, \textit{In4} and \textit{In6}. Since \textit{In6} is a higher priority (it is a larger number), the output is set for six and \textit{In4} is ignored.

\subsection{Testing the Circuit}

The circuit is now complete. It should be tested by entering various combinations of inputs and observing that the output always displays the highest numbered input. 

\section{Deliverable}

To receive a grade for this lab, create the Nine-line Priority Encoder circuit as defined in this lab. Be sure the standard identifying information is at the top left of the circuit, similar to this:

\bigskip
% The minipage environment keeps the three lines together - no page break.
\begin{minipage}{\linewidth}
	\begin{verbatim}
	George Self
	Lab 09: Nine-line Priority Encoder
	February 18, 2018
	\end{verbatim}
\end{minipage}
\bigskip

Save the file with this name: \emph{\texttt{Lab09\_Encoder}} and submit that file for grading.